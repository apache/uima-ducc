% Create well-known link to this spot for HTML version
\ifpdf
\else
\HCode{<a name='DUCC_TERMINOLOGY'></a>}
\fi
\chapter{Terminology and Acronyms}

\section{Terms }
    This section defines terms and phrases as used in the context of DUCC. 

\begin{description}
\item[Automatic Service ] An automatic service is a registered service that is started automatically
  by DUCC when the DUCC system is booted.

\item[Dependent service or job ] A dependent service or job is a job or service that specifies one
  or more service endpoint in their job specification. The service or job is dependent upon the
  referenced service being operational before being started by DUCC.

\item[DUCC ] DUCC stands for "Distributed UIMA Cluster Computing."

\item[Implicit service ] An emplicit service is a service that is started externally to DUCC but
  referenced by some dependent service or job. 

\item[Registered service ] A registered service is a service that is registered with DUCC. DUCC
  saves the service specification and fully manages the service, insuring it is running when needed,
  and shutdown when not. DUCC manages the usage of the service and (in a future verseion of DUCC)
  automatically increases and decreases the number of service instances as dictated by demand.

\item[On-Demand Service] An on-demand service is a registered service that is not started when DUCC
  is started. Instead, the service is started when referenced in some job or services service
  dependency, and stopped when the referencing entity exits.

\item[Service Instance ] A service instance is one physical process which runs a CUSTOM or UIMA-AS
  service.

\item[Orchestrator (OR) ] The Orchestrator coordinates all work in the system. All new work enters
  through the orchestrator which guides it through the various DUCC components.

\item[Process Manager (PM) ] The Process Manager coordinates distribution of work among the Agents.

\item[Resource Manager (RM) ] The Resource Manager allocates and schedules physical resources among
  the jobs.

\item[Service Class ] The service classes are

implicit, referring to a service started independently from DUCC, 
submitted, referring to a service submitted as a job to DUCC, and 
registered, referring to a registered DUCC service. 


\item[Service Endpoint ] In DUCC, the service endpoint provides a unique identifier for a service
  and in the case of UIMA-AS services, a well-known address for contacting the service. For CUSTOM
  services, the endpoint is of the form CUSTOM:string where string is any alphanumeric string
  provided by the service owner. For UIMA-AS services, the endpoint is of the form UIMA-AS:queue
  name:ActiveMQ broker URL.

\item[Service Manager (SM)] The Service Manager manages the life-cycles of UIMA-AS and custom
  services. It coordinates registration of services, starting and stopping of services, and ensures
  that services are available and remain available for the lifetime of the jobs.

\item[Agent] DUCC Agent processes run on every node in the system. The Agent receives orders to
  start and stop processes on each node. Agents also monitor nodes, sending heartbeat packets with
  node statistics to interested components (such as the RM and web-server). All Job Driver and Job
  Process processes are managed as children of the agents.

\item[Ducc-mon]  Ducc-mon is the DUCC web-server. All DUCC state of import or interest is presented
  here including job state, cluster state, DUCC daemon state, and visualization of the system.
  Various controlling actions such as canceling jobs, submitting reservations, and administrative
  functions are supported.

\item[Job Driver (JD)]The Job Driver is a thin Java wrapper that encapsulates a Job's Collection
  Reader. The JD executes as a process that is scheduled and deployed by DUCC.

\item[Job Process (JP) ]The Job Process is a thin java wrapper that encapsulates a job's Analysis
  Engine. The JP executes in a process that is scheduled and deployed by DUCC.

\item[Job specification ]The Job Specification is a collection of properties that describe a job. It
  identifies the UIMA components (CR, AE, etc) that comprise the job, and it specifies system-wide
  properties of the job (classpaths, RAM requirements, etc). The properties may be provided as (key,
  value) pairs to the CLI/API, or in a Java propeties file.

\item[Job ] A DUCC job consists of the components required to deploy and execute a UIMA pipeline over
  a computing cluster. It consist of a JD to run the Collection Reader, a set of JPs to run the UIMA
  AEs, and a Job Specification to describe how the parts fit together.

\item[Share Quantum ] In DUCC, a "share quantum" refers to some quantity of memory; for example,
  15GB. The RM schedules resources according to share quanta. The share quantum is the smallest unit
  of memory that can be assigned. See the section describing the Resource Manager for details.

  The terms "share" and "share quantum" are synonymous in DUCC. 

\item[Process ]A process is one physical process executing on a machine in the DUCC cluster. DUCC
  jobs are comprised of one or more processes (JDs and JPs).

  From the Resource Management view, a process is comprised of one or more share quanta. 

\item[Weighted Fair Share ] The Weighted Fair Share calculation is used to apportion resources in a
  "fair" manner to the outstanding work in the system. To account for some work being more
  "important" than others, a weighting factor may be applied to bias the fair-share calculations in
  favor of such work.

  See the Resource Manager section for more details on Weighted Fair Share in DUCC. 

\item[Work Items ] A work item is one unit of work to be completed in a single DUCC process. It is
  usually initiated by the submission of a single CAS from the CR to a UIMA service. It could be
  thought of as a single "question" to be answered by a UIMA analytic. Usually each DUCC JP executes
  many work items per job.
\end{description}

\section{Acronyms}
This section defines acronims as used in the context of DUCC. 

\begin{description}
\item[AE:] UIMA Analysis Engine 
\item[CAS:] UIMA Common Analysis Structure 
\item[CC:] CAS Consumer 
\item[CM:] UIMA CAS Multiplier 
\item[CR:] UIMA Collection Reader 
\item[DUCC:] Distributed UIMA Cluster Computing 
\item[JD:] Job Driver 
\item[JP:] Job Process 
\item[OR:] Orchestrator 
\item[PM:] Process Manager 
\item[RM:] Resource Manager 
\item[SM:] Service Manager 
\item[UIMA:] Unstructured Information Management Architecture (see http://uima.apache.org/) 
\item[UIMA-AS:] UIMA Asynchronous Scaleout (see http://uima.apache.org/doc-uimaas-what.html) 
\end{description}

