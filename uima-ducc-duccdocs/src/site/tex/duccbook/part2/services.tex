
      \section{Overview.} 
      A DUCC service is defined by the following two criteria:
      \begin{itemize}
          \item A service is one or more long-running processes that await requests from
            UIMA pipeline components and return something in response.  These processes
            are usually managed by DUCC but need not be.
          \item A sservice is accompanied by a small program called a ``pinger'' that
            the DUCC Service Manager uses to guage the availability and health of the
            service.  This process must always be supplied to DUCC.
      \end{itemize}

      A service is usually a UIMA-AS service, but DUCC supports any arbitrary process
      as a serive.

      The DUCC service manager implements several high-level functions:
      
      \begin{itemize}
          \item Insure services are available for jobs before allowing the jobs to start. This fail-fast
            prevents unncessary allocation of resources (with potential eviction of healthy processes)
            for jobs that can't run, as well as quick feedback to users that something is amis.
      
          \item Manage the startup and management of services: allocate resources, spawn the
            processes, insure the processes stay alive, handle errors, etc.
      
          \item Report on the state and availablity of services.
       \end{itemize}

      \section{Service Pingers}
      A service pinger is a small program that queries a service on behalf of the
      DUCC Service Manager to:
      \begin{itemize}
        \item Report on the availiability of the service: does it respond to requests?
        \item Report on the healh of the service: is it overload, is it repsonding
          adequiately, etc.
      \end{itemize}
      
      Service pingers are always written in Java and must implement an abstract class,
      {\tt org.apache.uima.ducc.common.AServicePing}.   When a service is deployed by
      DUCC, the Service Manager spawns a DUCC process that instantiates the pinger for
      the service.  On a regular basis, the Service Manager sends a request to the pinger
      to query the service health.

      \subsection{Declaring a Pinger in A Service}

      If your service is a UIMA-AS service, there is no need to create or declare a pinger.  DUCC
      provides a default pinger.  If a CUSTOM pinger is required, it must be declared in the service
      descriptor, and the service must be registered.  See Section ~\ref{DUCC-SERVICES-CLI} for
      details on service registration and the ping directives.      

      \subsection{Implementing a Pinger}
      Pingers must implement the class {\tt org.apache.uima.ducc.common.AServicePing}.  The class
      is shown below in figure ~\ref{ABSTRACT-PINGER}.
      \begin{figure}[H]
\begin{verbatim}
package org.apache.uima.ducc.common;

public abstract class AServicePing
{
    /**
     * Called by the ping driver, to pass in useful things the pinger may want.
     * @param endpoint This is the name of the service endpoint, as passed in
     *                 at service registration.
     */
    public abstract void init(String endpoint)  throws Exception;

    /**
     * Stop is called by the ping wrapper when it is being killed.  Implementors may optionally
     * override this method with conenction shutdown code.
     */
    public abstract void stop();

    /**
     * Returns the object with application-derived health and statistics.
     * @return {@link ServiceStatistcs} This object contains the informaton the service manager and web server require
     *     for correct management and display of the service.
     */
    public abstract ServiceStatistics getStatistics();
    
}
\end{verbatim}
        \caption{} Service Ping Abstract Class
        \label{ABSTRACT-PINGER}

      \end{figure}
      
      The ServiceStatistics class defines these methods:
      \begin{description}
        \item[ServiceStatistics(boolean alive, boolean healthy, String info)] This is the constructor.
          \begin{description}
            \item[boolean alive] Set this to ``true'' if the service is responsive.  If a pinger responds
              ``false'' (or does not respond), the Service Manager will assume the service is unavailable
              and will not allow jobs dependent on this service to start.  (Dependent jobs that are already
              started are allowed to continue, but are annoated in the web server, such that developers
              will know the job may not be functioning because of the service.)
            \item[boolean healthy] The pinger may perform analysis on the service to determine whether
              the service is ``healthy'' or not.  This is strictly subjective and is used by the
              Service Manager only for reporting to the web server.
            \item[String info] This is any string in any format.  The pinger sets health and availability
              data into it for display in the webserver.  For example, the default UIMA-AS pinger sets
              ActiveMQ service statistics into this string.)
          \end{description}
          
          \item[void setAlive(boolean alive)] Set the ``aliveness'' of the service.

          \item[void setUnhealthy(boolean val)] Set the ``healthiness'' of the service.
            
          \item[void setInfo(String info)] update the service information string.
      \end{description}

      A sample CUSTOM pinger is shown in figure ~\ref{CUSTOM-PINGER} below. The pinger assumes a simple
      service port that, on connection, returns an integer.  If the connect and read of the integer succeds,
      the ping is marked successful. 

      \begin{figure}[H]
\begin{verbatim}
import java.io.DataInputStream;
import java.io.InputStream;
import java.net.Socket;
import org.apache.uima.ducc.common.AServicePing;
import org.apache.uima.ducc.common.ServiceStatistics;

public class CustomPing
    extends AServicePing
{
    String host;
    String port;
    public void init(String endpoint) throws Exception {
        String[] parts = endpoint.split(":");
        host = parts[1];
        port = parts[2];
    }

    public void stop()  {  }

    public long readLong(DataInputStream dis) throws Exception {
        return Long.reverseBytes(dis.readLong());
    }

    public ServiceStatistics getStatistics() {
        ServiceStatistics stats = new ServiceStatistics(false, false,"<NA>");
        try {
            Socket sock = new Socket(host, Integer.parseInt(port));
            DataInputStream dis = new DataInputStream(sock.getInputStream());

            long stat1 = readLong(dis); long stat2 = readLong(dis); 
            long stat3 = readLong(dis); long stat4 = readLong(dis);

            stats.setAlive(true);  stats.setHealthy(true);
            stats.setInfo(  "S1[" + stat1 + "] S2[" + stat2 + "] S3[" + stat3 + "] S4[" + stat4 + "]" );
        } catch ( Throwable t) {
        	t.printStackTrace();
            stats.setInfo(t.getMessage());
        }
        return stats;        
    }
}
\end{verbatim}
        \caption{} Sample UIMA-AS Service Pinger
        \label{CUSTOM-PINGER}

      \end{figure}
      

     \section{Service Types.}
      DUCC supports two types of services: UIMA-AS and CUSTOM:
      
      \begin{description}
          \item[UIMA-AS] This is a "normal" UIMA-AS service. DUCC fully supports all aspects of UIMA-AS
            services with minimal effort from developers.  A default ``pinger'' is supplied by DUCC
            for UIMA-AS services.  (It is legal to define a CUSTOM pinger for a UIMA-AS service,
            however.)
            
          \item[CUSTOM] This is any arbitrary service.  Developers must provide a CUSTOM pinger
            and declare it int he service registration.            
      \end{description}

      \section{Service Endpoints.} Services are referenced by a specifier called a service
      endpoint. The service endpoint is a formatted string used to uniquely identify each
      service, and to supply contact information to the pingers.  A service endpoint
      is of the form 
\begin{verbatim}
      <service-type>:<unique id and contact information>
\end{verbatim}
      
      The {\em service-type} must be either UIMA-AS or CUSTOM.
      
      The {\em unique id and contact information} is any string needed to insure the service is
      uniquely name.  This string is passsed to the service pinger and must contain sufficient
      information for the pinger to contact the service.  For UIMA-AS services, service endpoint is
      inferred by the CLI by inspection of the UIMA XML descriptor.  For reference: the UIMA-AS
      service endpoint is of the form:
\begin{verbatim}
      UIMA-AS:queue-name:broker-url
\end{verbatim}
      where {\em queue-name} is the name of the ActiveMQ queue used by the service, and {\em broker-url}
      is the ActiveMQ broker URL.
      
      \section{Dependent and Pre-Requisite Services and Jobs.} A {\em dependent service} is a
      service which is dependent on at least one other service to perform it's function. A {\em
        dependent job} is a job which is dependent on at least one service to perform it's function.

      An {\em independent service} service is a service which is required by another job or
      service. (Note that there are no independent jobs.)

      \section{Service Classes.} Services may be started externally to DUCC, explicitly through
      DUCC as a job, or as registered services. These form three natural classes of services with
      slightly different management characteristics.

      \subsection{Implicit Services.} An implicit service is started externally to DUCC and discovered by DUCC only
      when it is referenced by a job's service-dependency parameter. On submission of a job with a
      dependency on an implicit service, the SM sets up a "ping" thread that check if the service
      exists at the endpoint. If so, the SM adds the service to its list of known services and marks
      the job "ready to schedule". If the service is a UIMA-AS service the SM establishes a monitor
      thread on the queue for reporting purposes. The service is monitored throughout the lifetime of
      the job. If the service should stop responding, its state is updated as "not-responding" but the
      job is allowed to continue as DUCC cannot tell if the job is still using it or not, or if the
      outage is temporary. If the job is a CUSTOM service, the service owner may specifiy custom code
      to run in the ping thread; for CUSTOM services, this same code is used to run both ping and
      monitor functions.
      
      When the job exits, a timer is set and DUCC continues to monitor the service against the
      possibility that subsequent jobs will need it. Once the last job using the service has exited
      and the service timer expired, the SM stops the monitors and purges the service from its
      records.
      
      \subsection{Submitted Services.} A submitted service is a service that is submitted to DUCC as a job. A
      submitted service is essentially a normal DD-style job (a job in which the user supplies the
      full UIMA-AS DD), but without a Collection Reader. Because DUCC is managing this service it can
      provide more support than for implicit services.
      
      Submitted services can be dependent upon other services. When such a service enters the system,
      DUCC verifies it's pre-requisite services. When (or if) all pre-requisite services are availble
      DUCC marks the new service "ready to schedule". The lifecycle of the service is monitored so
      that dependent services and jobs are marked "ready to schedule" only after the submitted service
      has completed its initialization phase. A ping thread and queue monitor are also started against
      the newly submitted service. If the submitted service is unable to successfully initialize,
      services and jobs that are dependent on it are marked "not runnable" and the DUCC Orchestrator
      cancels them.
      
      DUCC manages the lifecycle of submitted services, but because they are submitted by entities
      other than DUCC, the SM performs no additional management for them. When a submitted service is
      canceled by its owner, DUCC stops the ping and queue monitors. Any jobs or services dependent on
      it are allowed to continue until they complete or fail due to unavailability of the service.
      
      \subsection{Registered Services.} Registered services are fully managed by DUCC. A service is
      registered with DUCC using the CLI to provide the full job specification of the service, the
      initial number of instances of the service, and whether the service should be automatically
      started when DUCC itself is started. Registered services started when DUCC is started are
      called automatic services.  Registered services that are started only when referenced by other
      dependent jobs or services are called on-demand services. The service is registered with the
      submitter's credentials and is run with that user's credentials when it is started.

      \todo Fix and properly place this paragraph.
          Ping and monitor threads are started. Jobs and other services may use these services in the same
          manner as submitted services. If an automatic service instance should die or be canceled out of
          the scope of the SM, the SM will restart the instance, maintaining the registered number of
          instances at all time. Automatic services are not terminated when their dependent jobs/services
          exit; they're termanted only when DUCC itself is terminated, or by use of the service stop
          command.

      There are several subclasses of Registered Services:
      \begin{description}

        \item[Automatic Services] An automatic service is a registered service that is flagged to be
          automatically started when the DUCC system is started. When DUCC is started, the SM checks the
          service registry for all service that are marked for automatic startup. The SM submits the
          registered service specification on behalf of its owner. Each such submission is for a single
          service instance.  If found, the SM repeatedly submits the specification until the registered
          number of instances is reached.
          
        \item[On-Demand Services] An on-demand service is a registered service that is started only when
          referenced by the service-dependency of another job or service. f the service is already
          started, the dependent job/service is marked ready to schedule as indicated above. If not, the
          service registry is checked and if a start-on-demand service with an endpoint matching the
          service-dependency is found, DUCC submits the service on behalf of the service owner (in the
          same manner as for automatic servic establishing the registered number of service instances, a
          ping thread, and a monitor). When the service has completed initialization the dependent
          job/service is marked ready to schedule. If the on-demand service cannot be found in the
          registery, the referring entity is marked not-startable and the DUCC Orchestrator cancels it.
          
          Subsequent jobs and services that reference the on-demand service will use the started
          instances.  When the last job/service that references the on-demand service exits, a
          (configurable) timer is established to keep the service alive for a while (in anticipation that
          it will be needed again soon.)  When the keep-alive timer exipires, and there are no more
          dependent jobs/services, the on-demand service is automatically stopped to free up its resources
          for other work.

          \item[External Services] External services consist of only a ping thread.  The service
            itself is not managed in any way by DUCC.  This is useful for managing dependencies
            on services that are not under DUCC control: DUCC can detect and report on the health
            of these services and take appropriate actions on dependent jobs if the services
            are not responsive.
      \end{description}
          
    \subsection{Registered Service Management.} The CLI for registered services provides several functions:

    \begin{description}
        \item[Register] Register files a service specification with the SM. The service may optionally
          be started as part of registration. The service definition and state is persisted over system
          restarts and is deleted only with the Unregister function.
          
        \item[Unregister] Unregister removes the service specification. The service is stopped if it is
          started and not busy. (Note that if the service is busy, jobs and services that are dependent
          on it may subsequently fail.)
          
        \item[Modify] Modify allows dynamic update of some parameters of registered services:
            \begin{itemize}
              \item Automatic and On-Demand state.
              \item The minimum number of service instances to start when the service is started.  
            \end{itemize}

        \item[Start] Start submits the service specification to the DUCC Orchestrator (repeatedly,
          until the correct number of instances are started). If the service is explicitly started
          with the start CLI, the service continues to run even after the last reference is gone,
          regardless of whether it is automatic or on-demand. Start is also used to increase the
          number of running instances of a service. The registry may be optionally updated to
          reflect the new number of started instances.
          
        \item[Stop] Stop stops the instances for a registered service. The registry may be
          optionally updated to reflect the new number of instances that are still running.

        \item[Query] A CLI-based query is supplied to report on all services known to DUCC, their
          states, their instances, their dependent jobs, and performance statistics for the service.
    \end{description}
        
