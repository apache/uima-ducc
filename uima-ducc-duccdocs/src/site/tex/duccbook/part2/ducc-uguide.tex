\chapter{Command Line Interface}

    \paragraph{The DUCC Job Descriptor}
    The DUCC Job Descriptor includes properties to enable automated management and scale-out 
    over large computing clusters.  The job descriptor includes
    \begin{itemize}
      \item References to the various UIMA components required by the job (CR, CM, AE, CC, and maybe DD)
      \item Scale-out requirements: number of processes, number of threads per process, etc
      \item Environment requirments: log directory, working directory, environment variables, etc,
      \item JVM paramenters
      \item Scheduling class
      \item Error-handling preferences: acceptable failure counts, timeouts, etc
      \item Debugging and monitoring requirements and preferences
    \end{itemize}
  

    The Command Line Interface is provided in several forms:

    \begin{enumerate}
      \item A Java executable jar.
      \item A script.
      \item Direct invocation of each commands's {\tt main}.
    \end{enumerate}

    When using the executable jars and scripts the full execution environment is estableshed
    silently.  When directly invoking a command's {\tt main} one must set the java {\tt CLASSPATH} to
    specify the appropriate jar for the command, as described in subsequent sections.

    The following commands are provided:
    \begin{description}
    \item[ducc\_submit] Submit a job for ececution.
    \item[ducc\_cancel] Cancel a job in progress.
    \item[ducc\_reserve] Request a reservation of full or partial machines.
    \item[ducc\_unreserve] Cancel a reservation.
    \item[ducc\_monitor] Monitor the progress of a job that is already submitted.
    \item[ducc\_process\_submit] Submit an arbitrary process (managed reservation) for execution.
    \item[ducc\_process\_cancel] Cancel an arbitrary process.
    \item[ducc\_service\_submit] Submit a (non-registered) service instance for execution.
    \item[ducc\_service\_cancel] Cancel a (non-registered) service instance.
    \item[ducc\_services] Register, unregister, start, stop, modify, and query a service.
    \end{description}
    
    The next section describes these commands in detail.

    %% These all input sections
    % Create well-known link to this spot for HTML version
\ifpdf
\else
\HCode{<a name='DUCC_CLI_SUBMIT'></a>}
\fi

    \section{ducc\_submit}
    \label{sec:cli.ducc-submit}
       The source for this section is ducc\_duccbook/documents/part-user/cli/submit.xml.
       \paragraph{Description:}
           The submit CLI is used to submit work for execution by DUCC. DUCC assigns a unique id to the
           job and schedules it for execution. The submitter may optionally request that the progress of
           the job is monitored, in which case the state of the job as it progresses through its
           lifetime is printed on the console.
       \paragraph{Usage:}
           \begin{description}
             \item[Executable Jar] java -jar \ducchome/lib/uima-ducc-submit.jar {\em options}
             \item[Script wrapper] \ducchome/bin/ducc\_submit {\em options}
             \item[Java Main]      java -cp \ducchome/lib/uima-ducc-submit.jar org.apache.uima.ducc.cli.DuccJobSubmit {\em options}
           \end{description}

        \paragraph{Options:}
           \begin{description}

           \item[--all\_in\_one $<$local $|$ remote $>$]
               Run driver and pipeline in single process.  If {\em local} is specified, the
               process is executed on the local machine, for example, in the current Eclipse session.
               If {\em remote} is specified, the jobs is submitted to DUCC as a {\em manged reservation}
               and run on some (presumably larger) machine allocated by DUCC.

           \item[--cancel\_job\_on\_interrupt].  If the job is started with --wait\_for\_completion, this
             option causes the job to be canceled if the submit command is terminated,
             e.g., with CTL-C. If --cancel\_job\_on\_interrupt is not
             specified, the job monitor will be terminated but the job will continue to run.

             If --wait\_for\_completion is not specified this option is ignored. 

           \item[--classpath] The CLASSPATH used for the job.  If specified, this is used
             for both the driver and each process. If not specified the classpath found by the underlying
             {\tt DuccJobSubmit.main()} method is used.

           \item[--classpath\_order {[UserBeforeDucc | DuccBeforeUser]} ]
             When DUCC deploys a process, set the user-supplied classpath before DUCC-supplied
             classpath, or the reverse.
             
           \item[--debug] Enable debugging messages. This is primarily for debugging DUCC itself.

           \item[--description {[text]}] The text is any string used to describe the job. It is
             displayed in the Web Server.             

           \item[--driver\_attach\_console] If specified, redirect remote job driver stdout and stderr
             to the local submitting console.

           \item[--driver\_classpath {[classpath]}]
             This is the classpath for the Job Driver, necessary for DUCC to find the Collection Reader. 

           \item[--driver\_debug {[debugger-address]}] Append JVM debug flags to the JVM arguments
             to start the JobDriver in remote debug mode.  The remote process debugger will attempt
             to contact the specified port. The address is of the form {\tt host:port}.

           \item[--driver\_descriptor\_CR {[descriptor.xml]} ] This is the XML descriptor for the
             Collection Reader.  This descriptor is a resource that is searched for in the CLASSPATH
             and data path as described in the ~\hyperref[par:cli.submit.notes]{notes below}.

           \item[--driver\_descriptor\_CR\_overrides {[list]} ]
             
             This is the Job Driver collection reader configuration overrides. They are specified as 
             name/value pairs in a comma-delimeted list. For example: 
             \begin{verbatim}
--driver\_descriptor\_CR\_overrides name1=value1,name2=value2...
             \end{verbatim}
             
             
%           \item[--driver\_environment {[list]} ]
%
%             This specifies environment parameters for the Job Driver. If present, they are added to the 
%             Job Driver's environment as the process is spawned. It must be a quoted, blank-delimeted 
%             lsit of name-value pairs. For example: 
%             \begin{verbatim}
%"TERM=xterm DISPLAY=:1.0" 
%             \end{verbatim}
%             
%             Note: On Secure Linux systems, the environemnt variable 
%             LD\_LIBRARY\_PATH may not be passed to the user's program. If it is 
%             necessary to pass LD\_LIBRARY\_PATH to the JP or JD processes, it must be 
%             specified as DUCC\_LD\_LIBRARY\_PATH. Ducc (securely) passes this as 
%             LD\_LIBRARY\_PATH, after the JP or JD has assumed the user's identity. For 
%             example: 
%             \begin{verbatim}
%"--process\_environment TERM=xterm DISPLAY=:1.0 DUCC\_LD\_LIBRARY\_PATH=/my/own/
%            \end{verbatim}

           \item[--driver\_exception\_handler {[classname]}] This specifies a developer-supplied
             exception handler for the Job Driver.  It must
             implement org.apache.uima.ducc.common.jd.plugin.IJdProcessExceptionHandler.

           \item[--driver\_jvm\_args {[list]} ]

             This specifes extra JVM arguments to be provided to the Job Driver process. It is a blank delimeted 
             list of strings. Example: 
             \begin{verbatim}
--driver\_jvm\_args -Xmx100M -Xms50M 
             \end{verbatim}
             
           \item[--driver\_memory\_size {[size-in-GB]} ]

             This specifies the size of memory for the Job Driver, in GB. Example: 
             \begin{verbatim}
--driver\_memory\_size 16 
             \end{verbatim}

           \item[--environment {[env vars]}] Blank-delimeted list of environment variables. If
             specified, this is used for all DUCC processes in the job.Example:
\begin{verbatim}
             --environment "TERM=xterm DISPLAY=me.org.net:1.0". 
\end{verbatim}
             
             Note: On Secure Linux systems, the environemnt variable 
             LD\_LIBRARY\_PATH may not be passed to the user's program. If it is 
             necessary to pass LD\_LIBRARY\_PATH to the JP or JD processes, it must be 
             specified as DUCC\_LD\_LIBRARY\_PATH. Ducc (securely) passes this as 
             LD\_LIBRARY\_PATH, after the JP or JD has assumed the user's identity. For 
             example: 
             \begin{verbatim}
-environment TERM=xterm DISPLAY=:1.0 DUCC\_LD\_LIBRARY\_PATH=/my/own/path
            \end{verbatim}

           \item[--help ]

             Prints the usage text to the console. 

           \item[--jvm {[path-to-java]}  ]

             States the JVM to use. If not specified, the same JVM used by the Agents is used.  This is
             the full path to the JVM, not the JAVA\_HOME.
             Example: 
\begin{verbatim}
--jvm /share/jdk1.6/bin/java 
\end{verbatim}
             
           \item[--log\_directory {[path-to-log directory]} ]

             This specifies the path to the directory for the user logs. If not specified, the default is the 
             user's home directory. Example: 
             \begin{verbatim}
--log\_directory /home/bob 
             \end{verbatim}
             
             Within this directory DUCC creates a subdirectory for each job, using the numerical 
             ID of the job. The format of the generated log file names as described
             \hyperref[chap:job-logs]{here}.
             
             Note: Note that --log\_directory specifies only the path to a directory where 
             logs are to be stored. In order to manage multiple processes running in multiple 
             machines DUCC, sub-directory and file names are generated by DUCC and may 
             not be directly specified. 

           \item[--process\_attach\_console] If specified, redirect remote process (as
             opposed to driver) stdout and stderr to the local submitting console.

           \item[--process\_classpath {[ClASSPATH]} ]

             This specifies the Java CLASSPATH to use in each Job Process (JP) and must be 
             specified. Example: 
             \begin{verbatim}
--process\_classpath a.jar:b.jar 
             \end{verbatim}
             
           \item[--process\_DD {[DD descriptor]}  ]

             This specifies a UIMA Deployment Descriptor for the job processes for DD-style jobs. 
             This is mutually exclusive with --process\_descriptor\_AE, --process\_descriptor\_CM, 
             and --process\_descriptor\_CC. This descriptor is a resource that is searched for in the 
             CLASSPATH and data path as described in the ~\hyperref[par:cli.submit.notes]{notes below}.
             For example:
             \begin{verbatim}
--process\_DD /home/billy/resource/DD\_foo.xml 
             \end{verbatim}

           \item[--process\_debug {[debugger-address]}] Append JVM debug flags to the JVM
             arguments to start the Job Process in remobe debug mode.  The remote process will
             start its debugger and attempt to contact the (Eclipse) debugger on the specified port.
             The address is of the form {\tt host:port}.
             
           \item[--process\_deployments\_max {[integer]} ]

             This specifies the maximum nunber of Job Processes to deploy at any given time. If not 
             specified, DUCC will attempt to provide the largest number of processes, within the 
             constraints of fair\_share scheduling and the number of pending work items still to be done 
             in the job. Example:
             \begin{verbatim}
--process\_deployments\_max 66 
             \end{verbatim}


           \item[--process\_descriptor\_AE {[descriptor]}  ]

             This specifies Analysis Engine descriptor to be deployed in the Job Processes. This 
             descriptor is a resource that is searched for in the CLASSPATH and data path as described 
             in the ~\hyperref[par:cli.submit.notes]{notes below}.
             It is mutually exclusive with --process\_DD For example: 
             \begin{verbatim}
--process\_descriptor\_AE /home/billy/resource/AE\_foo.xml 
             \end{verbatim}


           \item[--process\_descriptor\_AE\_overrides {[list]}  ]

             This specifies AE overrides. It is a comma-delimeted list of name/value pairs. Example: 
             \begin{verbatim}
--process\_descriptor\_AE\_Overrides name1=value1,name2=value2 
             \end{verbatim}
             
           \item[--process\_descriptor\_CC {[descriptor]}  ]

             This specifies the CAS Consumer descriptor to be deployed in the Job Processes. This 
             descriptor is a resource that is searched for in the CLASSPATH and data path as described 
             in the ~\hyperref[par:cli.submit.notes]{notes below}.
             It is mutually exclusive with --process\_DD For example: 
             \begin{verbatim}
--process\_descriptor\_CC /home/billy/resourceCCE\_foo.xml 
             \end{verbatim}
             
           \item[--process\_descriptor\_CC\_overrides {[list]}  ]

             This specifies CC overrides. It is a comma-delimeted list of name/value pairs. Example: 
             \begin{verbatim}
--process\_descriptor\_CC\_overrides name1=value1,name2=value2 
             \end{verbatim}
             
           \item[--process\_descriptor\_CM {[descriptor]} ]

             This specifies the CAS Multiplier descriptor to be deployed in the Job Processes. This 
             descriptor is a resource that is searched for in the CLASSPATH and data path as described 
             in the ~\hyperref[par:cli.submit.notes]{notes below}.
             It is mutually exclusive with --process\_DD For example: 
             \begin{verbatim}             
--process\_descriptor\_CM /home/billy/resource/CM\_foo.xml 
             \end{verbatim}

           \item[--process\_descriptor\_CM\_overrides {[list]}  ]

             This specifies CM overrides. It is a comma-delimeted list of name/value pairs. Example: 
             \begin{verbatim}
--process\_descriptor\_CM\_overrides name1=value1,name2=value2 
\end{verbatim}
             
           \item[--process\_environment {[environment]} ]

             This specifies environment parameters for the Job Processes. If present, they are added 
             to the Job Process environment as the process is spawned. It must be a quoted, blankdelimeted 
             lsit of name-value pairs. For example: 
             \begin{verbatim}

--process\_environment TERM=xterm DISPLAY=:1.0
             \end{verbatim}
  
             Note: On Secure Linux systems, the environemnt variable 
             LD\_LIBRARY\_PATH may not be passed to the user's program. If it is 
             necessary to pass LD\_LIBRARY\_PATH to the JP or JD processes, it must be 
             specified as DUCC\_LD\_LIBRARY\_PATH. Ducc (securely) passes this as 
             LD\_LIBRARY\_PATH, after the JP or JD has assumed the user's identity. For 
             example: 

             \begin{verbatim}
--process\_environment TERM=xterm DISPLAY=:1.0 DUCC\_LD\_LIBRARY\_PATH=/my/own/
             \end{verbatim}

           \item[--process\_failures\_limit {[integer]} ]

             This specifies the maximum number of individual Job Process (JP) failures that are to be 
             tolerated before killing the job. The default is 15. If this limit is exceeded over the lifetime 
             of a job DUCC terminates the entire job. 
             \begin{verbatim}
--process\_failures\_limit 23
\end{verbatim}
                          
           \item[--process\_initialization\_failures\_cap {[integer]} ] This specifies the maximum
             number of independent Job Process initialization failures (i.e.  System.exit(), kill
             -9, Java Exceptions, etc.) before the number of Job Processes is capped at the number
             in state Running currently.  One this limit is reached, the system will allow processes
             which are already running to continue, but will assign no new processes to the job.
             The default is 99. Example:
             \begin{verbatim}
--process\_initialization\_failures\_cap 62 
             \end{verbatim}
             
             Note that the job is NOT killed if there are processes that have passed initialization and are 
             running. If this limit is reached, the only action is to not start new processes for the job. 

           \item[--process\_initialization\_time\_max {[integer]}] This is the maximimum time a process
             is allowed to remain in the ``initializing'' state, before DUCC terminates it.

           \item[--process\_jvm\_args {[list]} ] This specifies additinal arguments to be passed to
             the Job Process JVM as a blank-delimeted list of strings. Example:
             \begin{verbatim}
--process\_jvm\_args -Xmx400M -Xms100M 
             \end{verbatim}
             
           \item[--process\_memory\_size {[size]} ] This specifies the maximum amount of RAM in GB
             to be allocated to each Job Process.  This value is used by the Resource Manager to
             allocate resources.

           \item[--process\_per\_item\_time\_max {[integer]} ] This specifies the maximum time in
             minutes that the Job Driver will wait for a Job Processes to process a CAS. If a
             timeout occurs the process is terminated and the CAS marked in error (not retried). If
             not specified, the default is 1 minute. Example:
             \begin{verbatim}
--process\_per\_item\_time\_max 60 
             \end{verbatim}
             
           \item[--process\_thread\_count {[integer]} ] This specifies the number of threads per
             process to be deployed. It is used by the Resource Manager to determine how many
             processes are needed, by the Agent to determine howmany threads to spawn, and by the
             Job Driver to determine how many CASs to dispatch. If not specified, the default is
             4. Example:
             \begin{verbatim}
--process\_thread\_count 7 
             \end{verbatim}
             
           \item[--scheduling\_class {[classname]} ] This specifies the name of the scheuling class
             the RM will use to determine the resource allocation for each process. The names of the
             classes are installation dependent. If not specified, the default is taken from the
             global DUCC configuration ducc.properties.  Example:
             \begin{verbatim}
--schedling\_class normal 
             \end{verbatim}
          

           \item[--service\_dependency{[list]}] This specifies a comma-delimeted list of services the job
             processes are dependent upon. Service dependencies are discussed in detail
             \hyperref[sec:service.endpoints]{here}. Example:
\begin{verbatim}
--service_dependency UIMA-AS:RandomSleepAE:tcp:bluej682:61616 UIMA-AS:OtherEp:tcp:bluej123:123 
\end{verbatim}

           \item[--specifiecaiton {[file]}  ]

             All the parameters used to submit a job may be placed in a standard Java properties file. 
             This file may then be used to submit the job (rather than providing all the parameters 
             directory to submit). 

             For example, 
\begin{verbatim}
ducc\_submit --specification job.props 
\end{verbatim}

             where the job.props contains: 
\begin{verbatim}
working_directory                   = /home/bob/projects/ducc/ducc_test/test/bin 
process_failures_limit              = 20 
driver_descriptor_CR                = org.apache.uima.ducc.test.randomsleep.FixedSleepCR 
driver_environment                  = DUCC_LD_LIBRARY_PATH=/a/other/bogus/path 
process_environment                 = AE_INIT_TIME=10000 DUCC_LD_LIBRARY_PATH=/a/bogus/path 
driver_classpath                    = /home/bob/duccapps/ducky_driver.jar 
log_directory                       = /home/bob/ducc/logs/ 
process_thread_count                = 1 
driver_descriptor_CR_overrides      = jobfile:../simple/jobs/1.job,compression:10 
process_initialization_failures_cap = 99 
process_per_item_time_max           = 60 
driver_jvm_args                     = -Xmx500M 
process_descriptor_AE               = org.apache.uima.ducc.test.randomsleep.FixedSleepAE 
process_classpath                   = /home/bob/duccapps/ducky_process.jar 
description                         = ../simple/jobs/1.job[AE] 
process_jvm_args                    = -Xmx100M -DdefaultBrokerURL=tcp://localhost:61616 
scheduling_class                    = normal 
process_memory_size                 = 15 
\end{verbatim}

             Note that properties in a specifications file may be overridden by other command-line
             parameters, as discussed \hyperref[chap:cli]{here}.

           \item[--timestamp ]

             If specified, messages from the submit process are timestamped. This is intended primarily 
             for use with a monitor with --wait\_for\_completion. 

           \item[--wait\_for\_completion ]             
             If specified, the submit command monitors the job and prints periodic
             state and progress information to the console.  When the job completes, the monitor
             is terminated and the submit command returns.
             
           \item[--working\_directory ]             
             This specifies the working directory to be set by the Job Driver and Job Process processes. 
             If not specified, the current directory is used.
  \end{description}
             
  \paragraph{Notes:}
  \phantomsection\label{par:cli.submit.notes}
  When searching for UIMA XML resource files such as descriptors, DUCC searches both the 
  classpath and the data path according to the following rules: 
  
  \begin{enumerate}
  \item If the resource ends in .xml it is assumed the resource is a file and the path is either an 
    absolute path or a path relative to the specified working directory. If the file is not found 
    the search exits and the job is terminated. 
    
  \item If the resource does not end in .xml, DUCC creates a path by replacing the "." 
    separators with "/" and appending ".xml". It then searches two places: 
    \begin{enumerate}
    \item The user's CLASSPATH as a file (that is, not in a jar), and 
    \item In the jar files provided in the user's CLASSPATH. 
    \end{enumerate}
    If the resource is found in either place the search is successful. Otherwise the search 
    fails and the job is terminated. 
    
    The resource search-order rules apply to all of the following submit parameters: 
    � --driver\_descriptor\_CR 
    � --process\_descriptor\_AE 
    � --process\_descriptor\_CC 
    � --process\_descriptor\_CM 
  \end{enumerate}

    % 
% Licensed to the Apache Software Foundation (ASF) under one
% or more contributor license agreements.  See the NOTICE file
% distributed with this work for additional information
% regarding copyright ownership.  The ASF licenses this file
% to you under the Apache License, Version 2.0 (the
% "License"); you may not use this file except in compliance
% with the License.  You may obtain a copy of the License at
% 
%   http://www.apache.org/licenses/LICENSE-2.0
% 
% Unless required by applicable law or agreed to in writing,
% software distributed under the License is distributed on an
% "AS IS" BASIS, WITHOUT WARRANTIES OR CONDITIONS OF ANY
% KIND, either express or implied.  See the License for the
% specific language governing permissions and limitations
% under the License.
% 
% Create well-known link to this spot for HTML version
\ifpdf
\else
\HCode{<a name='DUCC_CLI_CANCEL'></a>}
\fi
    \section{ducc\_cancel}
    \label{sec:cli.ducc-cancel}

    \paragraph{Description:}
    The cancel CLI is used to cancel a job that has previously been submitted but which has not yet 
    completed. 

    \paragraph{Usage:}
    \begin{description}
    \item[Script wrapper] \ducchome/bin/ducc\_cancel {\em options}
    \item[Java Main]      java -cp \ducchome/lib/uima-ducc-cli.jar org.apache.uima.ducc.cli.DuccJobCancel {\em options}
    \end{description}

    \paragraph{Options:}
    \begin{description}
        \item[$--$debug ]          
          Prints internal debugging information, intended for DUCC developers or extended problem determination.                    
        \item[$--$id {[jobid]}]
          The ID is the id of the job to cancel. (Required)
        \item[$--$reason {[quoted string]}]
          Optional. This specifies the reason the job is canceled for display in the web server. Note that
          the shell requires a quoted string.  Example:
\begin{verbatim}
ducc_cancel --id 12 --reason "This is a pretty good reason."
\end{verbatim}
        \item[$--$dpid {[pid]}]
          If specified only this DUCC process will be canceled.  If not
          specified, then entire job will be canceled.  The {\em pid} is the DUCC-assigned process ID of the
          process to cancel.  This is the ID in the first column of the Web Server's job details page, under
          the column labeled ``Id''.
        \item[$--$help]
          Prints the usage text to the console. 
        \item[$--$role\_administrator] The command is being issued in the role of a DUCC administrator.
          If the user is not also a registered administrator this flag is ignored.  (This helps to
          protect administrators from accidentally canceling jobs they do not own.)
     \end{description}
        
    \paragraph{Notes:}
    None.


    % 
% Licensed to the Apache Software Foundation (ASF) under one
% or more contributor license agreements.  See the NOTICE file
% distributed with this work for additional information
% regarding copyright ownership.  The ASF licenses this file
% to you under the Apache License, Version 2.0 (the
% "License"); you may not use this file except in compliance
% with the License.  You may obtain a copy of the License at
% 
%   http://www.apache.org/licenses/LICENSE-2.0
% 
% Unless required by applicable law or agreed to in writing,
% software distributed under the License is distributed on an
% "AS IS" BASIS, WITHOUT WARRANTIES OR CONDITIONS OF ANY
% KIND, either express or implied.  See the License for the
% specific language governing permissions and limitations
% under the License.
% 
%%
%% ducc_monitor is currently unsupported. Perhaps it will come back after
%% release 1.  So the doc skeleton is preserved.
%%

% Create well-known link to this spot for HTML version
\ifpdf
\else
\HCode{<a name='DUCC_CLI_MONITOR'></a>}
\fi
    \section{ducc\_monitor}

    \paragraph{Description:}
    
    It may be desired to monitor a job's progress after it has been submitted. The monitor CLI 
    connects to the DUCC message flow and provides job status as it progresses including state 
    changes, error counts, and number of work items processed. 
    
    \paragraph{Usage:}
    \begin{description}
    \item[Script wrapper] \ducchome/bin/ducc-monitor {\em options}
    \item[Java Main]      java -cp \ducchome/lib/uima-ducc-cli.jar org.apache.uima.ducc.cli.DuccJobMonitor {\em options}
    \end{description}

    \paragraph{Options:}
    \begin{description}
        \item[$--$debug ]          
          Prints internal debugging information, intended for DUCC developers or extended problem determination.
        \item[$--$id {[jobid]}]
          The ID is the id of the job to monitor. (Required)
        \item[$--$help]
          Prints the usage text to the console. 
        \item[$--$quiet] 
          Disable CLI informational miessages;
        \item[--timestamp]
          Enables timestamps on the monitor messages.
     \end{description}
        
    \paragraph{Notes:}

    % 
% Licensed to the Apache Software Foundation (ASF) under one
% or more contributor license agreements.  See the NOTICE file
% distributed with this work for additional information
% regarding copyright ownership.  The ASF licenses this file
% to you under the Apache License, Version 2.0 (the
% "License"); you may not use this file except in compliance
% with the License.  You may obtain a copy of the License at
% 
%   http://www.apache.org/licenses/LICENSE-2.0
% 
% Unless required by applicable law or agreed to in writing,
% software distributed under the License is distributed on an
% "AS IS" BASIS, WITHOUT WARRANTIES OR CONDITIONS OF ANY
% KIND, either express or implied.  See the License for the
% specific language governing permissions and limitations
% under the License.
% 
% Create well-known link to this spot for HTML version
\ifpdf
\else
\HCode{<a name='DUCC_CLI_RESERVE'></a>}
\fi
    \section{ducc\_reserve}

    \paragraph{Description:}
    The reserve CLI is used request a reservation of resources. Reservations can be for entire 
    machines or partial machines, based on memory requirements. All reservations are persistent: 
    the resources remain dedicated to the requestor until explicitly returned. All reservations are 
    performeed on an "all-or-nothing" basis: either the entire set of requested resources is reserved, 
    or the reservation request fails. 

    All forms of ducc\_reserve block until the reservation is complete (or fails) at which point the DUCC
    ID of the reservation and the names of the reserved nodes are printed to the console and the
    command returns.

    \paragraph{Usage:}
        \begin{description}
        \item[Script wrapper] \ducchome/bin/ducc\_reserve {\em options}
        \item[Java Main]      java -cp \ducchome/lib/uima-ducc-cli.jar org.apache.uima.ducc.cli.DuccReservationSubmit {\em options}
        \end{description}

    \paragraph{Options:}
    
        \begin{description}

            \item[$--$debug ]          
              Prints internal debugging information, intended for DUCC developers or extended problem determination.
              
            \item[$--$description {[text]}]               
              The text is any string used to describe the reservation. It is displayed in the Web Server. 
              
            \item[$--$help ]             
              Prints the usage text to the console. 
                            
            \item[$--$_memory\_size {[integer]}]               
              This specifies the amount of memory the reserved machine must support. After rounding
              up it must match the total usable memory on the machine. (Required)

            \item[$--$scheduling\_class {[classname]}]               
              This specifies the name of the scheduling class the RM will use to determine the resource 
              allocation for each process. It must be one implementing the RESERVE policy.
              If not specified, the RESERVE default is taken from the site class definitions file
              described \hyperref[subsubsec:class.configuration]{here.} 
              
            \item[$-$f, $--$specification {[file]}]               
              All the parameters used to request a reservation may be placed in a standard Java 
              properties file. This file may then be used to submit the request (rather than providing all 
              the parameters directory to submit). 

        \end{description}
            
    \paragraph{Notes:}
    Reservations must be for full machines, in a job class implementing the RESERVE scheduling
    policy. The default DUCC distribution configures class {\em reserve} for full machine
    reservations. 



    % 
% Licensed to the Apache Software Foundation (ASF) under one
% or more contributor license agreements.  See the NOTICE file
% distributed with this work for additional information
% regarding copyright ownership.  The ASF licenses this file
% to you under the Apache License, Version 2.0 (the
% "License"); you may not use this file except in compliance
% with the License.  You may obtain a copy of the License at
% 
%   http://www.apache.org/licenses/LICENSE-2.0
% 
% Unless required by applicable law or agreed to in writing,
% software distributed under the License is distributed on an
% "AS IS" BASIS, WITHOUT WARRANTIES OR CONDITIONS OF ANY
% KIND, either express or implied.  See the License for the
% specific language governing permissions and limitations
% under the License.
% 
% Create well-known link to this spot for HTML version
\ifpdf
\else
\HCode{<a name='DUCC_CLI_UNRESERVE'></a>}
\fi
    \section{ducc\_unreserve}
    \label{sec:cli.unreserve}

    \paragraph{Description:}
    The unreserve CLI is used to release reserved resources. 

    \paragraph{Usage:}
    \begin{description}
    \item[Script wrapper] \ducchome/bin/ducc\_unreserve {\em options}
    \item[Java Main]      java -cp \ducchome/lib/uima-ducc-cli.jar org.apache.uima.ducc.cli.DuccReservationCancel {\em options}
    \end{description}

    \paragraph{Options:}
    \begin{description}
        \item[$--$debug ]          
          Prints internal debugging information, intended for DUCC developers or extended problem determination.
        \item[$--$id {[jobid]}]
          The ID is the id of the reservation to cancel. (Required)
        \item[$--$help]
          Prints the usage text to the console. 
        \item[$--$role\_administrator] The command is being issued in the role of a DUCC administrator.
          If the user is not also a registered administrator this flag is ignored.  (This helps to
          protect administrators from inadvertently canceling jobs they do not own.)          
     \end{description}
        
    \paragraph{Notes:}
    None.


        \section{ducc\_service\_submit}

    \paragraph{Description:}
    The ducc\_service\_submit CLI is used to submit a job as a service to DUCC. The CLI is similar to
    ducc\_submit with the following key differences:
    
    \begin{itemize}
        \item There is no Collection Reader. 

        \item There is no Job monitor for services because services don't generally end of their own
          accord.
        \end{itemize}
        
        Service jobs must supply a fully-formed DD XML descriptor.  On submission of a service, the
        DUCC CLI examines the service DD descriptor for the queue name, and the supplied jvm\_args for
        a broker URL. It forms a service ID of the following form which may be referenced in the
        --service\_dependency clauses of jobs and services which are dependent on this service:
\begin{verbatim}
UIMA-AS:[queue-name]:[broker-url] 
\end{verbatim}

    The {\em queue-name} is the ActiveMQ queue name used by the service.

    The {\em ducc\_service\_submit} and {\em ducc\_service\_cancel} commands are primarily used by
    the DUCC Service Manager for starting and stopping instances of registered services.  It is legal
    and supported for these to be used outside of the context of registered services, however.  DUCC
    recognizes the processes as service processes and manages dependencies accordinly. 

    \paragraph{Usage:}
    \begin{description}
    \item[Executable Jar] java -jar \ducchome/lib/uima-ducc-sesrvice-submit.jar {\em options}
    \item[Script wrapper] java -jar \ducchome/bin/ducc-service-submit {\em options}
    \item[Java Main]      java -cp \ducchome/lib/uima-ducc-service-submit.jar org.apache.uima.ducc.cli.DuccServiceSubmit {\em options}
    \end{description}

    \paragraph{Options:}
    \begin{description}

        \item[--classpath] The classpath used for the service.
          
        \item[--classpath\_order {[UserBeforeDucc | DuccBeforeUser]} ]
          When DUCC deploys a process, set the user-supplied classpath before DUCC-supplied
          classpath, or the reverse.
          
        \item[--debug ]
          Enable debugging messages. This is primarily for debugging DUCC itself. 
          
        \item[--description {[text]}] The text is any string used to describe the job. It is displayed
          in the Web Server.
          
        \item[--environment {[env vars]}] Blank-delimeted list of environment variables.  Example:
\begin{verbatim}
"TERM=xterm DISPLAY=me.org.net:1.0". 
\end{verbatim}

        \item[--help ]
        This prints the usage text to the console. 

      \item[--jvm {[path-to-java]}] This specifies the JVM to use. If not specified, the same JVM
        used by the Agents is used.

      \item[--jvm\_args {[list]} ]        
        This specifes extra JVM arguments to be provided to the server process. It is a blank delimeted 
        list of strings. Example: 
\begin{verbatim}
--jvm_args -Xmx100M -Xms50M 
\end{verbatim}

      \item[--log\_directory {[path-to-log directory]}] 
        This specifies the path to the directory for the user logs. If not specified, the default is the 
        user's home directory. Example: 

        Within this directory DUCC creates a subdirectory for each job, using the numerical 
        ID of the job. The format of the generated log file names is described ELSEWHERE.NEED.REF.

        Note: Note that --log\_directory specifies only the path to a directory where 
        logs are to be stored. In order to manage multiple processes running in multiple 
        machines DUCC, sub-directory and file names are generated by DUCC and may 
        not be directly controlled by the user.

      \item[--process\_classpath {[ClASSPATH]}] 
        This specifies the Java CLASSPATH to use in each Job Process (JP) and must be 
        specified.

      \item[--process\_DD {[DD descriptor]}] 
        This specifies the UIMA Deployment Descriptor for the service.

      \item[--process\_environment {[environment]}] This specifies environment parameters for the Job
        Processes. If present, they are added to the Job Process environment as the process is
        spawned. It must be a quoted, blankdelimeted lsit of name-value pairs.

        Note: On Secure Linux systems, the environemnt variable 
        LD\_LIBRARY\_PATH may not be passed to the user's program. If it is 
        necessary to pass LD\_LIBRARY\_PATH to the JP or JD processes, it must be 
        specified as DUCC\_LD\_LIBRARY\_PATH. Ducc (securely) passes this as 
        LD\_LIBRARY\_PATH, after the JP or JD has assumed the user's identity. For 
        example: 
\begin{verbatim}
"--process_environment TERM=xterm DISPLAY=:1.0 DUCC_LD_LIBRARY_PATH=/my/own/lib.so" 
\end{verbatim}
        
      \item[--process\_failures\_limit {[integer]}] 
        This specifies the maximum number of individual Job Process (JP) failures that are to be 
        tolerated before killing the job. The default is 15. If this limit is exceeded over the lifetime 
        of a job DUCC terminates the entire job. 

      \item[--process\_initialization\_failures\_cap {[integer]}] 
        This specifies the maximum number of independent Job Process initialization failures (i.e. 
        System.exit(), kill-15...) before the number of Job Processes is capped at the number in 
        state Running currently. The default is 99.

        Note that the job is NOT killed if there are processes that have passed initialization and are 
        running. If this limit is reached, the only action is to not start new processes for the job. 

      \item[--process\_jvm\_args {[list]}] 
        This specifies additinal arguments to be passed to the Job Process JVM. Example: 
\begin{verbatim}
--process_jvm_args -Xmx400M -Xms100M 
\end{verbatim}
        
      \item[--process\_memory\_size {[size]}] This specifies the maximum amount of RAM in GB to be
        allocated to each Job Process.  This value is used by the Resource Manager to allocate
        resources. if this amount is exceeded by a Job Process the Agent terminates the process with
        a ShareSizeExceeded message.

      \item[--scheduling\_class {[classname]}] This specifies the name of the scheuling class the RM
        will use to determine the resource allocation for each process. The names of the classes are
        installation dependent. If not specified, the default is taken from the global DUCC
        configuration ducc.properties. 

      \item[--service\_dependency{[list]}] This specifies a comma-delimeted list of services the job
        processes are dependent upon.  Each endpoint must be of the form UIMA-AS:endpoint:broker\_url
        where endpoint is the UIMA-AS service endpoint and broker\_url is the ActiveMQ broker URL.

        In the example are two dependencies, one with endpoint RandomSleepAE and broker
        tcp:bluej682:61616, and the other with endpoint OtherEp and broker URL
        tcp:bluej123:123. Example:
\begin{verbatim}
--service_dependency UIMA-AS:RandomSleepAE:tcp:bluej682:61616 UIMA-AS:OtherEp:tcp:bluej123:123 
\end{verbatim}

      \item[--service\_linger {[seconds]}] This is the time in milliseconds to wait after last
        referring job or service exits before stopping a non-autostarted service.

      \item[--service\_ping\_class {[classname]}] This is the class used to ping a service, if the
        default DUCC-supplied class is not used.  It is always required for CUSTOM services, and
        may be specified to override the default for UIMA-AS services.  It must extend
        org.apache.uima.ducc.common.AServicePing.

      \item[--service\_ping\_classpath {[classpath]}] If {\em service\_ping\_class} is specified,
        this is the classpath containing service\_custom\_ping class and dependencies.  If not
        specified, the Agent's classpath is used (which usually is not correct!).

      \item[--service\_ping\_dolog {[boolean]}] If specified, write pinger stdout and stderr
        messages to a log, else suppress the log.

      \item[--service\_ping\_jvm\_args {[java-system-property-assignments]}] -D jvm system property
        assignments to pass to jvm when running the pinger.

      \item[--service\_ping\_timeout {[time-in-ms]}] This is the time in milliseconds to wait for a ping to the
        service. 

      \item[--service\_request\_endpoint {[string]}] This specifies the expected service id.  The string
        must be in form {\tt queue:broker-url}.  This must match the service ID that is derived from
        the {\em --process\_DD}; if it does not match, the submit fails.  This is intended for use
        by applications using the service API and CLI as a fail-fast if something is wrong with the
        application.

      \item[--specifiecaiton {[file]}] All the parameters used to submit a job may be placed in a
        standard Java properties file.  This file may then be used to submit the job (rather than
        providing all the parameters directory to submit).
        For example, 

\begin{verbatim}
ducc_submit --specification job.props 
\end{verbatim}
        
        where the job.props contains: 

\begin{verbatim}
working_directory=/Users/challngr/projects/ducc/ducc_test/test/bin 
process_failures_limit=20 
driver_environment= DUCC_LD_LIBRARY_PATH=/a/other/bogus/path 
process_environment=AE_INIT_TIME=10000 DUCC_LD_LIBRARY_PATH=/a/bogus/path 
log_directory=/Users/challngr/ducc/logs/ 
process_initialization_failures_cap=99 
process_descriptor_AE=org.apache.uima.ducc.test.randomsleep.FixedSleepAE 
process_classpath=/home/bob/projects/ducky-service.jar 
description=../simple/jobs/1.job[AE] 
process_jvm_args=-Xmx100M -DdefaultBrokerURL=tcp://localhost:61616 
scheduling_class=fixed 
process_memory_size=15 
\end{verbatim}
        
        \item[--working\_directory {[directory-name]}]
          This specifies the working directory to be set by the Job Driver and Job Process processes. 
          If not specified, the current directory is used.
    \end{description}
        
    \paragraph{Notes:}
    When searching for UIMA XML resource files such as descriptors, DUCC searches both the 
    classpath and the data path according to the following rules: 

    \begin{enumerate}
        \item If the resource ends in .xml it is assumed the resource is a file and the path is either
          an absolute path or a path relative to the specified working directory. If the file is not
          found the search exits and the job is terminated.

        \item If the resource does not end in .xml, DUCC creates a path by replacing the "." 
          separators with "/" and appending ".xml". It then searches the CLASSPATH for the 
          resource as a file. 
    \end{enumerate}

    If the resource is found in either place the search is successful. Otherwise the search 
    fails and the job is terminated. 

    Note: Note that in the current implementation, resources are NOT searched    
    for inside jars in the classpath. Files must be supplied. 


    % Create well-known link to this spot for HTML version
\ifpdf
\else
\HCode{<a name='DUCC_CLI_SERVICE_CANCEL'></a>}
\fi
    \section{ducc\_service\_cancel}
    \label{sec:cli.service-cancel}
    \paragraph{Description:}

    The ducc\_service\_cancel CLI is used to terminate a service instance.

    Note: This command will for for {\em registered} services, but if the service is registered for
    {\em autostart}, the Service Manager will restart the process.  This can be used to advantage, to
    recycle specific registered services instances when needed.

    \paragraph{Usage:}
    \begin{description}
    \item[Executable Jar] java -jar \ducchome/lib/uima-ducc-service-cancel.jar {\em options}
    \item[Script wrapper] \ducchome/bin/ducc\_service\_cancel {\em options}
    \item[Java Main]      java -cp \ducchome/lib/uima-ducc-service-cancel.jar org.apache.uima.ducc.cli.DuccServiceCancel {\em options}
    \end{description}

    \paragraph{Options:}
    \begin{description}
        \item[--debug ]          
          Prints internal debugging information, intended for DUCC developers or extended problem determination.          
        \item[--id {[jobid]}]
          The ID is the id of the service instance to cancel.
        \item[--help]
          Prints the usage text to the console. 
        \item[--role\_administrator] The command is being issued in the role of a DUCC administrator.
          If the user is not also a registered administrator this flag is ignored.  (This helps to
          protect administrators from inadvertently canceling work they do not own.)
     \end{description}
        
    \paragraph{Notes:}
    None.


    2y% Create well-known link to this spot for HTML version
\ifpdf
\else
\HCode{<a name='DUCC_CLI_PROCESS_SUBMIT'></a>}
\fi
    \section{ducc\_process\_submit}
    \label{sec:cli.ducc-process-submit}
    \paragraph{Description:}
       Use {\em ducc\_process\_submit} to submit a Managed Reservation, also known as an
       {\em arbitrary process} to DUCC.  The intention
       of this function is an alternative to utilities such as {\em ssh}, in order to allow the
       spawned processes to be fully managed by DUCC.  This allows the DUCC scheduler to allocate
       the necessary resources (and prevent over-allocation), and the DUCC run-time environment
       to manage process lifetime.

       If {\em process\_attach\_console} is specified, Stdin, Stderr, and Stdout of the remote
       process are redirected to the submitting console.  It is thus possible to run interactive
       sessions with remote processes where the resources are managed by DUCC.

    \paragraph{Usage:}
    \begin{description}
    \item[Script wrapper] \ducchome/bin/ducc\_process\_submit {\em options}
    \item[Java Main]      java -cp \ducchome/lib/uima-ducc-cli.jar org.apache.uima.ducc.cli.DuccManagedReservationSubmit {\em options}
    \end{description}

    \paragraph{Options:}
    \begin{description}
    
        \item[$--$cancel\_on\_interrupt ] Cancel managed reservation on interrupt
          (Ctrl-C).  If running with {\em $--$wait\_for\_completion} and this flag is specified,
          terminating the submit process will result in the remote process being terminated.

        \item[$--$description {[text]}] The text is any string used to describe the process. It is
          displayed in the Web Server. When specified on a command-line the text usually must be
          surrounded by quotes to protect it from the shell.

        \item[$--$debug ] Prints internal debugging information, intended for DUCC developers or
          extended problem determination.

        \item[$--$environment {[env vars]}] Blank-delimited list of environment variable
          assignments for the process. Example:
          \begin{verbatim}
--environment TERM=xterm DISPLAY=:1.0
          \end{verbatim}
             
          Additional entries may be copied from the user's environment based on the setting of
          ducc.submit.environment.propagated in the global DUCC configuration ducc.properties.

          Note: When used as a CLI option, the environment string must usually be
          quoted to protect it from the shell.

        \item[$--$help] Prints the usage text to the console.

        \item[$--$log\_directory {[path-to-log directory]} ]

          This specifies the path to the directory for the user logs. If not specified, the default
          is \$HOME/ducc/logs. Example: 
\begin{verbatim}
--log_directory /home/bob 
\end{verbatim}
          
          Within this directory DUCC creates a sub-directory for each process, using the numerical 
          ID of the job. The format of the generated log file names as described
          \hyperref[chap:job-logs]{here}.
          
          Note: Note that $--$log\_directory specifies only the path to a directory where 
          logs are to be stored. In order to manage multiple processes running in multiple 
          machines DUCC, sub-directory and file names are generated by DUCC and may 
          not be directly specified. 

        \item[$--$process\_attach\_console] If specified, redirect remote process stdio is 
          redirected the local submitting console.
          
        \item[$--$process\_executable {[program name]}] This is the full path to a program to be
          executed.

        \item[$--$process\_executable\_args {[argument list]}] This is a list of arguments for
          {\em process\_executable}, if any.   When specified on a command-line the text usually must be
          surrounded by quotes to protect it from the shell.

        \item[$--$process\_memory\_size {[size]} ] This specifies the maximum amount of RAM in GB to
          be allocated to each process.  This value is used by the Resource Manager to allocate
          resources. if this amount is exceeded by a process the Agent terminates the process with a
          ShareSizeExceeded message.

        \item[$--$scheduling\_class {[classname]} ] This specifies the name of the scheduling class the
          RM will use to determine the resource allocation for each process. The names of the
          classes are installation dependent. If not specified, the default is taken from the global
          DUCC configuration ducc.properties.

        \item[$--$specification, $-$f {[file]} ] All the parameters used to submit a process may be placed
          in a standard Java properties file.  This file may then be used to submit the process
          (rather than providing all the parameters directory to submit).
          
          For example, 
\begin{verbatim}
ducc_process_submit --specification job.props 
ducc_process_submit -f job.props 
\end{verbatim}

          where job.props contains: 
\begin{verbatim}
working_directory   = /home/bob/projects
environment         = AE_INIT_TIME=10000 LD_LIBRARY_PATH=/a/bogus/path 
log_directory       = /home/bob/ducc/logs/ 
description         = Simple Process
scheduling_class    = fixed 
process_memory_size = 15 
\end{verbatim}

        \item[$--$wait\_for\_completion ] If specified, the submit command does not return control to
          the console immediately, and instead monitors the DUCC state traffic and prints
          information about the process as it progresses.
          
        \item[$--$working\_directory ] This specifies the working directory to be set by the Job
          Driver and Job Process processes.  If not specified, the current directory is used.

     \end{description}
        
    \paragraph{Notes:}


    % Create well-known link to this spot for HTML version
\ifpdf
\else
\HCode{<a name='DUCC_CLI_PROCESS_CANCEL'></a>}
\fi
    \section{ducc\_process\_cancel}

    \paragraph{Description:}
    The cancel CLI is used to cancel a process that has previously been submitted but which has not yet 
    completed. 

    \paragraph{Usage:}
    \begin{description}
    \item[Executable Jar] java -jar \ducchome/lib/uima-ducc-process-cancel.jar {\em options}
    \item[Script wrapper] \ducchome/bin/ducc\_process\_cancel {\em options}
    \item[Java Main]      java -cp \ducchome/lib/uima-ducc-process-cancel.jar org.apache.uima.ducc.cli.DuccManagedReservationCancel {\em options}
    \end{description}

    \paragraph{Options:}
    \begin{description}
        \item[$--$debug ]          
          Prints internal debugging information, intended for DUCC developers or extended problem determination.          
        \item[$--$id {[jobid]}]
          The DUCC ID is the id of the process to cancel.
        \item[$--$help]
          Prints the usage text to the console.
        \item[$--$reason]
          Optional. This specifies the reason the process is canceled, for display in the web server. 
        \item[$--$role\_administrator] The command is being issued in the role of a DUCC administrator.
          If the user is not also a registered administrator this flag is ignored.  (This helps to
          protect administrators from inadvertantly canceling work they do not own.)
     \end{description}
        
    \paragraph{Notes:}
    None.


    % Create well-known link to this spot for HTML version
\ifpdf
\else
\HCode{<a name='DUCC_CLI_SERVICES'></a>}
\fi
    \section{ducc\_services}
    \label{sec:cli.ducc-services}

    \paragraph{Description:}

        The ducc\_services CLI is used to manage service registration. It has a number of functions 
        as listed below.
        
        The functions include: 
        \begin{description}
            \item[Register] This registers a service with the Service Manager by saving a service
              specification in the Service Manager's registration area. The specification is
              retained by DUCC until it is unregistered.

              The registration consists primarily of a service specification identical to that used
              with \hyperref[sec:cli.service-submit]{ducc\_submit\_service}.  This specification is
              used when the Service Manager needs to start a service instance.  A second properties
              file, the {\em meta properties} for the service, contains additional state and
              management properties.  The registered properties for a service are made available for
              viewing from the DUCC Web Server's \hyperref[sec:ws-service-details]{service details}
              page.
              
            \item[Unregister] This unregisters a service with the Service Manager. When a service is
              unregistered DUCC optionally stops the service instance, if any, and discard the
              saved specification.
              
            \item[Start] The start function instructs DUCC to allocate resources for a service and to
              start it in those resources. The service remains running until explicitly stopped. DUCC
              will attempt to keep the service instances running if they should fail. The start function
              is also used to increase the number of running service instances if desired.
              
            \item[Stop] The stop function stops some or all service instances.
              
            \item[Query] The query function returns detailed information about all known services, both
              registered and otherwise.
              
            \item[Modify] The modify function allows some aspects of a registered service to be updated
              without reregistereing the service. It optionally alters the running service instances to
              conform with the updates.    
        \end{description}
            

    \paragraph{Usage:}
       \begin{description}
          \item[Executable Jar] java -jar \ducchome/lib/uima-ducc-services.jar {\em options}
          \item[Script wrapper] \ducchome/bin/ducc\_services {\em options}
          \item[Java Main]      java -cp \ducchome/lib/uima-ducc-services.jar org.apache.uima.ducc.cli.DuccServiceApi {\em options}
          \end{description}
          
          The ducc\_services CLI requires one of the verbs listed above as the first argument. Other arguments are determined
          by the verb.

    \paragraph{Options:}

    \subsection{Common Options}
        These options are common to all of the service verbs:
        \begin{description}
           \item[$--$debug ]          
             Prints internal debugging information, intended for DUCC developers or extended problem determination.                    
           \item[$--$help]
             Prints the usage text to the console. 
        \end{description}
        
    \subsection{ducc\_services --register Options}
    \label{subsec:cli.ducc-services.register}
       The {\em register} function submits a service specification to DUCC.  DUCC stores this 
       information until it is {\em unregistered}.  Once registered, a service may be
       started, stopped, etc.

       \begin{description}
           \item[$--$register {[specification file] [options]}] The specification file is optional.  If
             specified, it has the same contents as described for the \hyperref[sec:cli.service-submit]{{\em
                 ducc\_service\_submit}} command.  As with {\em ducc\_service\_submit}, any of the
             keywords in the specification may be overridden on the command line.

           \item[$--$autostart {[true or false]}] This indicates whether to register the service as
             an autostarted service.  If not specified, the default is {\em false}.
             
           \item[$--$instances {[n]}] This specifies the number of instances to start when the service
             is started.  If not specified, the default is 1.
             
           \item[Other keywords] These are the same as described for \hyperref[sec:cli.service-submit]{{\em
                 ducc\_service\_submit}}
       \end{description}


    \subsection{ducc\_services --start Options}

    The start function instructs DUCC to allocate resources for a service and to start it in those
    resources. The service remains running until explicitly stopped. DUCC will attempt to keep the
    service instances running if they should fail. The start function is also used to increase the
    number of running service instances if desired.
    
       \begin{description}
       \item[$--$start {[service-id or endpoint]}] This indicates that a service is to be started. The service id
         is either the numeric ID assigned by DUCC when the service is registered, or the service
         endpoint string.  Example:
\begin{verbatim}
ducc_services --start 23 
ducc_services --start UIMA-AS:Service23:tcp://bob.com:12345 
\end{verbatim}
         
       \item[$--$instances {[integer]}] This is the number of instances to start. If omitted, sufficient
         instances to match the registered number are started. If more than the registered number of
         instances is running this command has no effect.

         If the number of instances is specified, the number is added
         to the currently number of running instances. Thus if five instances are running and
\begin{verbatim}
         ducc_services --start 33 --instances 5
\end{verbatim}
         is issued, five more service instances are started for service 33 for a total of ten,
         regardless of the number specified in the registration. The registry is updated if the
         {\em --update} option is also specified. Examples:
\begin{verbatim}
ducc_services --start 23 --intances 5 
ducc_services --start UIMA-AS:Service23:tcp://bob.com:12345 --instances 3 --update 
\end{verbatim}

       \item[$--$update]If specified, the registry is updated to the total number of started
         instances.  Example:
\begin{verbatim}
ducc_services --start UIMA-AS:Service23:tcp://bob.com:12345 --instances 3 --update 
\end{verbatim}
       \end{description}

    \subsection{ducc\_services --stop Options}
    The stop function instructs DUCC to stop some number of service instances. If no specific number
    is specified, all instances are stopped. This is used only for registered services. Use
    \hyperref[sec:cli.service-cancel]{{\em ducc\_service\_cancel}} command to stop submitted services.

    \begin{description}

  \item[$--$stop {[service-id or endpoint]}] This specifies the service to be stopped. The service id
         is either the numeric ID assigned by DUCC when the service is registered, or the service
         endpoint string. Example:
\begin{verbatim}
ducc_services --stop 23 
ducc_services --stop UIMA-AS:Service23:tcp://bob.com:12345 
\end{verbatim}
         
       \item[$--$instances {[integer]}] This is the number of instances to stop. If omitted, all
         instances for the service are stopped.  If the number of instances is specified, then only
         the specified number of instances are stopped. Thus if ten instances are running for a
         service with numeric id 33 and
\begin{verbatim}
ducc_services --stop 33 --instances 5
\end{verbatim}
         is issued, five (randomly selected) service instances are stopped for
         service 33, leaving five running.  The registry is updated if the {\em --update} option is
         specified. The registered number of instances is never reduced to zero even if the number of
         running instances is reduced to zero.

         Example: 
\begin{verbatim}
ducc_services --stop 23 --intances 5 
ducc_services --stop UIMA-AS:Service23:tcp://bob.com:12345 --instances 3  
\end{verbatim}

       \item[$--$update] If specified, the registry is updated to the total number of instances
         remaining, but is never reduced below one (1). Example: 
\begin{verbatim}
ducc_services --stop UIMA-AS:Service23:tcp://bob.com:12345 --instances 3 --update
\end{verbatim}

    \end{description}

    \subsection{ducc\_services --modify Options}
    The modify function dynamically updates some of the attributes of a registered service. 
    
    \begin{description}
        \item[$--$modify {[service-id or endpoint]}]  This identifies the service to modify. The service id is either
          the numeric ID assigned by DUCC when the service is registered, or the service endpoint
          string.  Example:
\begin{verbatim}
ducc_services --modify 23 --instances 3 
ducc_services --modify UIMA-AS:Service23:tcp://bob.com:12345 --intances 2 
\end{verbatim}

        \item[ --instances {[integer]}] This updates the number of services instances that are
          started when the service is started.  Only the registration is updated. If the {\em $--$activate}
          option is also specified, running instances are stopped or started as needed to match the
          new number.

          Example: 
\begin{verbatim}
ducc_services --modify 23 --intances 5 
ducc_services --modify UIMA-AS:Service23:tcp://bob.com:12345 --instances 3 --activate 
\end{verbatim}

        \item[ --activate {[integer]}] When specified, the number of running service instances is
          increased or decreased to match the newly specified number.

          Example: 
\begin{verbatim}
ducc_services --modify 23 --intances 5 
ducc_services --modify UIMA-AS:Service23:tcp://bob.com:12345 --instances 3 --activate 
\end{verbatim}

        \item[ --autostart {["true" or "false"]}] This changes the autostart property for the
          registered services. When set to "true", the service is started automatically when the
          DUCC system is started.  If the service is not currently started, it will now start
          with the registered number of instances.  If the service is running, the instances
          remain running.

          One way to think of this is: if {\em autostart} is true, DUCC will attempt to keep
          the registered number of instances running at all times.  If {\em autostart} is
          false, all instance start and stop is manual.
          Example: 
\begin{verbatim}
ducc_services --stop UIMA-AS:Service23:tcp://bob.com:12345 --autostart false 
\end{verbatim}
        \end{description}

    \subsection{ducc\_services --query Options}
    The query function returns details about all known services of all types and classes, including 
    the DUCC ids of the service instances (for submitted and registered services), the DUCC ids of 
    the jobs using each service, and a summary of each service's queue and performance statistics, 
    when available. 
    
    All information returned by {\em ducc\_services $--$query} is also available via the
    \hyperref[ws:services-page]{Services Page} of the Web Server as well as the DUCC Service API (see 
    the JavaDoc).

    \begin{description}
    \item[$--$query {[service-id or endpoint]}] This indicates that a service is to be stopped. The
      service id is either the numeric ID assigned by DUCC when the service is registered, or the
      service endpoint string.

      If no id is given, information about all services is returned. 

      Below is a query against a system with three services. 

      The service with endpoint {\tt UIMA-AS:FixedSleepAE\_6:tcp://bobmach291:61617} is a service
      submitted outside of DUCC so it is marked as Implicit and has no implementing processes that
      are known to DUCC. It is used by job 0 (``References'') and is active, available, and being
      actively pinged. The ActiveMq queue statistics are shown.

      The service with endpoint {\tt UIMA-AS:FixedSleepAE\_5:tcp://bobmach:61617} is a 
      registered service, whose registered numeric id is 2. It is registered for two instances and 
      no autostart. Since it is not autostarted, it will be terminated when it is no longer used. It 
      will linger for 5 seconds after the last referencing job completes, in case a subsequent job 
      that uses it enters the system (not a realistic linger time!). It has two active
      instances whose DUCC Ids are 9 and 5. It is currently used (referenced) 
      by DUCC jobs 1 and 5. 

      The service with endpoint {\tt UIMA-AS:FixedSleepAE\_1:tcp://bobmach:61617} is a 
      submitted service. It was submitted twice, and so has two implementers, DUCC service 
      jobs 0 and 1. It is referenced by job 7. It will continue to run until somebody cancels it, 
      even if it is not used. 

\begin{verbatim}
Service: UIMA-AS:FixedSleepAE_6:tcp://bobmach291:61617 
Service Class : Implicit 
Implementors : (N/A) 
References : 0 
Dependencies : none 
Service State : Available 
Ping Active : true 
Autostart : false 
Manual Stop : false 
Queue Statistics: 
Consum Prod Qsize minNQ maxNQ expCnt inFlgt  DQ  NQ Disp 
    78  240   170     2 36414      0      0 636 806  636 

Service: UIMA-AS:FixedSleepAE_5:tcp://bobmach291:61617 
Service Class : Registered as ID 2 instances[2] linger[5] 
Implementors : 9 8 
References : 1 5 
Dependencies : none 
Service State : Available 
Ping Active : true 
Autostart : false 
Manual Stop : false 
Queue Statistics: 
Consum Prod Qsize minNQ maxNQ expCnt inFlgt  DQ  NQ Disp 
    52   44     0     0     3      0      0 402 402  402 

Service: UIMA-AS:FixedSleepAE_1:tcp://bobmach291:61617 
Service Class : Submitted 
Implementors : 1 0 
References : 7 
Dependencies : none 
Service State : Available 
Ping Active : true 
Autostart : false 
Manual Stop : false 
Queue Statistics: 
Consum Prod Qsize minNQ   maxNQ expCnt inFlgt  DQ  NQ Disp 
    52    0     0     1 1504371      0      0  35  35   35 
\end{verbatim}
    \end{description}
    \paragraph{Notes:}





\chapter{Application Programming  Interface}

%% this inputs a chapter
\chapter{Job Logs}


The DUCC logs are managed by log4j and are configured using ducc\_runtime/log4j.xml. It 
is not in the scope of this document to describe log4j or its configuration mechanism. Details on 
log4j can be found at http://logging.apache.org/log4j/1.2/. 

The "user logs" are the Job Driver (JD) and Job Process (JP) logs. There is one log for each process 
of a job. The JD log is divided between two physical files: 

\begin{enumerate}
  \item The logs and stdout written by the UIMA collection reader. The collection reader uses the 
    UIMA logger which is by default directed to stdout. 

    \item The diagnostic logs written the the DUCC JD wrapper around the job's collection reader. 
      This log is written using log4j. 
\end{enumerate}

A number of other usefiles are written to the log directory: 
\begin{enumerate}

  \item A properties file containing the full job specification for the job. This includes all the 
    parameters specified by the user as well as the default parameters. This file is written to 
    job-specification.properties. 


  \item The UIMA pipeline descriptor constructed by DUCC that describes the process that is 
    dispatched to each Job Process (JP). The name of this file is of the form 

\begin{verbatim}
         JOBID-uima-ae-descriptor-PROCESS.xml 
\end{verbatim}

    where 

    \begin{description}
        \item[JOBID] This is the numerical id of the job as assigned by DUCC.
        \item[PROCESS] This is the process id of the Job Driver (JD) process.
        \end{description}
      
      \item The UIMA-AS service descriptor that defines the process that defines the job as as UIMAAS 
        service. The name of this file is of the form 
\begin{verbatim}
         JOBID-uima-as-dd-PROCESS.xml 
\end{verbatim}
    
        where 
        \begin{description}
            \item[JOBID] This is the numerical id of the job as assigned by DUCC.
            \item[PROCESS] This is the process id of the Job Driver (JD) process.
        \end{description}

      \item A Java serialized object containing the performance breakdown for the job. This is used 
        by the Web Server to display the breakdown. This file is written to job-performancesummary.ser. 
 \end{enumerate}

 The JP logs are written by default to HOME/ducc/logs, where HOME is the submitting user's 
 home directory. In this directory, a subdirectory whose name is the numerical id of the job is 
 created by DUCC, where all logs for the job are written. 
 
 The collection reader's log is written to the file HOME/ducc/logs/JOBID/jd.out.logvia log4j. 
 It is written in multiple generations, and its size is governed by the same log4j configuration file 
 used for the DUCC Daemon processes. The size of each generation and the number of generations 
 is configured in the jdout appender stanza. 

 Each JP log and the diagnostic JD log is of the following form:

\begin{verbatim}
         JOBID-TYPE-NODE-PROCESS.log 
\end{verbatim}
 
 where 

\begin{description}
    \item[JOBID] This is the numerical id of the job as assigned by DUCC.
    \item[TYPE] This is either the string "UIMA" for JP logs, or "JD" for JD logs.
    \item[NODE] This is the name of the machine where the process runs.
    \item[PROCESS] This is the process id of the process on the indicated node.
\end{description}

This shows the contents a sample log directory for a small job that consisted of two processes.

\begin{verbatim}
         100-JD-bluej290-1-29383.log 
         100-uima-ae-descriptor-29383.xml 
         100-uima-as-dd-29383.xml 
         100-UIMA-bluej290-2-32766.log 
         100-UIMA-bluej291-63-13594.log 
         jd.out.log 
         job-performance-summary.ser 
         job-specification.properties 
\end{verbatim}

In this example, 

\begin{itemize}
     \item[] The file 100-JD-bluej290-1-29383.log is the diagnostic JD log, where the JD executed on node
       bluej290-1 in process 29383.

     \item[] The file 100-uima-ae-descriptor-29383.xml is the UIMA pipeline descriptor describing the
       service process that is launched in each JP, where the JD process is 29383.

     \item[] The file 100-uima-as-dd-29383.xml is the UIMA-AS service descriptor where the client is
       the JD process running in process 29383.

     \item[] The file 100-UIMA-bluej290-2-32766.log is a JP log for job 100, that ran on node
       bluej290-2, in process 32766.

     \item[] The file 100-UIMA-bluej291-63-13594.log is a JP log for job 100, that ran on node
       bluej291-63, in process 13594

     \item[] The file jd.out.log is the user's JD log, containing the user's collection reader output.

     \item[] The file job-performance-summary.ser is the serialized performance breakdown that is
       displayed in the Web Server The file job-specification.propeties is the properties file
       describing the job.
\end{itemize}
     


%% this inputs a chapter
% Create well-known link to this spot for HTML version
\ifpdf
\else
\HCode{<a name='DUCC_WEBSERVER'></a>}
\fi
\chapter{DUCC Web Server}

    The DUCC Web Server default address is accessed from the URL http://wshost:42133. Each local
    installation configures the host for "wshost" and may override the default port of 42133.

    The Webserver is designed to be mostly self-documenting. The design is intentionally simple 
    and contains a link to this document. Column headers and reason/state codes have display a short 
    description if you hover your mouse over it. 

    The columns can all be sorted by clicking on the column headers. 

% Create well-known link to this spot for HTML version
\ifpdf
\else
\HCode{<a name='DUCC_WS_COMMON'></a>}
\fi
    \section{Common Links}

        Every page contains a common header containing links and controls. The links permit navigation
        to other content at the site. The controls provide page-wise configuration of the content at
        that page.

        The following links are available on every page of the web server: 

        \begin{description}
          \item[Authentication] \hfill \\ 
            Authentication is in order to cancel jobs and reservations, to create a
            reservation, and to perform administration. It is not required to simply view the pages.

            \begin{itemize}
              \item Login - Authenticate and start a session with the Web Server.             
              \item Logout - Terminate the Web Server session 
            \end{itemize}

          \item[Preferences]
            Set preferences for table style, date style, filters, etc.
            
          \item[DuccBook] \hfill \\
            This is a link to the HTML version of the document you are reading.

          \item[Jobs] \hfill \\
            This navigates to the Jobs page, showing all the jobs in the system.

          \item[Reservations] \hfill \\
            This navigates to the Reservations page, showing all the reservations
            in the system and provides a button that can be used to request new reservations. 

          \item[Services] \hfill \\
            This navigates to the Services page, showing all the services in the
            system.

          \item[System] \hfill \\
            This opens a submenu with system-related links:
            \begin{itemize}
              \item[] Administration - This opens a page with administrative functions. 
              \item[] Classes - This shows all the scheduling classes defined to the system. 
              \item[] Daemons - This shows the status of DUCC's management processes. 
              \item[] DuccBook - This manual. 
              \item[] Machines - This shows the status of all the ducc worker nodes. 
            \end{itemize}
      \end{description}              

      % Create well-known link to this spot for HTML version
      \ifpdf
      \else
      \HCode{<a name='DUCC_WS_JOBS'></a>}
      \fi
      
    \section{Jobs Page}
    \label{sec:ws.jobs-page}
        The Web Server's home page is also the Jobs page. This page has links to all the rest of the content 
        at the site and shows the status of all the jobs in the system. 
    
        The Jobs page contains the following columns: 

        \begin{description}

            \item[Id] \hfill \\
              This is the ID as assigned by DUCC. This field is hyperlinked to a
              \hyperref[sec:ws-job-details]{Job Details} page for that job that shows the breakdown of
              all the processes assigned to the job and their state.
              
            \item[Start] \hfill \\
              This is the time the Job is accepted into DUCC.
              
            \item[Duration] \hfill \\
              This shows two times.  In green the length of time the job has been running.  In black is
              the estimated time of completion, based on current resources and remaining work.  When
              the job completes, the time shown is the total elapsed time of the job.
                            
            \item[User] \hfill \\
              This is the userid of the job owner.
              
            \item[Class] \hfill \\
              This is the resource class the job is submitted to.
              
            \item[State] \hfill \\
              This shows the state of the job.  The normal job progression is shown below, with an
              explanation of what each state means.
              \begin{description}
                  \item[Received] - The job has been vetted, persisted, and assigned a unique ID. 
                  \item[WaitingForDriver] - The job is waiting for the Job Driver to initialize. 
                  \item[WaitingForServices] - The job is waiting for verification from the
                    Service Manager that required services are started and responding.  This may
                    cause DUCC to start services if necessary.  In that even this state will
                    persist until all pre-requisite services are ready.
                  \item[WaitingForResources] - The job is waiting to be scheduled. In busy
                    systems this may require preemption of existing work.  In that case this
                    state will persist until preemption is complete.
                  \item[Initializing] - The job initializing. Usually this
                    is the UIMA-AS initialization phase.  In the default configuration, only
                    two (2) processes are allocated by the Resource Manager.  No additional
                    resources are allocated until at least one of the new processes successfully
                    completes initialization.  Once initialization is complete the Resource Manager
                    will double the number of allocated processes until the user's fair share of
                    the resources is attained.
                  \item[Running] - At least one process is now initialized and running. 
                  \item[Completing] - The last work item has completed and DUCC is freeing resources.
                    If the job had many resources allocated at the time the job exited this state
                    will persist until all allocated resources are freed.
                  \item[Completed] - The job is complete. 
              \end{description}
                  
            \item[Reason or Extraordinary Status] \hfill \\

              % See this structure:
              % org.apache.uima.ducc.transport.event.common.IDuccCompletionType
              
              This field contains miscellaneous information pertaining to the job.  If the job exits
              the system for any reason, that reason is shown here.  If the job's pre-requisite
              services are unavailable (or ailing) that fact is displayed here.  If there is a
              job monitor running, that fact is shown here.  Most of the values for this field
              support ``hovers'' containing additional information about the reason.
         
              \begin{description}
                  \item[EndOfJob] - The job and completed ran with no errors. 
                  \item[Error] - All work items are processes but at least one had an error. 
                  \item[CanceledByDriver] - The Job Driver (JD) terminated the job. The reason for
                    termination is seen by hovering over the text with your mouse.
                  \item[CanceledBySystem] - The job was canceled because DUCC was shutdown. 
                  \item[CanceledBySser] - The job owner or DUCC administrator canceled the job. 
                  \item[DriverInitializationFailure] - The Job Deiver (JD) process is unable to initialize. Hover over 
                    the field with your mouse for details (if any are available), and check your JD log. 
                  \item[DriverProcessFailed] - The Job Driver (JD) process failed for some reason. Hover over the 
                    field with your mouse for details (if any), and check your JD log. 
                  \item[MonitorActive] The job has a console monitor active.  This is enabled with the
                    job's ``wait\_for\_completion'' parameter on job submission.
                  \item[ServicesUnavailable] - The job declared a dependency on one or more services, and the 
                    Service Manager (SM) cannot find or start the required service. 
                  \item[Premature] - The job was terminated for some unknown reason before all work items were 
                    processed. Check the JP logs for details. 
                  \item[ProcessInitializationFailure] - Too many processes failed during
                    initialization and the job was canceled by DUCC.  Check the JP logs for the
                    reason.
                  \item[ProcessFailure] - Too many processes failed while running and DUCC canceled
                    the job.  Check the JP logs for the reason.
                  \item[ResourcesUnavailable] - The Resource Manager (RM) is unable to allocate resources for 
                    the job. For non-preemptable jobs this could be because the limit on that type of allocation is 
                    reached, or all the nodes are already allocated and work cannot be preempted to make space for 
                    it. For all jobs, it could be because the job class is invalid. 
                    \item[{\em service\_name}] If there is a service name in this field it indicates the job is
                      dependent on the service but the service is not responding to the Ducc Service Monitor's
                      pinger.
              \end{description}

            \item[Services] \hfill \\
              This is the number of services the job has declared dependencies on.  There is a ``hover'' that
              shows the ids of the services, if any.

            \item[Processes] \hfill \\
              This is the number of processes currently assigned to the job.

            \item[Init Fails] \hfill \\
              This is the total number of initialization failures experienced by the job. This
              field is hyperlinked to pages with log excerpts highlighting the specific failures.
              
            \item[Run Fails] \hfill \\
              This is the total number of process failures experienced by the job. This field is
              hyperlinked to pages with log excerpts highlighting the specific failures.
              
            \item[Pgin] This is the number of page-in events, over all processes, on the machines
              running the job.

            \item[Swap] This is the total swap space, over all the processes, being used by the job.

            \item[Size] \hfill \\
              This is the declared memory size of the job
              
            \item[Total] \hfill \\
              This is the total number of work items declared by the job.
              
            \item[Done] \hfill \\
              This is the total number of work items successfully completed for the job.
              
            \item[Error] \hfill \\
              This is the total number of exceptions thrown or other errors experienced by work
              items. This field is hyperlinked to pages containing log excerpts highlighting
              the failures.
              
            \item[Dispatch] \hfill \\
              This is the total number CASs that are currently dispatched. 

              This usually represents the quantity derived from the following formula:
\begin{verbatim}              
     min( (initialized.processes * threads.per.process), (incomplete.work.items - errors) )
\end{verbatim}

              The actual number is a measured number, not a calculated number, and may differ
              slightly from the formula if the measurement is taken immediately after process
              start-up, or in the time between a work item completing and a new one being
              dispatched.
              
            \item[Retry] \hfill \\
              This is the number of CASs that were retried for any reason.  Reasons for retry
              include preemption for fair-share, work-item timeout, or error conditions.

              Note: If a work item in any process fails, the entire process is considered
              suspect, and all work-items in the process are terminated.  Work items in the
              process which did not have errors are re-dispatched (retried) to a different
              process.
              
            \item[Preempt] \hfill \\
              This is the total number of processes that have been preempted to make room for
              other work due to Fair Share.
              
            \item[Description] \hfill \\
              This is the description string from the $--$description string from submit.
            \end{description}


      % Create well-known link to this spot for HTML version
      \ifpdf
      \else
      \HCode{<a name='DUCC_WS_JOB_DETAILS'></a>}
      \fi
        
    \section{Job Details Page}
    \label{sec:ws-job-details}

    This page shows details of all the processes that run in support of a job. 
    The information is divided among four tabs:
    \begin{description}
      \item[Processes] This tab contains details on all the processes for the job, both
        active, and defunct.
      \item[Work Items] This tab shows details for each individual work-item in the job.
      \item[Performance] This tab shows a performance break-down of all the UIMA analytics
        in the job.
      \item[Specification] This tab shows the job specification for the job.
      \end{description}
      
    \subsection{Processes}
    \label{sec:ws-processes}
    The processes page contains the following columns:
    
    \begin{description}

        \item[Id] \hfill \\
          This is the DUCC-assigned numeric id of the process (not the Operating System's
          processid). Process 0 is always the Job Driver.          

        \item[Log] \hfill \\
          This is the log name for the process. It is hyperlinked to the log itself.

        \item[Size] \hfill \\
          This is the size of the log in MB. If you find you have trouble viewing the log
          from the Web Server it could be because it is too big to view in the server and needs to
          be read by some other means than the Web Server.  (It is not currently paged in by 
          the Web Server, it is read in full.)

        \item[Hostname] \hfill \\
          This is the name of the node where the process ran.

        \item[PID] \hfill \\
          This is the Unix process ID (PID) of the process.

        \item[State Scheduler] \hfill \\
          % The information comes from here:
          % State Scheduler: org.apache.uima.ducc.transport.event.common.IResourceState.ResourceState

          This shows the Resource Manager state of the job. It is one of:
          \begin{description}
              \item[Allocated] - The node is currently allocated for this job by the RM.
              \item[Deallocated] - The resource manager has deallocated the shares for the job on
                this node.
          \end{description}

        \item[Reason Scheduler or extraordinary status] \hfill \\
          \phantomsection\label{itm:job-details-sched}


          % The information comes from here:
          % Reason Scheduler: org.apache.uima.ducc.transport.event.common.IResourceState.ProcessDeallocationType
          This column provides a reason for the scheduler state, when the scheduler state is other than ``Allocated''. 
          These all have ``hovers'' that provide more information
          if it is available.

            \begin{description}          
                \item[AutonomousStop] - The process terminated unexpectedly of its own accord ("crashed", or
                  simply exited.) 

                \item[Exception] - The process is terminated by the JD exception handler. 

                \item[Failed] - The process is terminated by the Agent because the JP wrapper was able to detect and 
                  communicate a fatal condition (Exception) in the pipeline.. 
                  
                \item[FailedInitialization] - The process is terminated because the UIMA initialization step failed. 
                  
                \item[Forced] - The node is preempted by RM for other work because of fair share. 
                  
                \item[JobCanceled] - The job was canceled by the user or a system administrator. 
                  
                \item[JobCompleted] - The process is canceled because of DUCC restart. 
                  
                \item[JobFailure] - The job failure limit is exceeded, causing the job to be canceled by the JD.                    
                  
                \item[InitializationTimeout] - The UIMA initialization phase exceeded the configured timeout. 
                  
                \item[Killed] - The agent terminated the process for some reason. The ``Reason Agent'' field
                  should have more details in this case.
          
                \item[Stopped]	- The process was terminated by the Agent for some reason.  The hover should
                  contain more information.
                          
                \item[Voluntary] - The job is winding down, there's no more work for this node, so it stops. 
                  
                \item[Unknown] - None of the above. This is an exceptional condition, sometimes an
                  internal DUCC error. Check the JP and JD logs for possible causes..
            \end{description}

          \item[State Agent] \hfill \\
          \phantomsection\label{itm:job-details-state}

          % This state comes from here:
          % State Agent: org.apache.uima.ducc.transport.event.common.IProcessState.ProcessState
            This shows the DUCC Agent's view of the state of the process.
            \begin{description}
               \item[Starting] The DUCC process manager as issued a request to the assigned to
                 start the process.
               \item[Initializing] The process is initializing.  Usually this means the UIMA analytic
                 pipeline (Job Process) is executing it's initialization method.
              \item[Running] The Job Process has completed the initialization phase and is ready for, 
                or actively executing work.
              \item[Stopped] The DUCC Agent reports the process is stopped and (and has exited).
              \item[Failed] The DUCC Agent reports the process failed with errors.  This usually
                means that UIMA-AS has detected exceptions in the pipeline and reported them
                to the Job Driver for logging.
              \item[FailedInitialization] The process died during the UIMA initialization phase.
              \item[InitializationTimeout] The process exceeded the site's limit for time spent
                in UIMA initialization.
              \item[Killed] The DUCC Agent killed the process for some reason.  There are
                three rosins for this:
                \begin{enumerate}
                  \item The Job Processes failed to initialize,
                  \item The Job Process timed out during initialization,
                  \item The process Exocet's its allowed swap.
                \end{enumerate}
              \item[Abandoned] It is possible to cancel a specific process of a job.  Usually
                this is because it became ``stuck'' because of hardware failure.  It a process
                is killed in \hyperref[sec:cli.ducc-cancel]{this way}, the state is recoreded as {\em Abandonded}.
            \end{description}
            
          \item[Reason Agent] \hfill \\
          \phantomsection\label{itm:job-details-agent}

          This shows extended reason information if a process exited other than having run out
          of work to do.

            \begin{description}
              \item[AgentTimedOutWatingForORState] The DUCC Agent is expecting a state update
                from the DUCC Orchestrator.  Timer on this wait has expired.  This usually 
                indicates an infrastructure or communication problem.
              \item[Croaked] The process exited for no good or clear reason, it simply vanished.
              \item[Deallocated] WHAT IS THIS?
              \item[ExceededShareSize] The process exceeded it's declared memory size.
              \item[ExceededSwapThreshold] The process exceeded the configured swap threshold.
              \item[FailedInitialization] The process was terminated because the UIMA 
                initialization step failed.
              \item[InitializationTimeout] The process was terminated because the UIMA initialization
                step took too long.
              \item[JPHasNoActiveJob] This is set when an agent looses connectivity while its
                JPs are running. The job finishes (stopped or killed). The agent regains
                connectivity. The OR publish no longer includes the job but the agent still has
                processes running for that job. The agent kills ghost processes with the reason:
                JPHasNoActiveJob.
              \item[LowSwapSpace] The process was terminated because the system is about to run
                out of swap space.  This is a preemptive measure taken by DUCC to avoid exhaustion
                of swap, to effect orderly eviction of the job before the operating system starts
                its own reaping procedures.
              \item[AdministratorInitiated] The process was canceled by an administrator.
              \item[UserInitated] The process was canceled by the owning user.
            \end{description}
            
          \item[Time Init] \hfill \\
            This is the clock time this process spent in initialization.
            
          \item[Time Run] \hfill \\
            This is the clock time this process spent in executing, not including
            initialization.
            
          \item[Time GC] \hfill \\
            This is amount of time spent in Java Garbage Collection for the process.
            
          \item[Count GC] \hfill \\
            This is the number of garbage collections performed by the process.
            
          \item[Pgin] \hfill \\
            This is the number of page-in events on behalf of the process.

          \item[Swap] \hfill \\
            This is the amount of swap space on the machine being consumed by the process.

          \item[\%GC] \hfill \\
            Percentage of time spent in garbage collections by this process, relative to total of
            initialization + run times.
            
          \item[\%CPU] \hfill \\
            Currant CPU percent consumed by the process.  This will be $>$ 100\% on 
            multi-core systems if more than one core is being used.  Each core contributes
            up to 100\% CPU, so, for example, on a 16-core machine, this can be as high
            as 1600\%.
            
          \item[RSS] \hfill \\
            The amount of real memory being consumed by the process (Resident Storage Size)
            
          \item[Time Avg] \hfill \\
            This is the average time in seconds spent per work item in the process.
            
          \item[Time max] \hfill \\
            This is the minimum time in seconds spent per work item in the process.
            
          \item[Time min] \hfill \\
            This is the minimum time in seconds spent per work item in the process.
            
          \item[Done] \hfill \\
            This is the number of work items processed in this process.
            
          \item[Error] \hfill \\
            This is the number of exceptions processing work items in this process.
            
          \item[Retry] \hfill \\
            This is the number of work items that were retried in this process for any reason, excluding
            preemption.
            
          \item[Preempt] \hfill \\
            This is the number of work items that were preempted from this process, if
            fair-share caused preemption.
            
          \item[JConsole URL] \hfill \\
            This is a URL that can be used to connect via JMX to the processes, e.g. via
            jconsole.

      \end{description}

   \subsection{Work Items}
   \label{sec:ws-work-items}
   This tab provides details for each individual work item.  Columns include:

   % The data comes from here: org.apache.uima.ducc.common.jd.files.IWorkItemState.State    
   \begin{description}
     \item[SeqNo]  \hfill \\
       This is the sequence work items are fetched from the Collection Reader's
       getNext() method by the DUCC Job Driver.
     \item[Id]  \hfill \\
       This is the name of the work item.
     \item[Status]  \hfill \\
       The is the current state of the work item.  
       States include:
       \begin{description}
         \item[ended] The work item is complete.
         \item[error] The work item ended with errors.
         \item[operating] The work item is current being executed.
         \item[retry] The work item is being retried.
         \item[start] The work item has been picked up for execution and DUCC is waiting
           for confirmation that it is running.
         \item[queued] The work item has been queued to ActiveMQ but not picked up by any
           Job Process yet.
       \end{description}
       If a work item has not yet been retrieved from the Collect Reader it does not show
       on this page.
     \item[Queuing Time (sec)]  \hfill \\
       The time spent in ActiveMQ after being queued, and before
       being picked up by a Job Process.
     \item[Processing Time (sec)]  \hfill \\
       The time spent processing the work item.
     \item[Node (IP)]  \hfill \\
       The node IP where the work item was processed.
     \item[Node (Name]  \hfill \\
       The node name where the work item was processed.
     \item[PID]  \hfill \\
       The Unix Process Id that the work item was processed in.
   \end{description}
   

   \subsection{Performance}
   \label{sec:performance}
   This tab shows performance summaries of all the pipeline components.  The statistics
   are aggregated over all instances of each component in each process of the job.
   
   \begin{description}
     \item[Name]  \hfill \\
       The short name of the analytic.  The full name is shown in the command-line
       tool \hyperref[sec:cli.ducc-perf-stats]{ducc\_perf\_stats}
     \item[Total]  \hfill \\
       This is the total time in days, hours, minutes, and seconds taken by each
       component of the pipeline.
     \item[\% of Total]  \hfill \\
       This is the percent of the total usage consumed by this analytic.
     \item[Avg]  \hfill \\
       This is the average time spent by all the instances of the analytic.
     \item[Min]  \hfill \\
       This is the minimum time spent by any instance of the analytic.
     \item[Max]  \hfill \\
       This is the maximum time spent by any instance of the analytic.
   \end{description}
   
   \subsection{Specification}
   This tab shows the full job specification in the form of a Java Properties
   file.  This will include all the parameters specified by the user, plus those
   filled in by DUCC.


      % Create well-known link to this spot for HTML version
      \ifpdf
      \else
      \HCode{<a name='DUCC_WS_RESERVATIONS'></a>}
      \fi
      
\section{Reservation Page}
\label{sec:ws-reservations}

This page shows details of all reservations.  There are two types of reservations: {\em managed}
and {\em unmanaged.}.

A {\em managed reservation} is a reservation whose process is fully managed by DUCC.  This process
is any arbitrary process and is submitted with the
\hyperref[sec:cli.ducc-process-submit]{ducc\_process\_submit} CLI.  The lifetime of the reservation
starts at the time DUCC assignes a unique ID, and ends when the process terminates for any reason.

An {\em unmanaged reservation} is essentially a sandbox for the user.  DUCC starts no processes
in the reservation and manages none of the processes which run on that node.  The lifetime of the
reservation starts at the time DUCC assigns a unique ID, and ends when the submitter or system
administrator cancels it.  {\em Managed reservations} can potentially last an indefinite
period of time.

The Reservations page contains the following columns: 
\begin{description}

\item[Id] \hfill \\
  This is the unique DUCC numeric id of the reservation as assigned when the reservation is made.
  If this is a {\em managed} reservation, the ID is hyperlinked to a
  \hyperref[sec:ws-managed-reservation-details]{Managed Reservation Details} page with extended
  details on the process running in the reservation.

\item[Start] \hfill \\
  This is the time the reservation was mde.
  
\item[End] \hfill \\
  This is the time the reservation was canceled or otherwise ended.
  
\item[User] \hfill \\
  This is the userid if the person who made the reservation.
  
\item[Class] \hfill \\
  This is the scheduling class used to schedule the reservation.
  
\item[Type] \hfill \\
  This is the reservation type, {\em managed} or {\em unmanaged}, as described 
  \hyperref[sec:ws-reservations]{above}.

\item[State] \hfill \\
  This is the status of the reservation. Values include: Received - Reservation
  has been vetted, persisted, and assigned unique Id.
  \begin{description}
  \item[WaitingForResources] - The reservation is waitng for the Resource Manager to find and 
    schedule esources. 
  \item[Assigned] - The reservation is active. 
  \item[Completed] - The reservation has been terminated.
  \end{description}

\item[Reason] \hfill \\
  If a reservation is not active, the reason. Reasons include:
  \begin{description}
  \item[CanceledBySystem] - The job was canceled because DUCC was shutdown. 
  \item[CanceledByUser] - The owner or administrator released the reservation. 
  \item[ResourcesUnavailable] - The Resource Manager was unable to find free or freeable resources 
    match the resource request. 
  \item[ProgramExit] - The reservation is a {\em managed} reservation and the associated
    process has exited.
  \end{description}

\item[Allocation] \hfill \\
  This is the number of resources (shares for FIXED policy reservartions, processes for
  RESERVE policy reservations) that are allocated.

\item[UserProcesses] This is the number of processes owned by the user running in all
  shares of the reservation.  
  
  Note that even for {\em unmanaged} reservations, the DUCC agent tracks processes owned
  by the user and reports on them.  This allows better identification and management of
  abandonded reservations.

\item[Size] \hfill \\
  The memory size in GB of the each allocated unit.  This is the amount of memory that
  was {\em requested}.  In the case of RESERVE policy reservations, that actual memory
  of the reserved machine may be greater.
  
\item[Host Names] \hfill \\
  The node names of the machines where the resources are allocated.
  
\item[Description] \hfill \\
  This is the descriptin string from the --description string from submit.
\end{description}



      % Create well-known link to this spot for HTML version
      \ifpdf
      \else
      \HCode{<a name='DUCC_WS_RESERVATIONS_DETAILS'></a>}
      \fi
      % 
% Licensed to the Apache Software Foundation (ASF) under one
% or more contributor license agreements.  See the NOTICE file
% distributed with this work for additional information
% regarding copyright ownership.  The ASF licenses this file
% to you under the Apache License, Version 2.0 (the
% "License"); you may not use this file except in compliance
% with the License.  You may obtain a copy of the License at
% 
%   http://www.apache.org/licenses/LICENSE-2.0
% 
% Unless required by applicable law or agreed to in writing,
% software distributed under the License is distributed on an
% "AS IS" BASIS, WITHOUT WARRANTIES OR CONDITIONS OF ANY
% KIND, either express or implied.  See the License for the
% specific language governing permissions and limitations
% under the License.
% 
\section{Managed Reservation Details Page}
\label{sec:ws-managed-reservation-details}

This page shows details of the processes which run in a managed reservation.  The
information is divided between two tabs:

   \begin{description}
       \item[Processes] This tab contains details on all the proceses contained in the
         reserved space.
       \item[Specification] This tab shows the specification for the process.
   \end{description}  

   \subsection{Processes}
   \label{sec:ws-manres-processes}

   The processes page contains the following columns:
   \begin{description}
      \item[ID] \hfill \\
        This is the DUCC-assigned numeric id of the process.  This format of this
        id is two numbers:
\begin{verbatim}
    RESID.SHAREID
\end{verbatim}
        Here, the {\em RESID} is the reservation ID.  The {\em SHAREID} is the 
        share ID assigned by the Resource Manager.  Together these form a unique
        ID for each process that runs in the reservation.
        
        Note: The current version of DUCC supports only one process per managed
        reservation.  Future versions are expected to support multiple processes
        within a single managed reservation.
        
      \item[Log] \hfill \\
        This is the log name for the process. It is hyperlinked to the log itself.
        
      \item[Size] \hfill \\
        This is the size of the log in MB. If you find you have trouble viewing the log
        from the web server it could be because it is too big to view in the browser.
        
      \item[Hostname] \hfill \\
        This is the name of the node where the process is running (or ran).
        
      \item[PID] \hfill \\
        This is the Unix process ID (PID) of the process.
        
      \item[State Scheduler] \hfill \\
        This shows the Resesource Manager state of the job. It is one of:
        
        \begin{description}
            \item[Allocated] - The node is still allocated for this job by the RM.
            \item[Deallocated] - The resource manager has deallocated the shares for the job on
              this node.
        \end{description}
        
      \item[Reason Scheduler or Extraordinary Status] \hfill \\
        These are the same as for the \hyperref[itm:job-details-sched]{job details.}

      \item[State Agent] \hfill \\
        These are the same as for the \hyperref[itm:job-details-state]{job details.}

      \item[Reason Agent] \hfill \\
        These are the same as for the \hyperref[itm:job-details-agent]{job details.}

      \item[Time Run] \hfill \\
        The current duration of the reservation, or total duration if it has 
        terminated.
        
      \item[RSS] \hfill \\
        The amount of real memory being consumed by the process (Resident Storage Size)

   \end{description}

   \subsection{Specification}
   \label{sec:ws-service-specification}
   This tab shows the full job specification in the form of a Java Properties
   file.  This will include all the parameters specified by the user, plus those
   filled in by DUCC.
        


      % Create well-known link to this spot for HTML version
      \ifpdf
      \else
      \HCode{<a name='DUCC_WS_SERVICES'></a>}
      \fi
      % 
% Licensed to the Apache Software Foundation (ASF) under one
% or more contributor license agreements.  See the NOTICE file
% distributed with this work for additional information
% regarding copyright ownership.  The ASF licenses this file
% to you under the Apache License, Version 2.0 (the
% "License"); you may not use this file except in compliance
% with the License.  You may obtain a copy of the License at
% 
%   http://www.apache.org/licenses/LICENSE-2.0
% 
% Unless required by applicable law or agreed to in writing,
% software distributed under the License is distributed on an
% "AS IS" BASIS, WITHOUT WARRANTIES OR CONDITIONS OF ANY
% KIND, either express or implied.  See the License for the
% specific language governing permissions and limitations
% under the License.
% 

    \section{Services Page}
    \label{ws:services-page}
        This page shows details of all services.           

        The Services page contains the following columns: 
        \begin{description}

            \item[Id] \hfill \\
              This is the unique numeric {\DUCC} id of the service.  This ID is hyperlinked to a
              \hyperref[sec:ws-service-details]{Service Details} page with extended
              details on the service.  Note that for some types of services, {\DUCC} may not
              know more about the service than is shown on the main page.

            \item[Name] \hfill \\
              This is the unique service endpoint of the service.  
              
            \item[State] \hfill \\
              This is the state of the service with respect to the service manager.  It is a
              consolidated state over all the service instances.  Valid states are
              \begin{description}
                \item[Available] At least one service instance is responding to the service
                  pinger, indicating it is functional.
                \item[Initializing] No service instances are available for use yet but at least one instance
                  is in its UIMA {\em initializing} phase.
                \item[Waiting] At least one service instance is in Running state, potentially available for use,
                  but no response has been received from the service pinger.  This usually occurs during the
                  start-up of a service.  If a service stops responding to its pinger after becoming
                  available, the state can regress to Waiting.
                \item[NotAvailable] No service instance is running or initializing. 
                \item[Stopped] The service has been stopped.
                \item[Stopping] The service has been stopped for some reason, but not all 
                  instances have terminated.  This is an intermediate state between Available and
                  NotAvailable to signify that the service is no longer available but not all its
                  resources have been returned yet.
              \end{description}

              {\DUCC} will start dependent jobs ONLY if its services are in state Available.  Otherwise
              {\DUCC} attempts to start the service, and if successful, allows the job to start.  

              If a job is already running and a service becomes other than Available, the
              \hyperref[sec:ws.jobs-page]{jobs page} indicates the service is not available but the job is 
              allowed to continue.
              
            \item[Last Use] \hfill \\
              The time this service was last used.
              
            \item[Instances] \hfill \\
              This is the number of instances (processes) currently registered for the service.  

            \item[Deployments] \hfill \\
              This is the number of actual instances deployed for the service.  Note that this may
              be greater, or less, than the number of registered instances, if the service owner
              decides to temporarily start or stop additional instances.

            \item[Start State] \hfill \\
              This service start state.

            \item[User] \hfill \\
              This is the userid of the service owner.
              
            \item[Class] \hfill \\
              This is the scheduling class the service is running in. 
              
              If a service is registered as ``ping-only'', no resources are allocated for it.  This
              is shown as a class of {\tt ping-only}.
                
      		\item[PgIn] \hfill \\
        	This is the number of page-in events on behalf of the service.

      		\item[Swap] \hfill \\
        	This is the amount of swap space consumed by the service.
        
            \item[Memory (registered)] \hfill \\
              This is the memory size, in GB, of each service instance

            \item[Jobs] \hfill \\
              This is the number of jobs currently using the service.  The IDs of the jobs are
              shown as hovers over this field.

            \item[Services] \hfill \\
              Services may themselves depend on other services.  This field shows the number of
              services dependent on this service.  The dependent service IDs are shown with a 
              hover over the field.

            \item[Reservations] \hfill \\
              This field shows the number of
              managed reservations dependent on this service. The IDs of the managed reservations
              are shown as a hover over the field.

              
            \item[Description] \hfill \\
              This is the description string from the --description string from submit.
        \end{description}


      % Create well-known link to this spot for HTML version
      \ifpdf
      \else
      \HCode{<a name='DUCC_WS_SERVICE_DETAILS'></a>}
      \fi
      \section{Service Details Page}
\label{sec:ws-service-details}

This page shows details of the processes which implement.  Note that in the case of
\hyperref[sec:services.implicit]{implicit} and \hyperref[subsub:services.ping-only]{ping-only}
services there will be no processes to show.

The information is divided between two tabs:

   \begin{description}
       \item[Processes] This tab contains details on all the proceses implementing
         the service, if any.
       \item[Specification] This tab shows the specification for the service.  In the
         case of \hyperref[sec:services.implicit]{implicit} services, this shows the generated Service Manager
         state for the service.
   \end{description}  

   \subsection{Processes}
   \label{sec:ws-services-processes}

   The processes page contains the following columns:
   \begin{description}
      \item[ID] \hfill \\
        This is the DUCC-assigned numeric id of the process.  This format of this
        id is two numbers:
\begin{verbatim}
    RESID.SHAREID
\end{verbatim}
        Here, the {\em RESID} is the reservation ID.  The {\em SHAREID} is the 
        share ID assigned by the Resource Manager.  Together these form a unique
        ID for each process that runs in the reservation.
                
      \item[Log] \hfill \\
        This is the log name for the process. It is hyperlinked to the log itself.
        
      \item[Size] \hfill \\
        This is the size of the log in MB. If you find you have trouble viewing the log
        from the web server it could be because it is too big to view in the browser.
        
      \item[Hostname] \hfill \\
        This is the name of the node where the process is running (or ran).
        
      \item[PID] \hfill \\
        This is the Unix process ID (PID) of the process.
        
      \item[State Scheduler] \hfill \\
        This shows the Resesource Manager state of the job. It is one of:
        
        \begin{description}
            \item[Allocated] - The node is still allocated for this job by the RM.
            \item[Deallocated] - The resource manager has deallocated the shares for the job on
              this node.
        \end{description}
        
      \item[Reason Scheduler or Extraordinary Status] \hfill \\
        These are the same as for the \hyperref[itm:job-details-sched]{job details.}

      \item[State Agent] \hfill \\
        These are the same as for the \hyperref[itm:job-details-state]{job details.}

      \item[Reason Agent] \hfill \\
        These are the same as for the \hyperref[itm:job-details-agent]{job details.}


      \item[Time Init] \hfill \\
        Most services are UIMA-AS services and therefore have an {\em initialization} phase
        to their lifetimes.  This field shows the time spent in that phase.

      \item[Time Run] \hfill \\
        The current duration of the reservation, or total duration if it has 
        terminated.
        
      \item[Time GC] \hfill \\
        This is amount of time spent in Java Garbage Collection for the process.

      \item[Pgin] \hfill \\
        This is the number of page-in events on behalf of the process.
        
      \item[Swap] \hfill \\
        This is the amount of swap space on the machine being consumed by the process.
        
      \item[\%CPU] \hfill \\
        Currnt CPU percent consumed by the process.  This will be $>$ 100\% on 
        multi-core systems if more than one core is being used.  Each core contributes
        up to 100\% CPU, so, for example, on a 16-core machine, this can be as high
        as 1600\%.

      \item[RSS] \hfill \\
        The amount of real memory being consumed by the process (Resident Storage Size)

      \item[JConsole URL] \hfill \\
        This is a URL that can be used to connect via JMX to the processes, e.g. via
        jconsole.

   \end{description}

   \subsection{Specification}
   \label{sec:ws-managed-reservation-specification}
   This tab shows the full job specification in the form of a Java Properties
   file.  This will include all the parameters specified by the user, plus those
   filled in by DUCC.
        
   The specification for a Service contains two types of entries:
   \begin{enumerate}
     \item Service specification properties, prefixed with ``svc''. These comprise
       the service specification that the Service Manager submits on behalf of
       a user in order to start registered services.
     \item Meta properties, prefixed with ``meta''.  This is the Service Manager's state
       record for the sesrvice as it is running.  In addition to state it contains
       properties required for service registration that are not used for
       service submission.
   \end{enumerate}
   


      % Create well-known link to this spot for HTML version
      \ifpdf
      \else
      \HCode{<a name='DUCC_WS_SYSTEM></a>}
      \fi
      % 
% Licensed to the Apache Software Foundation (ASF) under one
% or more contributor license agreements.  See the NOTICE file
% distributed with this work for additional information
% regarding copyright ownership.  The ASF licenses this file
% to you under the Apache License, Version 2.0 (the
% "License"); you may not use this file except in compliance
% with the License.  You may obtain a copy of the License at
% 
%   http://www.apache.org/licenses/LICENSE-2.0
% 
% Unless required by applicable law or agreed to in writing,
% software distributed under the License is distributed on an
% "AS IS" BASIS, WITHOUT WARRANTIES OR CONDITIONS OF ANY
% KIND, either express or implied.  See the License for the
% specific language governing permissions and limitations
% under the License.
% 

\section{System  Details Page}
\label{sec:system-details}

This page shows information relating to the DUCC System itself:
\begin{description}
  \item[Administration]This displays system administrators and implements
    the interface to various administrative controls.
  \item[Classes] This shows the current system's scheduling class definitions.
  \item[Daemons] This shows the status of all DUCC processes.
  \item[DuccBook] This is a link to the book you are reading.
  \item[Machines] This shows details of all the machines in the DUCC cluster.
\end{description}

\subsection{Administration}

   This page has two tabs:
   \begin{description}   
     \item[Administrators] This shows the user-ids that are authorized to administer
       DUCC.  In addition to executing the ``Control'' functions described below,
       administrators may cancel any job, reservation, or service, and may modify
       services they do not own.  

       In order to perform administrative functions, the following must be satisfied:
       \begin{enumerate}
         \item The user is logged-in to the web server.
         \item The user is a registered administrator.
         \item The user has set the role as ``administrator'' in the DUCC Preferences
           page.  This is a safeguard so that administrators who are also users
           are less likely to inadvertently affect other people's jobs.
       \end{enumerate}
     \item[Control] Currently DUCC supports a single administrative control function
       via the web server: Stop new job submissions and re-enable them.  If submissions
       are blocked, all existing work runs normally, but no new work is accepted.
     \end{description}


\subsection{Classes}
This page shows the definitions of the DUCC scheduling classes.  The scheduling classes are
discussed in more detail in the \hyperref[sec:rm.job-classes]{Resource Manager} section.

\subsection{Daemons}
\label{sec:system-details.daemons}

This page shows the current state of all DUCC processes.  By default, only the administrative
processes, Orchestrator, ProcessManager, ResourceManager, ServiceManager, and Webserver are
shown.  A button in the upper left of the page titled ``Show Agents'' enables display of
the status of all the DUCC agents as well. (Agents are suppressed by default because the
page is expensive to render for large systems.)

The columns shown on this page include

   \begin{description}
      \item[Status] \hfill \\
        This indicates whether the daemon is running and broadcasting state {\em up},
        or not {\em down}.  
        
        All DUCC daemons broadcast a heartbeat containing process state.  If the Status
        is {\em down}, either the daemon is not functioning, or something is preventing
        state from reaching the web server via DUCC's ActiveMq instance.

      \item[Daemon Name] \hfill \\
        This is the name of the process.

      \item[Boot Time] \hfill \\ 
        This shows the date and time of the latest boot of the specific process.
          
      \item[Host IP] \hfill \\ 
        This is the IP address of the processor where the process is running.

      \item[Host Name] \hfill \\ 
        This shows the hostname of the processor where the process is running.

      \item[PID] \hfill \\ 
        This is the Unix processid of the DUCC process.


      \item[Publication Size (last)] \hfill \\ 
        This shows the size of the most recent state publication of the process, in bytes.

      \item[Publication Size (max)] \hfill \\ 
        This shows the size of the largest state publication of the process, in bytes.

      \item[Heartbeat (last)] \hfill \\ 
        This shows the number of seconds since the last state publication for the process. 
         Large numbers here indicate potential cluster or DUCC problems.

      \item[Heartbeat (max)] \hfill \\ 
        This shows the longest delay since a state publication for the process was received
        at the web server.  Large numbers here indicate potential cluster or DUCC problems.

      \item[Heartbeat (max) TOD] \hfill \\ 
        This shows the time the longest delay of a state publication occurred.

      \item[JConsole URL] \hfill \\ 
        This is the jconsole URL for the process.

   \end{description}
      
\subsection{Machines}

This page shows the states of all the machines in the DUCC cluster.

The columns shown on this page include

   \begin{description}
      \item[Status] \hfill \\
        This shows the current state of a machine.  Values include:
        \begin{description}
          \item[defined] The node is in the DUCC
            \hyperref[sec:admin-ducc.nodes]{nodes file}, but no DUCC process has been
            started there, or else there is a communication problem and
            the state messages are not being delivered.
            \item[up] The node has a DUCC Agent process running on it and the
              web server is receiving regular heartbeat packets from it.
            \item[down] The node had a healthy DUCC Agent on it at some point
              in the past (since the last DUCC boot), but the web server has stopped
              receiving heartbeats from it. 

              The agent may have been manually shut down, may have crashed, or there
              may be a communication problem.

              Additionally, very heavy loads from jobs running the the node can cause
              the DUCC Agents heartbeats to be delayed.
        \end{description}


      \item[IP] \hfill \\
        This is the IP address of the node.

      \item[Name] \hfill \\
        This is the hostname of the node.

      \item[Reserve(GB) size] \hfill \\
        This is the largest reservation that can be made on this node.

        This is usually somewhat less than the physical memory size because it is 
        rounded down to the nearest \hyperref[chap:rm]{share quantum}.  The purpose of this
        column is to assist users in requesting the right size for full machine 
        reservations.

      \item[Memory(GB) total] \hfill \\
        This is the amount of memory, in GB, as reported by each machine.
        
        Usually the amount will be slightly less than the installed memory.  This is because
        a small bit of memory is usually reserved by the hardware for its own purposes.  For 
        example, a machine with 48GB of installed memory may report only 47GB available.

      \item[Swap(GB) in use] \hfill \\
        This is the total size in-use swap data.  DUCC shows any value greater than 0 in
        red as swapping can very significantly slow applications.  However, swap use does
        not always mean there is a performance problem.  This is flagged by DUCC simply
        as an alert of a potential problem

      \item[Alien PIDs] \hfill \\
        This shows the number of processes not owned by DUCC, the operating system, or
        jobs scheduled on each node.  The Unix Process IDS of these processes is displayed
        in a hover.

        DUCC preconfigures many of the standard operating 
        \hyperref[itm:props-rogue.process]{system process} and 
        \hyperref[itm:props-rogue.user]{userids}.  This list may be updated by each
        installation.

        A common cause of alien PIDs is errant process run in unmanaged reservations.  A
        user may reserve a machine for use as a sandbox.  If the reservation is released
        without properly terminating all the processes, they may linger.  When ducc 
        schedules the node for other purposes, significant performance penalties may be
        paid due to competition between the legitimately scheduled work and the leftover
        ``alien'' processes.  The purpose of this column is to bring attention to this situation.

      \item[Shares (total)] \hfill \\
        This shows the total number of scheduling share supported on this node.

      \item[Shares(in use)] \hfill \\
        This shows the total number of scheduling share in use on the node.

      \item[Heartbeat(last)] \hfill \\
        This shows the number of seconds since the last agent heartbeat from this machine.

      \end{description}
      




%% TODO TODO This needs breakout into its own file

      \section{Service Management}
          TODO TODO TODO BREAK OUT INTO SEPARATE SECTION
          THIS IS A TEMPORARY HOLDING SPOT 

      \paragraph{Overview.} 
      Services, in the context of DUCC, are long-running processes that await requests from
      UIMA pipeline components and return something in response. Services can be any arbitrary process
      using any arbitrary communication protocol but in the current version of DUCC only UIMA-AS
      services are fully supported.

      The DUCC service manager implements several high-level functions:
      
      Insure services are available for jobs before allowing the jobs to start. This fail-fast
      prevents unncessary allocation of resources (with potential eviction of healthy processes) for
      jobs that can't run, as well as quick feedback to users that something is amis.
      
      Automate the startup, care, and management of services.
      
      Report on the state of services: processes, queue depths, comsumers, and so on.  

      \paragraph{Service Types.}
      DUCC supports two types of services: UIMA-AS and CUSTOM:
      
      \begin{description}
          \item[UIMA-AS] This is a "normal" UIMA-AS service. DUCC fully supports all aspects of UIMA-AS
            services.
            
          \item[CUSTOM] This is any arbitrary service. DUCC supports monitoring of CUSTOM services
            and performs job dependency checks, but (in the current version) does not support start
            and stop of CUSTOM services.
      \end{description}

      \paragraph{Service Endpoints.} Services are referenced by a specifier called a service
      endpoint.. The service endpoint is a formatted string indicating:

      \begin{itemize}
         \item The service type: UIMA-AS or CUSTOM.

         \item The service name. For UIMA-AS services, this is the name of the queue in the ActiveMq
           Broker used for communication with the service. For CUSTOM services this is any arbitrary
           string as dictated by the service. Service names must be unique within the system.

         \item For UIMA-AS services only, the URL of the ActiveMq broker.  
      \end{itemize}

      \paragraph{Dependent and Pre-Requisite Services and Jobs.} A {\em dependent service} is a
      service which is dependent on at least one other service to perform it's function. A {\em
        dependent job} is a job which is dependent on at least one service to perform it's function.

      An {\em independent service} service is a service which is required by another job or
      service. (Note that there are no independent jobs.)

      \paragraph{Service Classes.} Services may be started externally to DUCC, explicitly through
      DUCC as a job, or as registered services. These form three natural classes of services with
      slightly different management characteristics.

      \paragraph{Implicit Services.} An implicit service is started externally to DUCC and discovered by DUCC only
      when it is referenced by a job's service-dependency parameter. On submission of a job with a
      dependency on an implicit service, the SM sets up a "ping" thread that check if the service
      exists at the endpoint. If so, the SM adds the service to its list of known services and marks
      the job "ready to schedule". If the service is a UIMA-AS service the SM establishes a monitor
      thread on the queue for reporting purposes. The service is monitored throughout the lifetime of
      the job. If the service should stop responding, its state is updated as "not-responding" but the
      job is allowed to continue as DUCC cannot tell if the job is still using it or not, or if the
      outage is temporary. If the job is a CUSTOM service, the service owner may specifiy custom code
      to run in the ping thread; for CUSTOM services, this same code is used to run both ping and
      monitor functions.
      
      When the job exits, a timer is set and DUCC continues to monitor the service against the
      possibility that subsequent jobs will need it. Once the last job using the service has exited
      and the service timer expired, the SM stops the monitors and purges the service from its
      records.
      
      \paragraph{Submitted Services.} A submitted service is a service that is submitted to DUCC as a job. A
      submitted service is essentially a normal DD-style job (a job in which the user supplies the
      full UIMA-AS DD), but without a Collection Reader. Because DUCC is managing this service it can
      provide more support than for implicit services.
      
      Submitted services can be dependent upon other services. When such a service enters the system,
      DUCC verifies it's pre-requisite services. When (or if) all pre-requisite services are availble
      DUCC marks the new service "ready to schedule". The lifecycle of the service is monitored so
      that dependent services and jobs are marked "ready to schedule" only after the submitted service
      has completed its initialization phase. A ping thread and queue monitor are also started against
      the newly submitted service. If the submitted service is unable to successfully initialize,
      services and jobs that are dependent on it are marked "not runnable" and the DUCC Orchestrator
      cancels them.
      
      DUCC manages the lifecycle of submitted services, but because they are submitted by entities
      other than DUCC, the SM performs no additional management for them. When a submitted service is
      canceled by its owner, DUCC stops the ping and queue monitors. Any jobs or services dependent on
      it are allowed to continue until they complete or fail due to unavailability of the service.
      
      \paragraph{Registered Services.} Registered services are fully managed by DUCC. A service is
      registered with DUCC using the CLI to provide the full job specification of the service, the
      initial number of instances of the service, and whether the service should be automatically
      started when DUCC itself is started. Registered services started when DUCC is started are
      called automatic services.  Registered services that are started only when referenced by other
      dependent jobs or services are called on-demand services. The service is registered with the
      submitter's credentials and is run with that user's credentials when it is started.

      \todo Fix and properly place this paragraph.
          Ping and monitor threads are started. Jobs and other services may use these services in the same
          manner as submitted services. If an automatic service instance should die or be canceled out of
          the scope of the SM, the SM will restart the instance, maintaining the registered number of
          instances at all time. Automatic services are not terminated when their dependent jobs/services
          exit; they're termanted only when DUCC itself is terminated, or by use of the service stop
          command.

      There are several subclasses of Registered Services:
      \begin{description}

        \item[Automatic Services] An automatic service is a registered service that is flagged to be
          automatically started when the DUCC system is started. When DUCC is started, the SM checks the
          service registry for all service that are marked for automatic startup. The SM submits the
          registered service specification on behalf of its owner. Each such submission is for a single
          service instance.  If found, the SM repeatedly submits the specification until the registered
          number of instances is reached.
          
        \item[On-Demand Services] An on-demand service is a registered service that is started only when
          referenced by the service-dependency of another job or service. f the service is already
          started, the dependent job/service is marked ready to schedule as indicated above. If not, the
          service registry is checked and if a start-on-demand service with an endpoint matching the
          service-dependency is found, DUCC submits the service on behalf of the service owner (in the
          same manner as for automatic servic establishing the registered number of service instances, a
          ping thread, and a monitor). When the service has completed initialization the dependent
          job/service is marked ready to schedule. If the on-demand service cannot be found in the
          registery, the referring entity is marked not-startable and the DUCC Orchestrator cancels it.
          
          Subsequent jobs and services that reference the on-demand service will use the started
          instances.  When the last job/service that references the on-demand service exits, a
          (configurable) timer is established to keep the service alive for a while (in anticipation that
          it will be needed again soon.)  When the keep-alive timer exipires, and there are no more
          dependent jobs/services, the on-demand service is automatically stopped to free up its resources
          for other work.

          \item[External Services] External services consist of only a ping thread.  The service
            itself is not managed in any way by DUCC.  This is useful for managing dependencies
            on services that are not under DUCC control: DUCC can detect and report on the health
            of these services and take appropriate actions on dependent jobs if the services
            are not responsive.
      \end{description}
          
    \paragraph{Registered Service Management.} The CLI for registered services provides several functions:

    \begin{description}
        \item[Register] Register files a service specification with the SM. The service may optionally
          be started as part of registration. The service definition and state is persisted over system
          restarts and is deleted only with the Unregister function.
          
        \item[Unregister] Unregister removes the service specification. The service is stopped if it is
          started and not busy. (Note that if the service is busy, jobs and services that are dependent
          on it may subsequently fail.)
          
        \item[Modify] Modify allows dynamic update of some parameters of registered services:
            \begin{itemize}
              \item Automatic and On-Demand state.
              \item The minimum number of service instances to start when the service is started.  
            \end{itemize}

        \item[Start] Start submits the service specification to the DUCC Orchestrator (repeatedly,
          until the correct number of instances are started). If the service is explicitly started
          with the start CLI, the service continues to run even after the last reference is gone,
          regardless of whether it is automatic or on-demand. Start is also used to increase the
          number of running instances of a service. The registry may be optionally updated to
          reflect the new number of started instances.
          
        \item[Stop] Stop stops the instances for a registered service. The registry may be
          optionally updated to reflect the new number of instances that are still running.

        \item[Query] A CLI-based query is supplied to report on all services known to DUCC, their
          states, their instances, their dependent jobs, and performance statistics for the service.
    \end{description}
        
