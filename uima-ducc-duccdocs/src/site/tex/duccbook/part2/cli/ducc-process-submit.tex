% Create well-known link to this spot for HTML version
\ifpdf
\else
\HCode{<a name='DUCC_CLI_PROCESS_SUBMIT'></a>}
\fi
    \section{ducc\_process\_submit}

    \paragraph{Description:}
       Usse {\em ducc\_process\_submit} to submit a Managed Reservation, also known as an
       arbitrary process to DUCC.  The process launched is any process at all.  The intention
       of this function is an alternative to utilities such as {\em ssh}, such that the
       spawned processes are fully managed by DUCC.  This allows the DUCC scheduler to allocate
       the necessary resources (and prevent over-allocation), and the DUCC runtime environment
       to manage process lifetime.

       If {\em process\_attach\_console} is specified, Stdin, Stderr, and Stdout of the remote
       process are redirected to the submitting console.  It is thus possible to run interactive
       sessions with remote processes where the resources are managed by DUCC.

    \paragraph{Usage:}
    \begin{description}
    \item[Executable Jar] java -jar \ducchome/lib/uima-ducc-process-submit.jar {\em options}
    \item[Script wrapper] java -jar \ducchome/bin/ducc-process-submit {\em options}
    \item[Java Main]      java -cp \ducchome/lib/uima-ducc-process-submit.jar org.apache.uima.ducc.cli.DuccManagedReservationCancel {\em options}
    \end{description}

    \paragraph{Options:}
    \begin{description}
    
        \item[--cancel\_managed\_reservation\_on\_interrupt ] Cancel managed reservation on interrupt
          (Ctrl-C).  If running with {\em--wait\_for\_completion} and this flag is specified,
          terminating the submit process will result in the remote process being terminated.

        \item[--description {[text]}] The text is any string used to describe the process. It is
          displayed in the Web Server.

        \item[--debug ] Prints internal debugging information, intended for DUCC developers or
          extended problem determination.

        \item[--environment {[env vars]}] Blank-delimeted list of environment variables.  example:
          "TERM=xterm DISPLAY=me.org.net:1.0".  If specified, this is set in the remote process as
          it is spawned.

        \item[--help] Prints the usage text to the console.

        \item[--log\_directory {[path-to-log directory]} ] This specifies the path to the directory
          for the user logs. If not specified, the default is the user's home directory.
          
          Within this directory DUCC creates a subdirectory for each job, using the numerical 
          ID of the job. The format of the generated log file names is descripbed in Chapter 5, Job 
          Logs [47]. 

        \item[--process\_attach\_console] If specified, redirect remote process (as
          opposed to driver) stdout and stderr to the local submitting console.
          
        \item[--process\_environment {[environment]} ] This specifies environment parameters for the
          Job Processes. If present, they are added to the Job Process environment as the process is
          spawned. It must be a quoted, blankdelimeted lsit of name-value pairs. For example:
\begin{verbatim}
"--process_environment TERM=xterm DISPLAY=:1.0" 
\end{verbatim}
          
          Note: On Secure Linux systems, the environemnt variable 
          LD\_LIBRARY\_PATH may not be passed to the user's program. If it is 
          necessary to pass LD\_LIBRARY\_PATH to the JP or JD processes, it must be 
          specified as DUCC\_LD\_LIBRARY\_PATH. Ducc (securely) passes this as 
          LD\_LIBRARY\_PATH, after the JP or JD has assumed the user's identity. For 
          example: 
          
\begin{verbatim}
"--process_environment TERM=xterm DISPLAY=:1.0 DUCC\_LD\_LIBRARY\\_PATH=/my/own/
\end{verbatim}

        \item[--process\_executable {[program name]}] This is the full path to a program to be
          executed.

        \item[--process\_executable\_args {[argument list]}] Blank-delimited list of arguments for
          {\em process\_executable}.

        \item[--process\_memory\_size {[size]} ] This specifies the maximum amount of RAM in GB to
          be allocated to each ]rocess.  This value is used by the Resource Manager to allocate
          resources. if this amount is exceeded by a process the Agent terminates the process with a
          ShareSizeExceeded message.

        \item[--scheduling\_class {[classname]} ] This specifies the name of the scheuling class the
          RM will use to determine the resource allocation for each process. The names of the
          classes are installation dependent. If not specified, the default is taken from the global
          DUCC configuration ducc.properties.

        \item[--specifiecaiton {[file]} ] All the parameters used to submit a process may be placed
          in a standard Java properties file.  This file may then be used to submit the process
          (rather than providing all the parameters directory to submit).
          
          For example, 
\begin{verbatim}
ducc_process_submit --specification job.props 
\end{verbatim}

          where the job.props contains: 
\begin{verbatim}
working\_directory=/Users/challngr/projects/ducc/ducc\_test/test/bin 
process\_environment=AE\_INIT\_TIME=10000 DUCC\_LD\_LIBRARY\_PATH=/a/bogus/path 
log\_directory=/Users/challngr/ducc/logs/ 
description=../simple/jobs/1.job[AE] 
scheduling\_class=normal 
process\_memory\_size=15 
\end{verbatim}

        \item[--wait\_for\_completion ] If specified, the submit command does not return control to
          the console immediately, and instead monitors the DUCC state traffic and prints
          information about the process as it progresses.
          
        \item[--working\_directory ] This specifies the working directory to be set by the Job
          Driver and Job Process processes.  If not specified, the current directory is used.

     \end{description}
        
    \paragraph{Notes:}

