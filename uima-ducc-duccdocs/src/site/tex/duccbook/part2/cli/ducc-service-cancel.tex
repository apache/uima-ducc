% Create well-known link to this spot for HTML version
\ifpdf
\else
\HCode{<a name='DUCC_CLI_SERVICE_CANCEL'></a>}
\fi
    \section{ducc\_service\_cancel}
    \label{sec:cli.service-cancel}
    \paragraph{Description:}

    The ducc\_service\_cancel CLI is used to terminate a service instance.

    Note: This command will for for {\em registered} services, but if the service is registered for
    {\em autostart}, the Service Manager will restart the process.  This can be used to advantage, to
    recycle specific registered services instances when needed.

    \paragraph{Usage:}
    \begin{description}
    \item[Executable Jar] java -jar \ducchome/lib/uima-ducc-service-cancel.jar {\em options}
    \item[Script wrapper] \ducchome/bin/ducc\_service\_cancel {\em options}
    \item[Java Main]      java -cp \ducchome/lib/uima-ducc-service-cancel.jar org.apache.uima.ducc.cli.DuccServiceCancel {\em options}
    \end{description}

    \paragraph{Options:}
    \begin{description}
        \item[--debug ]          
          Prints internal debugging information, intended for DUCC developers or extended problem determination.          
        \item[--id {[jobid]}]
          The ID is the id of the service instance to cancel.
        \item[--help]
          Prints the usage text to the console. 
        \item[--role\_administrator] The command is being issued in the role of a DUCC administrator.
          If the user is not also a registered administrator this flag is ignored.  (This helps to
          protect administrators from inadvertently canceling work they do not own.)
     \end{description}
        
    \paragraph{Notes:}
    None.

