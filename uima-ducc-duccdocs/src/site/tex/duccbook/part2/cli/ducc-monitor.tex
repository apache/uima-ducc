% Create well-known link to this spot for HTML version
\ifpdf
\else
\HCode{<a name='DUCC_CLI_MONITOR'></a>}
\fi
    \section{ducc\_monitor}

    \paragraph{Description:}
    
    It may be desired to monitor a job's progress after it has been submitted. The monitor CLI 
    connects to the DUCC message flow and provides job status as it progresses including state 
    changes, error counts, and number of work items processed. 
    
    \paragraph{Usage:}
    \begin{description}
    \item[Executable Jar] java -jar \ducchome/lib/uima-ducc-monitor.jar {\em options}
    \item[Script wrapper] java -jar \ducchome/bin/ducc-monitor {\em options}
    \item[Java Main]      java -cp \ducchome/lib/uima-ducc-monitor.jar org.apache.uima.ducc.cli.DuccJobMonitor {\em options}
    \end{description}

    \paragraph{Options:}
    \begin{description}
        \item[--debug ]          
          Prints internal debugging information, intended for DUCC developers or extended problem determination.
        \item[--id {[jobid]}]
          The ID is the id of the job to monitor.
        \item[--help]
          Prints the usage text to the console. 
        \item[--quiet] 
          Disable CLI informational miessages;
        \item[timestamp]
          Enables timestamps on the monitor messages.
     \end{description}
        
    \paragraph{Notes:}
