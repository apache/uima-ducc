    \section{ducc\_services}
    \label{sec:cli.ducc-services}

    \paragraph{Description:}

        The ducc\_services CLI is used to manage service registration. It has a number of functions 
        as listed below. Additionally the ducc\_services CLI wraps ducc\_service\_submit and 
        ducc\_service\_cancel for convenience. 
        
        The functions include: 
        \begin{description}
            \item[Register] This registers a service with the Service Manager. A registered service is
              retained by DUCC until it is unregistered.
              
            \item[Unregister] This unregisters a service with the Service Manager. When a service is
              unregistered DUCC optionally stops the service instance, if any, and discards all knowledge of
              it.
              
            \item[Start] The start function instructs DUCC to alllocate resources for a service and to
              start it in those resources. The service remains running until explictly stopped. DUCC
              will attempt to keep the service instances running if they should fail. The start function
              is also used to increase the number of running service instances if desired.
              
            \item[Stop] The stop function stops some or all service instances.
              
            \item[Query] The query function returns detailed information about all known services, both
              registerd and otherwise.
              
            \item[Modify] The modify function allows some aspectes of a registered service to be updated
              without reregistereing the service. It optionally alters the running service instances to
              conform with the updates.    
        \end{description}
            

    \paragraph{Usage:}
       \begin{description}
          \item[Executable Jar] java -jar \ducchome/lib/uima-ducc-services.jar {\em options}
          \item[Script wrapper] java -jar \ducchome/bin/ducc-services {\em options}
          \item[Java Main]      java -cp \ducchome/lib/uima-ducc-services.jar org.apache.uima.ducc.cli.DuccServiceApi {\em options}
          \end{description}
          
          The ducc\_services CLI requires one of the verbs listed above as the first argument. Other arguments are determined
          by the verb.

    \paragraph{Options:}

    \subsection{Common Options}
        These options are common to all of the service verbs:
        \begin{description}
           \item[--debug ]          
             Prints internal debugging information, intended for DUCC developers or extended problem determination.                    
           \item[--id {[jobid]}]
             The ID is the id of the job to cancel.
           \item[--help]
             Prints the usage text to the console. 
        \end{description}
        
    \subsection{ducc\_services --register Options}

       The {\em register} function submits a service description to DUCC.  DUCC stores this 
       information until it is {\em unregistered}.  Once registered, a service may be
       started, stopped, etc.

       \begin{description}
           \item[--register] {[specification file] [options]}]

             The properties file is optional. It is a standard Java properties files containing all the 
             registration options for the service. The override options are then applied to define the 
             service (taking precedence). It is possible to register a service using just a properties file, 
             just override options, or both. 
             
             The properties in the properties file are identical to the command-line parameters, but with 
             the leading "--" removed. In the example below, the properties file {\em service.reg} is used
             to register a service, with the number of instances specified as ``2''.
\begin{verbatim}
ducc_services --register service.reg --instances 2
\end{verbatim} 

\begin{verbatim}
process_environment = DUCC_LD_LIBRARY_PATH=/my/own/lib.so 
description = Test Service 0 
process_jvm_args = -Xmx100M -DdefaultBrokerURL=tcp://bluej291:61617 
process_classpath = ../../lib/ducc-test.jar 
process_memory_size = 15 
working_directory = /home/bob/service-descriptors 
process_DD = Service_FixedSleep_0.xml 
scheduling_class = fixed 
\end{verbatim}
         
           \item[--description {[text]}] The text is any string used to describe the job. It is
             displayed in the Web Server. 

           \item[--instances {[number-of-instances]}] This defines the default number of service
             instances to start. If not specified, the default is
             
           \item[--jvm {[path-to-java]}] States the JVM to use. If not specified, the same JVM used by
             the Agents is used. 

           \item[--log\_directory {[path-to-log directory]}] This specifies the path to the directory
             for the user logs. If not specified, the default is the user's home directory.

             Within this directory DUCC creates a subdirectory for each job, using the numerical 
             ID of the job. The format of the generated log file names is descripbed in JRC-NEED-REFERENCE.

           \item[--process\_classpath {[ClASSPATH]}] This specifies the Java CLASSPATH to use in each
             Job Process (JP) and must be specified.

           \item[--process\_DD {[DD descriptor]}] This specifies a UIMA Deployment Descriptor (DD) for
             the service.

           \item[--process\_environment {[environment]}] This specifies environment parameters for the
             Job Processes. If present, they are added to the Job Process environment as the process is
             spawned. It must be a quoted, blankdelimeted lsit of name-value pairs. For example:
\begin{verbatim}
"--process_environment TERM=xterm DISPLAY=:1.0" 
\end{verbatim}
             
           \item[--process\_failures\_limit {[integer]}] This specifies the maximum number of individual
             Job Process (JP) failures that are to be tolerated before killing the job. The default is
             15. If this limit is exceeded over the lifetime of a job DUCC terminates the entire job.

           \item[--process\_jvm\_args {[list]}] This specifies additinal arguments to be passed to the
             Job Process JVM. Example:

\begin{verbatim}
--process_jvm_args -Xmx400M -Xms100M 
\end{verbatim}
             
           \item[--process\_memory\_size {[size]}] This specifies the maximum amount of RAM in GB to be
             allocated to each Job Process.  This value is used by the Resource Manager to allocate
             resources. if this amount is exceeded by a Job Process the Agent terminates the process
             with a ShareSizeExceeded message.

           \item[--scheduling\_class {[classname]}] This specifies the name of the scheuling class the
             RM will use to determine the resource allocation for each process. The names of the classes
             are installation dependent.

             Note: Note that in general one should select a non-preemptable class such as 
             fixed> or reserve for services. Otherwise DUCC may grow or shrink the number 
             of processes used by the service. It IS legal and supported to use a fair-share class 
             however. 

           \item[--service\_ping\_classpath {[classpath]}] This specifies the classpath to be used when
             starting a user-supplied pinger.  See JRC-NEED-REF {\em --service\_ping\_class}.

           \item[--service\_request\_endpoint {[CUSTOM:string]}] This provides the name of the endpoint
             to used for non-UIMA-AS services. The endpoint must start with the characters "CUSTOM:"
             followed by any unique string (with no embedded blanks).
             service. Example:
\begin{verbatim}
--service_custom_endpoint CUSTOM:jrc.service.endpoint 
\end{verbatim}
             
           \item[--service\_ping\_jvm\_args {[list]}] This supplies extra arguments to the JVM for the
             CUSTOM ping object. Example:
\begin{verbatim}
--service_custom_jvm_args -Xmx 400M -Xms100M 
\end{verbatim}
             
           \item[--service\_ping\_class {[java class]}] This supplies the java class name for a CUSTOM
             ping object. The class must the interface org.apache.uima.ducc.IServiceMeta as described in
             the API section. DUCC wraps the customer ping object in a management object with a "main"
             and calls the implemented interfaces periodically to insure the custom service is
             functioning, and to gather performance statistics. 
             
           \item[{--service\_dependency[list]}]
             This specifies a             
             comma-delimeted list of services the job processes are dependent upon (``independent''
             services), if any.  For each independent service, the endpoint must be of the form
             UIMA-AS:queuename:broker-url or CUSTOM:string.
             
             In the example are two dependencies, a UIMA-AS service with endpoint RandomSleepAE and broker
             tcp:BobsGreatQueue:host1:61616, and the other with CUSTOM endpoint
             OtherEp:OtherHost:OtherPort.
             
\begin{verbatim}
--service_dependency UIMA-AS:BobsGreatQueue:host1:61616 CUSTOM:OtherEp:OtherHost:OtherPort
\end{verbatim}
             
           \item[--service\_linger {[time in seconds]}] This specifies the time, in seconds, that a
             service should be kept alive after its last reference has exited, in anticipation of new work
             entering the system and using it. This is only applicable to services that are not
             automatically started at boot time. 
             
           \item[--working\_directory {[directory]}] This specifies the working directory to be set by the
             Job Driver and Job Process processes.  If not specified, the current directory is
             used.
       \end{description}


    \subsection{ducc\_services --start Options}

    The start function instructs DUCC to alllocate resources for a service and to start it in those
    resources. The service remains running until explictly stopped. DUCC will attempt to keep the
    service instances running if they should fail. The start function is also used to increase the
    number of running service instances if desired.
    
       \begin{description}
       \item[--start {[service-id | endpoint]}] This indicates that a service is to be started. The service id
         is either the numeric ID assigned by DUCC when the service is registered, or the service
         endpoint string.  Example:
\begin{verbatim}
ducc_services --start 23 
ducc_services --start UIMA-AS:Service23:tcp://bob.com:12345 
\end{verbatim}
         
       \item[--instances {[integer]}] This is the number of instances to start. If omitted, the
         registered number of instances is started. If the number is specified, the number is added
         to the currently number of running instances. Thus if five instances are running and
         ducc\_services --start 33 --instances 5 is issued, five more service instances ar started
         for service 33 for a totoal of ten. The registry is updated only if the --update option is
         also specified.
\begin{verbatim}
Example: 

ducc_services --start 23 --intances 5 

ducc_services --start UIMA-AS:Service23:tcp://bob.com:12345 \ 
--instances 3 --update 
\end{verbatim}
       \item[--update]If specified, the registry is updated to the total number of started
         instances.  Example:
\begin{verbatim}
ducc_services --start UIMA-AS:Service23:tcp://bob.com:12345 \ 
--instances 3 --update 
\end{verbatim}
       \end{description}

    \subsection{ducc\_services --stop Options}
    The stop function instructs DUCC to stop some number of service instances. If no specific number
    is specified, all instances are stopped. This is used only for registered services. Use
    ducc\_service\_cancel [34] to stop submitted services.

    \begin{description}

  \item[--stop {[service-id | endpoint]}] This specifies the service to be stopped. The service id
         is either the numeric ID assigned by DUCC when the service is registered, or the service
         endpoint string. Example:
\begin{verbatim}
ducc_services --stop 23 
ducc_services --stop UIMA-AS:Service23:tcp://bob.com:12345 
\end{verbatim}
         
       \item[--instances {[integer]}] This is the number of instances to stop. If omitted, all
         instances for the service are stopped.  If a number is specified, then only the specified
         number of instances are stopped. Thus if ten instances are running and ducc\_services --stop
         33 --instances 5 is issued, five (randomly selected) service instances are stopped for
         service 33, leaving five running.  The registry is updated only if the --update option is
         specified. The registered number of instances is never reduced to 0.

         Example: 
\begin{verbatim}
ducc_services --stop 23 --intances 5 
ducc_services --stop UIMA-AS:Service23:tcp://bob.com:12345 --instances 3  
\end{verbatim}

       \item[--update] If specified, the registry is updated to the total number of instances
         remaining, but is never reduced below 1.

         Example: 
\begin{verbatim}
ducc_services --stop UIMA-AS:Service23:tcp://bob.com:12345 --instances 3 --update
\end{verbatim}

    \end{description}

    \subsection{ducc\_services --modify Options}
    The modify function dynamically updates some of the attributes of a registered service. 
    
    \begin{description}
        \item[--modify {[service-id | endpoint]}]  This identifies the service to modify. The service id is either
          the numeric ID assigned by DUCC when the service is registered, or the service endpoint
          string.  Example:
\begin{verbatim}
ducc_services --modify 23 --instances 3 
ducc_services --modify UIMA-AS:Service23:tcp://bob.com:12345 --intances 2 
\end{verbatim}

        \item[ --instances {[integer]}] This updates the number of services instances that are
          started when the service is started.  Only the registration is updated. If the --activate
          option is also specified, running instances are stopped or started as needed to match the
          new number.

          Example: 
\begin{verbatim}
ducc_services --modify 23 --intances 5 
ducc_services --modify UIMA-AS:Service23:tcp://bob.com:12345 --instances 3 --activate 
\end{verbatim}

        \item[ --activate {[integer]}] When specified, the number of running service instances is
          increased or decreased to match the newly specified number.

          Example: 
\begin{verbatim}
ducc_services --modify 23 --intances 5 
ducc_services --modify UIMA-AS:Service23:tcp://bob.com:12345 --instances 3 --activate 
\end{verbatim}

        \item[ --autostart {["true" or "false"]}] This changes the autostart property for the
          registered services. When set to "true", the service is started automatically when the
          DUCC system is started.

          Example: 
\begin{verbatim}
ducc_services --stop UIMA-AS:Service23:tcp://bob.com:12345 --autostart false 
\end{verbatim}
        \end{description}

    \subsection{ducc\_services -query Options}
    The query function returns details about all known services of all types and classes, including 
    the DUCC ids of the service instances (for submitted and registered services), the DUCC ids of 
    the jobs using each service, and a summary of each service's queue and performance statistics, 
    when available. 
    

    \begin{description}
    \item[--query {[service-id | endpoint]}] This indicates that a service is to be stopped. The
      service id is either the numeric ID assigned by DUCC when the service is registered, or the
      service endpoint string.

      If no id is given, information about all services is returned. 

      Below is a query against a system with three services. 

      The service with endpoint UIMA-AS:FixedSleepAE\_6:tcp://bobmach291:61617 is a service 
      submitted outside of DUCC so it is marked as Internal and has no implementing processes 
      tha are known to DUCC. It is used by job 0 and is active, available, and being actively 
      pinged. The ActiveMq queue statistics are shown. 

      The service with endpoint UIMA-AS:FixedSleepAE\_5:tcp://bobmach:61617 is a 
      registered service, whose registered numeric id is 2. It is registered for two instnaces and 
      no autostart. Since it is not autostarted, it will be terminated when it is no longer used. It 
      will linger for 5 seconds after the last referencing job completes, in case a subsequent job 
      that uses it enters the system (not a realistic linger time!). It is currently used (referenced) 
      by DUCC jobs 1 and 5. 

      The service with endpoint UIMA-AS:FixedSleepAE\_1:tcp://bobmach:61617 is a 
      submitted service. It was submitted twice, and so has two implementors, DUCC service 
      jobs 0 and 1. It is referenced by job 7. It will continue to run until somebody cancels it, 
      even if it is not used. 

\begin{verbatim}
Service: UIMA-AS:FixedSleepAE_6:tcp://bobmach291:61617 
Service Class : Implicit 
Implementors : (N/A) 
References : 0 
Dependencies : none 
Service State : Available 
Ping Active : true 
Autostart : false 
Manual Stop : false 
Queue Statistics: 
Consum Prod Qsize minNQ maxNQ expCnt inFlgt DQ NQ Disp 
78 240 170 2 36414 0 0 636 806 636 

Service: UIMA-AS:FixedSleepAE_5:tcp://bobmach291:61617 
Service Class : Registered as ID 2 instances[2] linger[5] 
Implementors : 9 8 
References : 1 5 
Dependencies : none 
Service State : Available 
Ping Active : true 
Autostart : false 
Manual Stop : false 
Queue Statistics: 
Consum Prod Qsize minNQ maxNQ expCnt inFlgt DQ NQ Disp 
52 44 0 0 3 0 0 402 402 402 

Service: UIMA-AS:FixedSleepAE_1:tcp://bobmach291:61617 
Service Class : Submitted 
Implementors : 1 0 
References : 7 
Dependencies : none 
Service State : Available 
Ping Active : true 
Autostart : false 
Manual Stop : false 
Queue Statistics: 
Consum Prod Qsize minNQ maxNQ expCnt inFlgt DQ NQ Disp 
52 0 0 1 1504371 0 0 35 35 35 
\end{verbatim}
    \end{description}
    \paragraph{Notes:}

