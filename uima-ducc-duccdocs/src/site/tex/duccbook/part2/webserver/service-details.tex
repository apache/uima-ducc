\section{Service Details Page}
\label{sec:ws-service-details}

This page shows details of the processes which implement.  Note that in the case of
\hyperref[sec:services.implicit]{implicit} and \hyperref[subsub:services.ping-only]{ping-only}
services there will be no processes to show.

The information is divided between two tabs:

   \begin{description}
       \item[Processes] This tab contains details on all the proceses implementing
         the service, if any.
       \item[Specification] This tab shows the specification for the service.  In the
         case of \hyperref[sec:services.implicit]{implicit} services, this shows the generated Service Manager
         state for the service.
   \end{description}  

   \subsection{Processes}
   \label{sec:ws-services-processes}

   The processes page contains the following columns:
   \begin{description}
      \item[ID] \hfill \\
        This is the DUCC-assigned numeric id of the process.  This format of this
        id is two numbers:
\begin{verbatim}
    RESID.SHAREID
\end{verbatim}
        Here, the {\em RESID} is the reservation ID.  The {\em SHAREID} is the 
        share ID assigned by the Resource Manager.  Together these form a unique
        ID for each process that runs in the reservation.
                
      \item[Log] \hfill \\
        This is the log name for the process. It is hyperlinked to the log itself.
        
      \item[Size] \hfill \\
        This is the size of the log in MB. If you find you have trouble viewing the log
        from the web server it could be because it is too big to view in the browser.
        
      \item[Hostname] \hfill \\
        This is the name of the node where the process is running (or ran).
        
      \item[PID] \hfill \\
        This is the Unix process ID (PID) of the process.
        
      \item[State Scheduler] \hfill \\
        This shows the Resesource Manager state of the job. It is one of:
        
        \begin{description}
            \item[Allocated] - The node is still allocated for this job by the RM.
            \item[Deallocated] - The resource manager has deallocated the shares for the job on
              this node.
        \end{description}
        
      \item[Reason Scheduler or Extraordinary Status] \hfill \\
        These are the same as for the \hyperref[itm:job-details-sched]{job details.}

      \item[State Agent] \hfill \\
        These are the same as for the \hyperref[itm:job-details-state]{job details.}

      \item[Reason Agent] \hfill \\
        These are the same as for the \hyperref[itm:job-details-agent]{job details.}


      \item[Time Init] \hfill \\
        Most services are UIMA-AS services and therefore have an {\em initialization} phase
        to their lifetimes.  This field shows the time spent in that phase.

      \item[Time Run] \hfill \\
        The current duration of the reservation, or total duration if it has 
        terminated.
        
      \item[Time GC] \hfill \\
        This is amount of time spent in Java Garbage Collection for the process.

      \item[Pgin] \hfill \\
        This is the number of page-in events on behalf of the process.
        
      \item[Swap] \hfill \\
        This is the amount of swap space on the machine being consumed by the process.
        
      \item[\%CPU] \hfill \\
        Currnt CPU percent consumed by the process.  This will be $>$ 100\% on 
        multi-core systems if more than one core is being used.  Each core contributes
        up to 100\% CPU, so, for example, on a 16-core machine, this can be as high
        as 1600\%.

      \item[RSS] \hfill \\
        The amount of real memory being consumed by the process (Resident Storage Size)

      \item[JConsole URL] \hfill \\
        This is a URL that can be used to connect via JMX to the processes, e.g. via
        jconsole.

   \end{description}

   \subsection{Specification}
   \label{sec:ws-managed-reservation-specification}
   This tab shows the full job specification in the form of a Java Properties
   file.  This will include all the parameters specified by the user, plus those
   filled in by DUCC.
        
   The specification for a Service contains two types of entries:
   \begin{enumerate}
     \item Service specification properties, prefixed with ``svc''. These comprise
       the service specification that the Service Manager submits on behalf of
       a user in order to start registered services.
     \item Meta properties, prefixed with ``meta''.  This is the Service Manager's state
       record for the sesrvice as it is running.  In addition to state it contains
       properties required for service registration that are not used for
       service submission.
   \end{enumerate}
   
