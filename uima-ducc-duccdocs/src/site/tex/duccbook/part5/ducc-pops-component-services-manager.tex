    
    \subsection{\varServicesManager~(\varSM)}    
    
    There is one \varServicesManager~per \varDUCC~cluster.
    
    The duties of the \varServicesManager~are:
    \textit{
      \input{part5/c01-SM.tex}
    }
        
    The \varServicesManager~provides additional functionality for operation of the
    \varDUCC~system.
    It is configurable and tunable.
    
    Although not essential for the main purpose of the \varDUCC~system, in
    practical terms for large systems the \varServicesManager~is highly 
    desirable for improved resource utilization.
    By using shared services, resources are more effectively employed and
    work items processing is completed sooner.
    
    Some dimensions:
    
    \begin{itemize}
    
      \item long warm-up time
      
      When the service takes a long time to warm-up, the \varJob~in progress
      may have to sit idle for a long time before the first work item can be
      processed until the service upon which it depends has initialized and 
      is ready.
      If the service is already up and ready, this delay can be avoided
      and the \varJob~experiences no service delay.
      
      \item large storage use
      
      When the service has a large memory footprint, it can be far more
      efficient to have multiple \varJobs~share the service rather than
      having separate copies for each.
      
      \item short processing time
      
      Related to large storage use, if the time to process a work item is
      relatively short, again it can be much more efficient to share the
      service amongst multiple \varJobs.
      
      \item not used
      
      If a service has not been used for a relatively long time, it may be 
      better to shut it down and reclaim the resources for use elsewhere, 
      especially on a busy cluster.
            
    \end{itemize}
    
    \subsubsection{Ping Driver} 
    
    This runs the watchdog thread for custom service pingers.
    It ascertains the liveness and healthiness of each service
    known to \varDUCC.
    
    \subsubsection{Service Handler} 
    
    Carries out Service Manager validated request operations.
            
    \subsubsection{Service Manager, API Handler} 
    
    Receives and validates service request operations:
    
    \begin{itemize}
      \item register
      \item unregister
      \item start
      \item stop
      \item query
      \item modify
    \end{itemize}
    
    The \varServicesManager~maintains a registry of services.
    The attributes of these services may include one of more of the following:
    
    \begin{description}
      \item \texttt{permanent}
      A permanent service is one that is kept up so long as the
      \varDUCC~system is running.
      \item \texttt{on-demand}
      An on-demand service is one that is kept up only during the
      lifetime of one or more \varJobs~that declare a dependency on the 
      service
      \item \texttt{lingering}
      A lingering service is one that continues for some limited time
      beyond the lifetime of the last dependent \varJob~in anticipation
      of another \varJob~arrival in the near future.
      \item \texttt{dynamic}
      A dynamic service is one that automatically expands and contracts
      in terms of number of instances to meet demand.
      \item \texttt{registered}
      A registered service is one that is pre-defined, whose definition
      is kept persistently and whose lifecycle is managed by \varDUCC.
      \item \texttt{custom}
      A custom (unregistered) service is one that is not pre-defined, whose definition
      is not kept persistently and whose lifecycle is not managed by \varDUCC.
    \end{description}
  
    The \varServicesManager~keeps within its registry information of
    two types: \texttt{service} and \texttt{meta}.
    
    Type \texttt{service} information includes the following attributes:
    \begin{itemize}
      \item classpath
      \item description
      \item environment
      \item jvm
      \item jvm args
      \item log directory
      \item deployment descriptor
      \item failures limit
      \item memory size
      \item scheduling class
      \item linger time
      \item pinger classpath
      \item pinger log
      \item pinger timeout
      \item service endpoint
      \item working directory
    \end{itemize}
    
    Type \texttt{meta} information includes the following attributes:
    \begin{itemize}
      \item autostart
      \item endpoint
      \item implementors
      \item instances (count)
      \item identifier (number)
      \item ping-active
      \item ping-only
      \item service-active
      \item service-class
      \item service-health
      \item service-state
      \item service-statistics
      \item service-type
      \item stopped
      \item user
      \item uuid
      \item work-instances (PID list)
    \end{itemize}
    