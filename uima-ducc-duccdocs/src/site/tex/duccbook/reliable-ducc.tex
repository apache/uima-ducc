% 
% Licensed to the Apache Software Foundation (ASF) under one
% or more contributor license agreements.  See the NOTICE file
% distributed with this work for additional information
% regarding copyright ownership.  The ASF licenses this file
% to you under the Apache License, Version 2.0 (the
% "License"); you may not use this file except in compliance
% with the License.  You may obtain a copy of the License at
% 
%   http://www.apache.org/licenses/LICENSE-2.0
% 
% Unless required by applicable law or agreed to in writing,
% software distributed under the License is distributed on an
% "AS IS" BASIS, WITHOUT WARRANTIES OR CONDITIONS OF ANY
% KIND, either express or implied.  See the License for the
% specific language governing permissions and limitations
% under the License.
% 
\documentclass[letterpaper]{article}

\def\DUCCSTANDALONE{}

% space between paragraphs
\usepackage{parskip}

\usepackage{hyperref}
% Margins
\usepackage[top=1in, bottom=.75in, left=.75in, right=.75in ]{geometry}

\usepackage{xcolor}
\usepackage{datetime}

% turn off section numbering
\setcounter{secnumdepth}{0}

\title{Reliable DUCC - Design}
\author{Written and maintained by the Apache \\
UIMA\texttrademark Development Community}
\date{}
\begin{document}
\maketitle

%Licensed to the Apache Software Foundation (ASF) under one
%or more contributor license agreements.  See the NOTICE file
%distributed with this work for additional information
%regarding copyright ownership.  The ASF licenses this file
%to you under the Apache License, Version 2.0 (the
%"License"); you may not use this file except in compliance
%with the License.  You may obtain a copy of the License at
%
%   http://www.apache.org/licenses/LICENSE-2.0
%
%Unless required by applicable law or agreed to in writing,
%software distributed under the License is distributed on an
%"AS IS" BASIS, WITHOUT WARRANTIES OR CONDITIONS OF ANY
%KIND, either express or implied.  See the License for the
%specific language governing permissions and limitations
%under the License.
Copyright \copyright~ 2012 The Apache Software Foundation

Copyright \copyright~ 2012 International Business Machines Corporation

{\addtolength{\leftskip}{10 mm}
     \paragraph{License and Disclaimer}
     The ASF licenses this documentation to you under the Apache License, Version 2.0 (the "License");
     you may not use this documentation except in compliance with the License.  You may obtain a copy of
     the License at
              
     \url{http://www.apache.org/licenses/LICENSE-2.0}
     
     Unless required by applicable law or agreed to in writing, this documentation and its contents are
     distributed under the License on an "AS IS" BASIS, WITHOUT WARRANTIES OR CONDITIONS OF ANY KIND,
     either express or implied.  See the License for the specific language governing permissions and
     limitations under the License.

     \paragraph{Trademarks}     
     All terms mentioned in the text that are known to be trademarks or service marks have been
     appropriately capitalized.  Use of such terms in this book should not be regarded as affecting the
     validity of the the trademark or service mark.

}

\vspace{.5in}

\newdateformat{mydate}{%
\monthname~\THEYEAR}

Publication date: \mydate\today

\newpage

% 
% Licensed to the Apache Software Foundation (ASF) under one
% or more contributor license agreements.  See the NOTICE file
% distributed with this work for additional information
% regarding copyright ownership.  The ASF licenses this file
% to you under the Apache License, Version 2.0 (the
% "License"); you may not use this file except in compliance
% with the License.  You may obtain a copy of the License at
% 
%   http://www.apache.org/licenses/LICENSE-2.0
% 
% Unless required by applicable law or agreed to in writing,
% software distributed under the License is distributed on an
% "AS IS" BASIS, WITHOUT WARRANTIES OR CONDITIONS OF ANY
% KIND, either express or implied.  See the License for the
% specific language governing permissions and limitations
% under the License.
% 
% common macros in a single place
% These are used in the main file, and in the stand-alone wrappers
\newcommand{\duccruntime}{\$DUCC\_HOME}
\newcommand{\ducchome}{\$DUCC\_HOME}
\newcommand{\todo}{{\sc \Large TODO:}  }
\newcommand{\note}{{\em Note:}  }
\newcommand{\cfbox}[2]{%
    \colorlet{currentcolor}{.}%
    {\color{#1}%
    \fbox{\color{currentcolor}#2}}%
}
\newcommand{\DUCC}{\em DUCC}
\newcommand{\WebServer}{WebServer}

\section{Multiple DUCC head nodes}

This first major section describes support for multiple active DUCC head nodes.

\subsection{Introduction}
    DUCC can be configured to run reliably by having multiple head nodes,
    comprising one {\em master} and one or more {\em backup} head nodes.
    DUCC exploits Linux {\em keepalived} virtual IP addressing to enable
    this capability.
    
    The advantages are that if the {\em master} DUCC host becomes
    unusable, the {\em backup} DUCC can take over seamlessly
    such that active distributed Jobs, Reservations, Managed Reservations 
    and Services continue uninterrupted.  Take over also facilitates
    continued acceptance of new submissions and monitoring of new and
    existing submissions without interruption.

\subsection{Daemons}
   Each head node, whether {\em master} or {\em backup}, runs a Broker,
   Orchestrator, PM, RM, and SM.
   
   The Cassandra database is expected to be located on a node(s) separate from the head nodes.
   
   Likewise, the JD node(s) is separate from the head nodes.
   
   The Agents are distributed, as before.
        
\subsection{Configuring Host Machines}    
    See {\em Configuring Simple Virtual IP Address Failover Using Keepalived} 
    which can be found at this web address: 
    \url{https://docs.oracle.com/cd/E37670_01/E41138/html/section_uxg_lzh_nr.html}.

	Linux Commands
	
	Starting keepalived
	
    \begin{verbatim}
    > sudo service keepalived start
    Starting keepalived:                                       [  OK  ]
   	\end{verbatim}
   	
   	Querying keepalived
	
    \begin{verbatim}
    > ip addr show dev eth0
    2: eth0: <BROADCAST,MULTICAST,UP,LOWER_UP> mtu 1500 qdisc mq state UP qlen 1000
    link/ether 00:21:5e:20:02:84 brd ff:ff:ff:ff:ff:ff
    inet 192.168.3.7/16 brd 192.168.255.255 scope global eth0
    inet 192.168.200.17/32 scope global eth0
    inet6 fe80::221:5eff:fe20:284/64 scope link 
       valid_lft forever preferred_lft forever
   	\end{verbatim}

	Stopping keepalived
	
    \begin{verbatim}
    > sudo service keepalived stop
    Stopping keepalived: 
   	\end{verbatim}

\subsection{Configuring DUCC}  
    To configure DUCC to run reliable, there are two properties that must
    be configured in the {\em site.ducc.properties} file.  Example:
    
    \begin{verbatim}
	ducc.virtual.ip.device = eth0
	ducc.virtual.ip.address = 192.168.200.17
   	\end{verbatim}
    
    Use the device and virtual IP address configured for your host machines. 
    
\subsection{Webserver}

	Webserver for Master

	The {\em master} DUCC Webserver will display all pages normally with additional
	information in the heading upper left:
	
	reliable: \textbf{master}
	
	Webserver for Backup
	
	The {\em backup} DUCC Webserver will display some pages normally with additional
	information in the heading upper left:
	
	\underline{reliable}: \textbf{backup}
   	
   	Hovering over \underline{reliable} will yield the following information:
   	{\em Click to visit master}
   	
   	Several pages will display the following information (or similar):
   	
   	\begin{verbatim}
	no data - not master
   	\end{verbatim}
	
\subsection{Code changes}

The key changes include a new script (see ducc\_head\_mode.py) to 
interact with Linux to determine virtual IP address status and 
corresponding Java code (see common.head.ADuccHead.java)
that interprets the status to make transitions between 
{\em master} and {\em backup} states.

\subsubsection{scripts and configuration files}

\textbf{ducc\_head\_mode.py}

This is a new script employed at runtime by the various daemons to
determine the current mode of operation.  Status is determined 
though invocation of this script upon receipt of each Orchestrator
publication.

   \begin{verbatim}
    # purpose:    determine reliable ducc status
    # input:      none
    # output:     one of { unspecified, master, backup }
    # operation:  look in ducc.properties for relevant keywords
    #             and employ linux commands to determine if system
    #             has matching configured virtual ip address
   \end{verbatim}
    
\textbf{ducc.properties}  

   \begin{verbatim}
    # These optional properties declare virtual ip device and address for "reliable" DUCC.
    # Reliable DUCC comprises a MASTER and BACKUP configured using Linux keepalived, which
    # must correspond with the below two DUCC properties.
    # The following signify "reliable" DUCC is *not* in effect:
    # - omission of one or both of these properties
    # - 0.0.0.0 specified as <ducc.virtual.ip.address>
    # When in effect, a DUCC head node will act as MASTER only when the command
    # > ip addr show dev <ducc.virtual.ip.device> displays <ducc.virtual.ip.address>
    # Automatic takeover by the DUCC head BACKUP from DUCC head MASTER (and vice versa) occurs 
    # when the above command results change.
    ducc.virtual.ip.device = eth0
    ducc.virtual.ip.address = 0.0.0.0
   \end{verbatim}
    
\subsubsection{agent}

{\renewcommand\labelitemi{}
\begin{itemize}
  \item \textbf{DuccWorkHelper} - use virtualIP address or head node
  \item \textbf{AgentEventListener} - ignore any incoming publications from backup producer
  \item \textbf{CGroupsTest} - employ changed DuccIdFactory signature
\end{itemize}
}

\subsubsection{cli}

{\renewcommand\labelitemi{}
\begin{itemize}
  \item \textbf{DuccMonitor} - use virtual IP address, or WS node, or head node
  \item \textbf{DuccUiUtilities} - use virtualIP address or head node (to submit, cancel..)
\end{itemize}
}

\subsubsection{common}

{\renewcommand\labelitemi{}
\begin{itemize}
  \item \textbf{AbstractDuccComponent} - for broker use virtual IP address or head node, remove commented-out code, remove print to console
  \item \textbf{ADuccHead} - abstract class with reliable DUCC share functionality
  \item \textbf{DuccHeadHelper, IDuccHead} - reliable DUCC utilities
  \item \textbf{IDuccHead} - reliable DUCC interface
  \item \textbf{IStateServices} - database access control RW or RO
  \item \textbf{NullStateServices} - database access control RW or RO
  \item \textbf{StateServices} - database access control RW or RO
  \item \textbf{DuccPropertiesHelper} - fetch virtual IP address
  \item \textbf{DuccPropertiesResolver} - Reliable DUCC properties keys
  \item \textbf{IDuccLoggerComponents} - Missing PM abbreviation
  \item \textbf{DuccIdFactory} - improved (generalized) to handle DB persisted sequence numbering
\end{itemize}
}

\subsubsection{database}

{\renewcommand\labelitemi{}
\begin{itemize}
  \item \textbf{IDuccHead} - reliable DUCC interface
  \item \textbf{DbOrchestratorProperties} - support for OR properties table
  \item \textbf{IDbOrchestratorProperties} - interface of OR properties table
  \item \textbf{IOrchestratorProperties} - interface for OR properties
  \item \textbf{IOrchestratorProperties} - database access control RW or RO
\end{itemize}
}

\subsubsection{orchestrator}

{\renewcommand\labelitemi{}
\begin{itemize}
  \item \textbf{DuccHead} - loggable wrapper around common.ADuccHead
  \item \textbf{OrchestratorCommonArea} - add restart capability for transition to {\em master}
  \item \textbf{OrchestratorComponent} - reject requests from CLI and JD and publications for Agents when not {\em master}, employ DB for Job and publication number persistence, use active workMap from common area, tag publication with node identity and producer state {\em master} or {\em backup}, make transitions between {\em master} and {\em backup} states
  \item \textbf{OrchestratorRecovery} - employ changed DuccIdFactory initialization requirements
  \item \textbf{ReservationFactory} - employ changed DuccIdFactory signature
  \item \textbf{StateJobAccounting} - log job state changes
  \item \textbf{StateManager} - use active workMap from common area
  \item \textbf{WorkMapHelper} - adding logging 
  \item \textbf{AOrchestratorCheckpoint} - refactor checkpointing, suspend when {\em backup} resume when {\em master}
  \item \textbf{IOrchestratorCheckpoint} - refactor checkpointing
  \item \textbf{OrchestratorCheckpoint} - refactor checkpointing
  \item \textbf{OrchestratorCheckpointDb} - refactor checkpointing
  \item \textbf{OrchestratorCheckpointFile} - refactor checkpointing
  \item \textbf{OrchestratorConfiguration} - employ changed DuccIdFactory for publication sequence numbering
  \item \textbf{OrDbDuccWorks} - specification to DB only when {\em master}
  \item \textbf{OrDbDuccWorks} - orchestrator properties to DB only when {\em master}
  \item \textbf{OrchestratorEventListener} - record to system events log daemon switches between {\em backup} and {\em master} 
  \item \textbf{ReservationFactory} - employ changed DuccIdFactory for Job numbering\item \textbf{ReservationFactory} - employ changed DuccIdFactory signature
  \item \textbf{JdScheduler} - suspend JD host management when {\em backup} resume when {\em master}
  \item \textbf{HealthMonitor} - use active workMap from common area
  \item \textbf{MaintenanceThread} - use active workMap from common area
  \item \textbf{AOrchestratorState} - refactor orchestrator state managements from files to DB
  \item \textbf{DuccWorkIdFactory} - refactor orchestrator state managements from files to DB
  \item \textbf{IOrchestratorState} - refactor orchestrator state managements from files to DB
  \item \textbf{OrchestratorState} - refactor orchestrator state managements from files to DB
  \item \textbf{OrchestratorStateDb} - refactor orchestrator state managements from files to DB
  \item \textbf{OrchestratorStateDbConversion} - refactor orchestrator state managements from files to DB
  \item \textbf{OrchestratorStateFile} - refactor orchestrator state managements from files to DB
  \item \textbf{AOrchestratorStateJson} - refactor orchestrator state managements from files to DB
  \item \textbf{SystemEventsLogger} - record all CLI interactions in system events log
  \item \textbf{TestSuite} - print whether {\em backup} or {\em master}
\end{itemize}
}

\subsubsection{pm}

{\renewcommand\labelitemi{}
\begin{itemize}
  \item \textbf{DuccHead} - loggable wrapper around common.ADuccHead
  \item \textbf{ProcessManagerComponent} - make transitions between {\em master} and {\em backup} states
\end{itemize}
}

\subsubsection{sm}

{\renewcommand\labelitemi{}
\begin{itemize}
  \item \textbf{DuccHead} - loggable wrapper around common.ADuccHead
  \item \textbf{ServiceHandler} - resume operations when state is {\em master}, quiesce operations when state is {\em backup}
  \item \textbf{ServiceManagerComponent} - make transitions between {\em master} and {\em backup} states, reject requests when in {\em backup} state, employ changed DuccIdFactory signature 
 \textbf{ServiceSet} - handle new state {\em Dispossessed}
\end{itemize}
}

\subsubsection{transport}

{\renewcommand\labelitemi{}
\begin{itemize}
  \item \textbf{JobDriverStateExchanger} - use virtualIP address or head node
  \item \textbf{AbstractDuccEvent} - tag publications with producer host identity and state {\em master} or {\em backup}
  \item \textbf{DaemonDuccEvent} - switch to {\em master} or {\em backup} state for recording to system event log
  \item \textbf{DuccEvent} - add events SWITCH\_TO\_MASTER and SWITCH\_TO\_BACKUP
  \item \textbf{JdEvent} - interrogate publications producer state {\em master} or {\em backup}
  \item \textbf{IService} - add service state Dispossessed, Service is not controlled by this Service Manager
\end{itemize}
}

\subsubsection{webserver}

{\renewcommand\labelitemi{}
\begin{itemize}
  \item \textbf{DuccBoot} - make boot reusable for switch to {\em master}
  \item \textbf{DuccData} - create reset function for switch to {\em master}
  \item \textbf{DuccHead} - loggable wrapper around common.ADuccHead
  \item \textbf{WebServerComponent} - make transitions between {\em master} and {\em backup} states
  \item \textbf{WebServerConfiguration} - make boot reusable for switch to {\em master}
  \item \textbf{DuccHandler} - servlet to produce reliable DUCC state {\em master} or {\em backup}
  \item \textbf{DuccHandlerClassic} - servlets to produce \"no data - not master\" when appropriate
  \item \textbf{DuccHandlerJsonFormat} - servlets to produce \"no data - not master\" when appropriate
  \item \textbf{DuccWebServer} - add method getPort
  \item \textbf{c4-ducc-mon.jsp} - web page header location for reliable DUCC state
  \item \textbf{ducc.js} - web page header updating for reliable DUCC state
\end{itemize}
}

\section{Installing and Cloning}

This second major section describes support for installation of head node master and backup(s).

TBD

\section{Autostart}

This third major section describes support for autostart of head node and agent daemons.

TBD

\section{Monitoring and Switching}

This fourth major section describes support monitoring of multiple head nodes and switching to an alternate when the primary is dysfunctional.

TBD

\end{document}
