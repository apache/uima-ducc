% 
% Licensed to the Apache Software Foundation (ASF) under one
% or more contributor license agreements.  See the NOTICE file
% distributed with this work for additional information
% regarding copyright ownership.  The ASF licenses this file
% to you under the Apache License, Version 2.0 (the
% "License"); you may not use this file except in compliance
% with the License.  You may obtain a copy of the License at
% 
%   http://www.apache.org/licenses/LICENSE-2.0
% 
% Unless required by applicable law or agreed to in writing,
% software distributed under the License is distributed on an
% "AS IS" BASIS, WITHOUT WARRANTIES OR CONDITIONS OF ANY
% KIND, either express or implied.  See the License for the
% specific language governing permissions and limitations
% under the License.
% 
% Create well-known link to this spot for HTML version
\ifpdf
\else
\HCode{<a name='DUCC_RELIABLE'></a>}
\fi
\chapter{Reliable DUCC}
\label{chap:reliable}

\section{Introduction}
    DUCC can be configured to run reliably by having multiple head nodes,
    comprising one {\em master} and one or more {\em backup} head nodes.
    DUCC exploits Linux {\em keepalived} virtual IP addressing to enable
    this capability.
    
    The advantages are that if the {\em master} DUCC host becomes
    unusable, the {\em backup} DUCC can take over seamlessly
    such that active distributed Jobs, Reservations, Managed Reservations 
    and Services continue uninterrupted.  Take over also facilitates
    continued acceptance of new submissions and monitoring of new and
    existing submissions without interruption.
        
\section{Configuring Host Machines}    
    See {\em Configuring Simple Virtual IP Address Failover Using Keepalived} 
    which can be found at this web address: 
    \url{https://docs.oracle.com/cd/E37670_01/E41138/html/section_uxg_lzh_nr.html}.

	Linux Commands
	
	Starting keepalived
	
    \begin{verbatim}
    > sudo service keepalived start
    Starting keepalived:                                       [  OK  ]
   	\end{verbatim}
   	
   	Querying keepalived
	
    \begin{verbatim}
    > ip addr show dev eth0
    2: eth0: <BROADCAST,MULTICAST,UP,LOWER_UP> mtu 1500 qdisc mq state UP qlen 1000
    link/ether 00:21:5e:20:02:84 brd ff:ff:ff:ff:ff:ff
    inet 192.168.3.7/16 brd 192.168.255.255 scope global eth0
    inet 192.168.200.17/32 scope global eth0
    inet6 fe80::221:5eff:fe20:284/64 scope link 
       valid_lft forever preferred_lft forever
   	\end{verbatim}

	Stopping keepalived
	
    \begin{verbatim}
    > sudo service keepalived stop
    Stopping keepalived: 
   	\end{verbatim}

\section{Configuring DUCC}  
    To configure DUCC to run reliable, there are two properties that must
    be configured in the {\em site.ducc.properties} file.  Example:
    
    \begin{verbatim}
	ducc.virtual.ip.device = eth0
	ducc.virtual.ip.address = 192.168.200.17
   	\end{verbatim}
    
    Use the device and virtual IP address configured for your host machines. 
    
\section{Webserver}

	Webserver for Master

	The {\em master} DUCC Webserver will display all pages normally with additional
	information in the heading upper left:
	
	\begin{verbatim}
	reliable: \textbf{master}
   	\end{verbatim}
	
	Webserver for Backup
	
	The {\em backup} DUCC Webserver will display some pages normally with additional
	information in the heading upper left:
	
	\begin{verbatim}
	\underline{reliable}: \textbf{backup}
   	\end{verbatim}
   	
   	Hovering over \underline{reliable} will yield the following information:
   	{\em Click to visit master}
   	
   	Several pages will display the following information (or similar):
   	
   	\begin{verbatim}
	no data - not master
   	\end{verbatim}
	