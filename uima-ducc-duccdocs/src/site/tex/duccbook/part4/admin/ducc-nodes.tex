\section{Ducc Node Definitions}

    The DUCC node definitions are specified by default in the file {\em ducc.nodes}.

    The DUCC node list is used to configure the nodes used to run jobs and assign reservations. A 
    DUCC Agent is started by DUCC on every node in the node list. 

    The node list can be composed of multiple node lists to assist organization of the DUCC cluster. 
    All the administrative commands operate upon node lists. By carefully organized these lists it is 
    possible to administer portions of a cluster independently. 
    
    A node list is a simple flat file where each line consists of a single node name or an import 
    statement. Nodes may be designated by IP address or by name. The node list may be commented 
    using the comment delimeter "\#". 
    
    An import statement is of the form 

\begin{verbatim}
import filename 
\end{verbatim}
    
    where "filename" is the name of another node list. The imported nodelist may itself contain import 
    statements to allow a nested organization of lists. 

    When an import statement is encountered, the named file is read and its contents appended to the 
    stream of incoming nodes. The list of nodes used by commands is composed of the nodes in the 
    first list and all imported files. 

    Examples: 

\begin{verbatim}
cat ducc.nodes 
# First four nodes 
ducc01.local.net 
ducc02.local.net 
ducc03.local.net 
ducc04.local.net 
import big.nodes 
# 64 GB 
# 64 GB 
# 128 GB 
# 128 GB 

cat big.nodes 


# Large memory nodes, all with 256 GB 
ducc11.local.net 
ducc12.local.net 
ducc13.local.net 
ducc14.local.net 
\end{verbatim}
    
