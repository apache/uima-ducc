% 
% Licensed to the Apache Software Foundation (ASF) under one
% or more contributor license agreements.  See the NOTICE file
% distributed with this work for additional information
% regarding copyright ownership.  The ASF licenses this file
% to you under the Apache License, Version 2.0 (the
% "License"); you may not use this file except in compliance
% with the License.  You may obtain a copy of the License at
% 
%   http://www.apache.org/licenses/LICENSE-2.0
% 
% Unless required by applicable law or agreed to in writing,
% software distributed under the License is distributed on an
% "AS IS" BASIS, WITHOUT WARRANTIES OR CONDITIONS OF ANY
% KIND, either express or implied.  See the License for the
% specific language governing permissions and limitations
% under the License.
% 
% Create well-known link to this spot for HTML version
\ifpdf
\else
\HCode{<a name='DUCC_CLI_SUBMIT'></a>}
\fi

    \section{ducc\_submit}
    \label{sec:cli.ducc-submit}
%       The source for this section is ducc\_duccbook/documents/part-user/cli/submit.xml.
       \paragraph{Description:}
           The submit CLI is used to submit work for execution by DUCC. DUCC assigns a unique id to the
           job and schedules it for execution. The submitter may optionally request that the progress of
           the job is monitored, in which case the state of the job as it progresses through its
           lifetime is printed on the console.
       \paragraph{Usage:}
           \begin{description}
             \item[Script wrapper] \ducchome/bin/ducc\_submit {\em options}
             \item[Java Main]      java -cp \ducchome/lib/uima-ducc-cli.jar org.apache.uima.ducc.cli.DuccJobSubmit {\em options}
           \end{description}

        \paragraph{Options:}
           \begin{description}

           \item[$--$all\_in\_one $<$local $|$ remote $>$]
               Run driver and pipeline in single process.  If {\em local} is specified, the
               process is executed on the local machine, for example, in the current Eclipse session.
               If {\em remote} is specified, the jobs is submitted to DUCC as a {\em managed reservation}
               and run on some (presumably larger) machine allocated by DUCC.

           \item[$--$attach\_console] If specified, redirect remote stdout and stderr
             to the local submitting console.

           \item[$--$cancel\_on\_interrupt] If specified, the job is monitored 
             and will be canceled if the submit command is interrupted, e.g. with CTRL-C. 
             This option always implies {\em $--$wait\_for\_completion}.

           \item[$--$classpath {[path-string]}] The CLASSPATH used for the job.  If specified, this is used
             for both the Job Driver and each Job Process. If not specified, the CLASSPATH of the
             process invoking this request is used.

           \item[$--$classpath\_order {[user-before-ducc $|$ ducc-before-user]} ]  
             OBSOLETE - ignored.

           \item[$--$debug] Enable debugging messages. 
             This is primarily for debugging DUCC itself.

           \item[$--$description {[text]}] The text is any string used to describe the job. It is
             displayed in the Web Server. When specified on a command-line the text usually 
             must be surrounded by quotes to protect it from the shell.  The default is ``none''.

           \item[$--$driver\_debug {[debug-port]}] Append JVM debug flags to the JVM arguments
             to start the JobDriver in remote debug mode.  The remote process debugger will attempt
             to contact the specified port.

           \item[$--$driver\_descriptor\_CR {[descriptor.xml]} ] This is the XML descriptor for the
             Collection Reader.  This 
             descriptor is a resource that is searched for in the filesystem or Java classpath as described 
             in the ~\hyperref[par:cli.submit.notes]{notes below}. (Required)

           \item[$--$driver\_descriptor\_CR\_overrides {[list]} ]             
             This is the Job Driver collection reader configuration overrides. They are specified as 
             name/value pairs in a whitespace-delimited list. Example: 
             \begin{verbatim}
     --driver_descriptor_CR_overrides name1=value1 name2=value2...
             \end{verbatim}
             
           \item[$--$driver\_exception\_handler {[classname]}] 
             This specifies a developer-supplied exception handler for the Job Driver.  
             It must implement {\em org.apache.uima.ducc.IErrorHandler} or extend
             {\em org.apache.uima.ducc.ErrorHandler}.  A built-in default exception handler is provided.
           
           \item[$--$driver\_exception\_handler\_arguments {[argument-string]}] This is a string
             containing arguments for the exception handler.  The contents of
             the string is entirely a function of the specified exception handler.  If not specified,
             a {\em null} is passed in.
             \\The built-in default exception handler supports an argument string of the following form
             (with NO embedded blanks):
             \begin{verbatim}
     max_job_errors=15 max_timeout_retrys_per_workitem=0
             \end{verbatim}
             Note: When used as a CLI option, the string must usually be
             quoted to protect it from the shell, if it contains blanks.
             The built-in default exception handler supports two arguments, whose
             default values are shown above.  The max\_job\_errors limit specifies the number
             of work item errors allowed before forcibly terminating the job.  The
             max\_timeout\_retrys\_per\_workitem limit specifies the number of times each
             work item is retried in the event of a time-out.

           \item[$--$driver\_jvm\_args {[list]} ]
             This specifies extra JVM arguments to be provided to the Job Driver process. It is a blank-delimited 
             list of strings. Example: 
             \begin{verbatim}
     --driver_jvm_args -Xmx100M -Xms50M
             \end{verbatim}
             Note: When used as a CLI option, the list must usually be
             quoted to protect it from the shell.

           \item[$--$environment {[env vars]}] Blank-delimited list of environment variables and
             variable assignments. 
             Entries will be copied from the user's environment if just the variable name is
             specified, optionally with a final '*' for those with the same prefix.
             If specified, this is used for all DUCC processes in the job. Example:
             \begin{verbatim}
     --environment TERM=xterm DISPLAY=:1.0 LANG UIMA_*
             \end{verbatim}
             Additional entries may be copied from the user's environment based on the setting of
             {\em ducc.submit.environment.propagated}
             in the global DUCC configuration ducc.properties.
             \\Note: When used as a CLI option, the environment string must usually be
             quoted to protect it from the shell.

   
             The following cause special runtime behavior. 
             They are considered experimental and are not guaranteed 
             to be effective from release to release.
             
  \begin{enumerate}
  
    \item[DUCC\_USER\_CP\_PREPEND {[path-to-ducc-jars-and-classes]}  ]
    If specified, this path is used to supply the DUCC classes required for running
    the Job Driver and Job Process(es), normally set to \$DUCC\_HOME/lib/uima-ducc/users/*.
    
    \item[DUCC\_WORK\_ITEM\_PERFORMANCE {[true]}  ]
    If specified, DUCC will create subdirectory work-item-performance in the user 
    specified log directory and will place an individual json.gz file for each 
    work item successfully processed comprising its performance breakdown.
    
    \item[DUCC\_KEEP\_TEMPORARY\_DESCRIPTORS {[true]}  ]
    If specified, DUCC will not delete any temporary descriptors it creates
    (these are usually deleted when the process ends.)  DUCC may make a local file copy if the
    descriptor is in a jar, or if the filesystem is not shared.
    
  \end{enumerate}


           \item[$--$help ]
             Prints the usage text to the console. 

           \item[$--$jvm {[path-to-java]}  ]
             States the JVM to use. If not specified, the same JVM used by the Agents is used.  This is
             the full path to the JVM, not the JAVA\_HOME.
             Example: 
             \begin{verbatim}
     --jvm /share/jdk1.6/bin/java
             \end{verbatim}
             
           \item[$--$log\_directory {[path-to-log-directory]} ]
             This specifies the path to the directory for the user logs. 
             If not fully specified the path is made relative to the value of the {\em $--$working\_directory}.
             If omitted, the default is \$HOME/ducc/logs. 
             Example: 
             \begin{verbatim}
     --log_directory /home/bob
             \end{verbatim}
             Within this directory DUCC creates a sub-directory for each job, using the unique numerical 
             ID of the job. The format of the generated log file names as described
             \hyperref[chap:job-logs]{here}.
             \\Note: The {\em $--$log\_directory} specifies only the path to a directory where 
             logs are to be stored. In order to manage multiple processes running in multiple 
             machines, sub-directory and file names are generated by DUCC and may 
             not be directly specified. 

           \item[$--$process\_debug {[debug-port]}] Append JVM debug flags to the JVM
             arguments to start the Job Process in remote debug mode.  The remote process will start
             its debugger and attempt to contact the debugger (usually Eclipse) on the specified
             port.
             
           \item[$--$process\_deployments\_max {[integer]} ]
             This specifies the maximum number of Job Processes to deploy at any given time. If not 
             specified, DUCC will attempt to provide the largest number of processes within the 
             constraints of fair\_share scheduling and the amount of work remaining.
             in the job. Example:
             \begin{verbatim}
     --process_deployments_max 66
             \end{verbatim}

           \item[$--$process\_descriptor\_AE {[descriptor]}  ]
             This specifies the Analysis Engine descriptor to be deployed in the Job Processes. This 
             descriptor is a resource that is searched for in the filesystem or Java classpath as described 
             in the ~\hyperref[par:cli.submit.notes]{notes below}.
             It is mutually exclusive with {\em $--$process\_descriptor\_DD}.
             Example: 
             \begin{verbatim}
     --process_descriptor_AE /home/billy/resource/AE_foo.xml
             \end{verbatim}


           \item[$--$process\_descriptor\_AE\_overrides {[list]}  ]
             This specifies AE overrides. It is a whitespace-delimited list of name/value pairs. Example: 
             \begin{verbatim}
     --process_descriptor_AE_Overrides name1=value1 name2=value2
             \end{verbatim}

           \item[$--$process\_descriptor\_CC {[descriptor]}  ]
             This specifies the CAS Consumer descriptor to be deployed in the Job Processes. This 
             descriptor is a resource that is searched for in the filesystem or Java classpath as described 
             in the ~\hyperref[par:cli.submit.notes]{notes below}.
             It is mutually exclusive with {\em $--$process\_descriptor\_DD}.
             Example: 
             \begin{verbatim}
     --process_descriptor_CC /home/billy/resourceCCE_foo.xml
             \end{verbatim}

           \item[$--$process\_descriptor\_CC\_overrides {[list]}  ]
             This specifies CC overrides. It is a whitespace-delimited list of name/value pairs. Example: 
             \begin{verbatim}
     --process_descriptor_CC_overrides name1=value1 name2=value2
             \end{verbatim}
           
           \item[$--$process\_descriptor\_CM {[descriptor]} ]
             This specifies the CAS Multiplier descriptor to be deployed in the Job Processes. This 
             descriptor is a resource that is searched for in the filesystem or Java classpath as described 
             in the ~\hyperref[par:cli.submit.notes]{notes below}.
             It is mutually exclusive with {\em $--$process\_descriptor\_DD}.
             Example: 
             \begin{verbatim}             
     --process_descriptor_CM /home/billy/resource/CM_foo.xml
             \end{verbatim}

           \item[$--$process\_descriptor\_CM\_overrides {[list]}  ]
             This specifies CM overrides. It is a whitespace-delimited list of name/value pairs. Example: 
             \begin{verbatim}
     --process_descriptor_CM_overrides name1=value1 name2=value2
             \end{verbatim}
           
           \item[$--$process\_descriptor\_DD {[descriptor]}  ]
             This specifies a UIMA Deployment Descriptor for the job processes for DD-style jobs. 
             This is mutually exclusive with {\em $--$process\_descriptor\_AE}, {\em $--$process\_descriptor\_CM}, 
             and {\em $--$process\_descriptor\_CC}. This 
             descriptor is a resource that is searched for in the filesystem or Java classpath as described 
             in the ~\hyperref[par:cli.submit.notes]{notes below}.
             Example:
             \begin{verbatim}
     --process_descriptor_DD /home/billy/resource/DD_foo.xml
             \end{verbatim}
             Alias: $--$process\_DD

           \item[$--$process\_failures\_limit {[integer]} ]
             This specifies the maximum number of individual Job Process (JP) failures allowed
             before killing the job. The default is twenty(20). If this limit is exceeded over the lifetime 
             of a job DUCC terminates the entire job. 
             Example:
             \begin{verbatim}
     --process_failures_limit 23
             \end{verbatim}
                          
           \item[$--$process\_initialization\_failures\_cap {[integer]} ] This specifies the maximum
             number of failures during a UIMA process's initialization phase.  If the number is
             exceeded the system will allow processes which are already running to continue, but
             will assign no new processes to the job.  The default is ninety-nine(99). Example:
             \begin{verbatim}
     --process_initialization_failures_cap 62
             \end{verbatim}
             Note that the job is NOT killed if there are processes that have passed initialization and are 
             running. If this limit is reached, the only action is to not start new processes for the job. 

           \item[$--$process\_initialization\_time\_max {[integer]}] This is the maximum time in minutes that 
             a process is allowed to remain in the ``initializing'' state, before DUCC terminates it.  The 
             error counts as an initialization error towards the initialization failure cap.

           \item[$--$process\_jvm\_args {[list]} ] This specifies additional arguments to be passed to
             all of the job processes as a blank-delimited list of strings. Example:
             \begin{verbatim}
     --process_jvm_args -Xmx400M -Xms100M
             \end{verbatim}
             Note: When used as a CLI option, the arguments must usually be
             quoted to protect them from the shell.
                          
           \item[$--$process\_memory\_size {[size]} ] This specifies the maximum amount of RAM in GB
             to be allocated to each Job Process.  This value is used by the Resource Manager to
             allocate resources.

           \item[$--$process\_per\_item\_time\_max {[integer]} ] This specifies the maximum time in
             minutes that the Job Driver will wait for a Job Processes to process a CAS. If a
             timeout occurs the process is terminated and the CAS marked in error (not retried). If
             not specified, the default is 24 hours. Example:
             \begin{verbatim}
     --process_per_item_time_max 60
             \end{verbatim}
             
           \item[$--$process\_error\_window\_threshold {[integer]} ] This specifies an upper bound for number 
             of process errors a JP can tolerate. When this threshold is reached, the JP will terminate. If 
             not specified, the default is 1 which means terminate on first process error. Example:
             \begin{verbatim}
     --process_error_window_threshold 2
             \end{verbatim}  
             
           \item[$--$process\_error\_window\_size {[integer]} ] This specifies a window the JP error handler 
             uses to determine if it should terminate on process error. When a defined number of errors 
             in a given window is reached, the JP will terminate. If not specified, the default is 1 which means 
             terminate on first process error. This property is used together with {\em $--$process\_error\_window\_threshold}. Example:
             \begin{verbatim}
     --process_error_window_size 10
             \end{verbatim}   
             
           \item[$--$process\_pipeline\_count {[integer]} ] This specifies the number of pipelines per
             process to be deployed, i.e. the number of work-items each JP will process simultaneously. 
             It is used by the Resource Manager to determine how many
             processes are needed, by the Job Process wrapper to determine how many threads to
             spawn, and by the Job Driver to determine how many CASs to dispatch. If not specified,
             the default is 4. Example:
             \begin{verbatim}
     --process_pipeline_count 7
             \end{verbatim}
             Alias:  $--$process\_thread\_count
             
           \item[$--$scheduling\_class {[classname]} ] This specifies the name of the scheduling class
             the used to determine the resource allocation for each process. The names of the
             classes are installation dependent. 
             If not specified, the FAIR\_SHARE default is taken from the site class definitions file
             described \hyperref[subsubsec:class.configuration]{here.} 
             Example:
             \begin{verbatim}
     --scheduling_class normal
             \end{verbatim}

           \item[$--$service\_dependency{[list]}] This specifies a blank-delimited list of services the job
             processes are dependent upon. Service dependencies are discussed in detail
             \hyperref[sec:service.endpoints]{here}. Example:
             \begin{verbatim}
     --service_dependency UIMA-AS:Service1:tcp:host1:61616 UIMA-AS:Service2:tcp:host2:123
             \end{verbatim}
           
           \item[$--$specification, $-$f {[file]}  ]
             All the parameters used to submit a job may be placed in a standard Java properties file. 
             This file may then be used to submit the job (rather than providing all the parameters 
             directory to submit). The leading $--$ is omitted from the keywords.

             For example, 
             \begin{verbatim}
     ducc_submit --specification job.props 
     ducc_submit -f job.props 
             \end{verbatim}
             where job.props contains: 
\begin{verbatim}
working_directory                   = /home/bob/projects/ducc/ducc_test/test/bin 
process_failures_limit              = 20 
driver_descriptor_CR                = org.apache.uima.ducc.test.randomsleep.FixedSleepCR 
environment                         = AE_INIT_TIME=10000 UIMA LD_LIBRARY_PATH=/a/bogus/path
log_directory                       = /home/bob/ducc/logs/ 
process_pipeline_count              = 1 
driver_descriptor_CR_overrides      = jobfile:../simple/jobs/1.job compression:10 
process_initialization_failures_cap = 99 
process_per_item_time_max           = 60 
driver_jvm_args                     = -Xmx500M 
process_descriptor_AE               = org.apache.uima.ducc.test.randomsleep.FixedSleepAE 
classpath                           = /home/bob/duccapps/ducky_process.jar 
description                         = ../simple/jobs/1.job[AE] 
process_jvm_args                    = -Xmx100M -DdefaultBrokerURL=tcp://localhost:61616 
scheduling_class                    = normal 
process_memory_size                 = 15 
\end{verbatim}
             Note that properties in a specification file may be overridden by other command-line
             parameters, as discussed \hyperref[chap:cli]{here}.

           \item[$--$suppress\_console\_log] If specified, suppress creation of the log files that 
             normally hold the redirected stdout and stderr.

           \item[$--$timestamp ]
             If specified, messages from the submit process are timestamped. This is intended primarily 
             for use with a monitor with --wait\_for\_completion. 

           \item[$--$wait\_for\_completion ]             
             If specified, the submit command monitors the job and prints periodic
             state and progress information to the console.  When the job completes, the monitor
             is terminated and the submit command returns.  If the command is interrupted, e.g. with CTRL-C,
             the job will not be canceled unless {\em $--$cancel\_on\_interrupt} is also specified.
             
           \item[$--$working\_directory ]             
             This specifies the working directory to be set by the Job Driver and Job Process processes. 
             If not specified, the current directory is used.
  \end{description}
             
  \paragraph{Notes:}
  \phantomsection\label{par:cli.submit.notes}
  When searching for UIMA XML resource files such as descriptors, DUCC searches either the 
  filesystem or Java classpath according to the following rules:
  \begin{enumerate}
    \item If the resource ends in .xml it is assumed the resource is a file in the filesystem 
      and the path is either an absolute path or a path relative to the specified working directory. [by location]
    \item If the resource does not end in .xml, it is assumed the resource is in the Java
      classpath. DUCC creates a resource name by replacing the "." separators with "/" and appending ".xml". [by name]
  \end{enumerate}
