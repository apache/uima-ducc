% 
% Licensed to the Apache Software Foundation (ASF) under one
% or more contributor license agreements.  See the NOTICE file
% distributed with this work for additional information
% regarding copyright ownership.  The ASF licenses this file
% to you under the Apache License, Version 2.0 (the
% "License"); you may not use this file except in compliance
% with the License.  You may obtain a copy of the License at
% 
%   http://www.apache.org/licenses/LICENSE-2.0
% 
% Unless required by applicable law or agreed to in writing,
% software distributed under the License is distributed on an
% "AS IS" BASIS, WITHOUT WARRANTIES OR CONDITIONS OF ANY
% KIND, either express or implied.  See the License for the
% specific language governing permissions and limitations
% under the License.
% 
% Create well-known link to this spot for HTML version
\ifpdf
\else
\HCode{<a name='DUCC_ERROR_HANDLER'></a>}
\fi
\chapter{Job Error Handler}
\label{chap:job-error-handler}

\paragraph {Overview} The {\em ErrorHandler} allows for the per Job customized handling of runtime anomalies.

\paragraph {Operation} The Job Driver comes with a built-in {\em ErrorHandler}.  Its purpose is to 
instruct the Job Driver on what action(s) to take when a work item error is encountered.

The {\em ErrorHandler} implements {\em org.apache.uima.ducc.IErrorHandler}.

\begin{verbatim}
public interface IErrorHandler {
	public void initialize(String initializationData);
	public IErrorHandlerDirective handle(String serializedCAS, Object userException);
}

public interface IErrorHandlerDirective {
	public boolean isKillJob();
	public boolean isKillProcess();
	public boolean isKillWorkItem();
}
\end{verbatim}

By default, the {\em ErrorHandler} returned directive:
\begin{enumerate}
\item returns isKillJob == false, unless the number of work items errors exceeds 15 for the Job
\item returns isKillProcess == true
\item returns isKillWorkItem == true
\end{enumerate}

\paragraph {Programmability} The Job Driver built-in (or custom) {\em ErrorHandler} behavior can be modified
according to the {\em driver\_exception\_handler\_arguments} string in the Job Specification.  
Currently recognized are:

\begin{description}
\item[max\_job\_errors=E], where E is the maximum number of work item errors tolerated before terminating the Job.  Default is 15.
\item[max\_timeout\_retrys\_per\_workitem=R], where R is the maximum number of work item timeouts tolerated before the work item is considered an error.  Default is 0.
\end{description}

\paragraph {Replacement} The {\em ErrorHandler} can be replaced.  The steps necessary are:
\begin{enumerate}
\item Create a new org.myOrg.myProject.MyErrorHandler.class that implements {\em org.apache.uima.ducc.IErrorHandler}, which is located in the uima-ducc-user.jar.
\item Put your replacement class in your Job Specification classpath.
\item Put your replacement class name as the value for your Job Specification driver\_exception\_handler.
\end{enumerate}
