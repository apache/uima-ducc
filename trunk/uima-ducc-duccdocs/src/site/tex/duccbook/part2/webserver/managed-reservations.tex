% 
% Licensed to the Apache Software Foundation (ASF) under one
% or more contributor license agreements.  See the NOTICE file
% distributed with this work for additional information
% regarding copyright ownership.  The ASF licenses this file
% to you under the Apache License, Version 2.0 (the
% "License"); you may not use this file except in compliance
% with the License.  You may obtain a copy of the License at
% 
%   http://www.apache.org/licenses/LICENSE-2.0
% 
% Unless required by applicable law or agreed to in writing,
% software distributed under the License is distributed on an
% "AS IS" BASIS, WITHOUT WARRANTIES OR CONDITIONS OF ANY
% KIND, either express or implied.  See the License for the
% specific language governing permissions and limitations
% under the License.
% 
\section{Managed Reservation Details Page}
\label{sec:ws-managed-reservation-details}

This page shows details of the processes which run in a managed reservation.  The
information is divided between three tabs:

   \begin{description}
       \item[Processes] This tab contains details on all the processes contained in the
         reserved space.
       \item[Specification] This tab shows the specification for the process.
       \item[Files] This tab shows the files in the log directory.
   \end{description}  

   \subsection{Processes}
   \label{sec:ws-manres-processes}

   The processes page contains the following columns:
   \begin{description}
      \item[Id] \hfill \\
        This is the {\DUCC}-assigned numeric id of the process.  This format of this
        id is two numbers:
\begin{verbatim}
    RESID.SHAREID
\end{verbatim}
        Here, the {\em RESID} is the reservation ID.  The {\em SHAREID} is the 
        share ID assigned by the Resource Manager.  Together these form a unique
        ID for each process that runs in the reservation.
        
        Note: The current version of {\DUCC} supports only one process per managed
        reservation.  Future versions are expected to support multiple processes
        within a single managed reservation.
        
      \item[Log] \hfill \\
        This is the log name for the process. It is hyperlinked to the log itself.
        
      \item[Log Size] \hfill \\
        This is the size of the log in MB. If you find you have trouble viewing the log
        from the web server it could be because it is too big to view in the browser.
        
      \item[Host Name] \hfill \\
        This is the name of the host where the process is running (or ran).
        
      \item[PID] \hfill \\
        This is the Unix process ID (PID) of the process.
        
      \item[State Scheduler] \hfill \\
        This shows the Resource Manager state of the job. It is one of:
        
        \begin{description}
            \item[Allocated] - The resource manager has allocated resources for this process on the host.
            \item[Deallocated] - The resource manager has deallocated resources for this process on the host.
        \end{description}
        
      \item[Reason Scheduler or Extraordinary Status] \hfill \\
        These are the same as for the \hyperref[itm:job-details-sched]{job details.}

      \item[State Agent] \hfill \\
        These are the same as for the \hyperref[itm:job-details-state]{job details.}

      \item[Reason Agent] \hfill \\
        These are the same as for the \hyperref[itm:job-details-agent]{job details.}

      \item[Exit] \hfill \\
        The process exit code or signal.

      \item[Time Run] \hfill \\
        The current duration of the reservation, or total duration if it has 
        terminated.
      
      \item[PgIn] \hfill \\
        This is the number of page-in events on behalf of the process.

      \item[Swap] \hfill \\
        This is the amount of swap space on the machine being consumed by the process.
      
      \item[\%CPU] \hfill \\
        Current CPU percent consumed by the process.  This will be $>$ 100\% on 
        multi-core systems if more than one core is being used.  Each core contributes
        up to 100\% CPU, so, for example, on a 16-core machine, this can be as high
        as 1600\%.
      
      \item[RSS] \hfill \\
        The amount of real memory being consumed by the process (Resident Storage Size)

   \end{description}

   \subsection{Specification}
   \label{sec:ws-service-specification}
   This tab shows the full managed reservation specification in the form of a Java Properties
   file.  This will include all the parameters specified by the user, plus those
   filled in by {\DUCC}.
        
   \subsection{Files}
   This tab shows the files in the log directory.
        