% 
% Licensed to the Apache Software Foundation (ASF) under one
% or more contributor license agreements.  See the NOTICE file
% distributed with this work for additional information
% regarding copyright ownership.  The ASF licenses this file
% to you under the Apache License, Version 2.0 (the
% "License"); you may not use this file except in compliance
% with the License.  You may obtain a copy of the License at
% 
%   http://www.apache.org/licenses/LICENSE-2.0
% 
% Unless required by applicable law or agreed to in writing,
% software distributed under the License is distributed on an
% "AS IS" BASIS, WITHOUT WARRANTIES OR CONDITIONS OF ANY
% KIND, either express or implied.  See the License for the
% specific language governing permissions and limitations
% under the License.
% 

\let\oldv\verbatim
\let\oldendv\endverbatim

\def\greybatim{\par\setbox0\vbox\bgroup\oldv}
\def\endgreybatim{\oldendv\egroup\fboxsep0pt \noindent\colorbox[gray]{0.8}{\usebox0}\par}

% Create well-known link to this spot for HTML version
\ifpdf
\else
\HCode{<a name='DUCC_RELIABLE'></a>}
\fi

\chapter{Reliable DUCC}
\label{chap:reliable}

\section{Introduction}
    DUCC can be configured to run reliably by having multiple head nodes,
    comprising one {\em master} and one or more {\em backup} head nodes.
    DUCC exploits Linux {\em keepalived} virtual IP addressing to enable
    this capability.
    
    The advantages are that if the {\em master} DUCC host becomes
    unusable, the {\em backup} DUCC can take over seamlessly
    such that active distributed Jobs, Reservations, Managed Reservations 
    and Services continue uninterrupted.  Take over also facilitates
    continued acceptance of new submissions and monitoring of new and
    existing submissions without interruption.
        
\section{Configuring Host Machines}    
    See {\em Configuring Simple Virtual IP Address Failover Using Keepalived} 
    which can be found at this web address: 
    \url{https://docs.oracle.com/cd/E37670_01/E41138/html/section_uxg_lzh_nr.html}.

% ========================================
\begin{minipage}{\textwidth}
% ========================================

	Sample /etc/sysconfig/keepalived\\

	\begin{greybatim}
	
# Options for keepalived. See `keepalived --help' output and keepalived(8) and
# keepalived.conf(5) man pages for a list of all options. Here are the most
# common ones :
#
# --vrrp               -P    Only run with VRRP subsystem.
# --check              -C    Only run with Health-checker subsystem.
# --dont-release-vrrp  -V    Dont remove VRRP VIPs & VROUTEs on daemon stop.
# --dont-release-ipvs  -I    Dont remove IPVS topology on daemon stop.
# --dump-conf          -d    Dump the configuration data.
# --log-detail         -D    Detailed log messages.
# --log-facility       -S    0-7 Set local syslog facility (default=LOG_DAEMON)
#

KEEPALIVED_OPTIONS="-D -P"

	\end{greybatim}
	
	\medskip
	IMPORTANT: Specify {\em -P} to enable keepalived IP takeover.
	
	\medskip
	See example here:
	\begin{verbatim}
    examples/reliable/etc/sysconfig/keepalived
   	\end{verbatim}
	
% ========================================	
\end{minipage}
% ========================================

% ========================================
\begin{minipage}{\textwidth}
% ========================================

	Sample MASTER /etc/keepalived/keepalived.conf\\
	
	\begin{greybatim}
	
! Configuration File for keepalived

global_defs {
	script_user ducc 
	enable_script_security
}

vrrp_script chk_ducc {
  script       "/etc/keepalived/keepalived_evaluator.py"
  interval 5   # check every 5 seconds
  fall 2       # require 2 failures for KO
  rise 1       # require 1 successes for OK
}

vrrp_instance VI_1 {
    state MASTER
    interface bond0
    virtual_router_id 51
    priority 100
    advert_int 1
    authentication {
        auth_type PASS
        auth_pass 1111
    }
    virtual_ipaddress {
        9.59.193.8
    }
    track_script {
    	chk_ducc
  	}
}

   	\end{greybatim}

% ========================================	
\end{minipage}
% ========================================

% ========================================
\begin{minipage}{\textwidth}
% ========================================

	Sample BACKUP /etc/keepalived/keepalived.conf\\
	
	\begin{greybatim}
	
! Configuration File for keepalived

global_defs {
	script_user ducc 
	enable_script_security
}

vrrp_script chk_ducc {
  script       "/etc/keepalived/keepalived_evaluator.py"
  interval 5   # check every 5 seconds
  fall 2       # require 2 failures for KO
  rise 1       # require 1 successes for OK
}

vrrp_instance VI_1 {
    state BACKUP
    interface bond0
    virtual_router_id 51
    priority 100
    advert_int 1
    authentication {
        auth_type PASS
        auth_pass 1111
    }
    virtual_ipaddress {
        9.59.193.8
    }
    track_script {
    	chk_ducc
  	}
}

   	\end{greybatim}
   	
% ========================================	
\end{minipage}
% ========================================

	\medskip
	In this example, the IP address 9.59.193.8 is managed by the keepalived daemons.
   	It is assigned to the current DUCC master.
   	If can float to another host when the presently assigned master is no longer viable.

	\medskip
	See example here:
	\begin{verbatim}
    examples/reliable/etc/keepalived/keepalived.conf
   	\end{verbatim}

% ========================================
\begin{minipage}{\textwidth}
% ========================================

	Sample /etc/keepalived/keepalived\_evaluator.py\\
	
	\begin{greybatim}
	
#!/usr/bin/env python
# -----------------------------------------------------------------------
# Licensed to the Apache Software Foundation (ASF) under one
# or more contributor license agreements.  See the NOTICE file
# distributed with this work for additional information
# regarding copyright ownership.  The ASF licenses this file
# to you under the Apache License, Version 2.0 (the
# "License"); you may not use this file except in compliance
# with the License.  You may obtain a copy of the License at
#
#     http://www.apache.org/licenses/LICENSE-2.0
#
# Unless required by applicable law or agreed to in writing,
# software distributed under the License is distributed on an
# "AS IS" BASIS, WITHOUT WARRANTIES OR CONDITIONS OF ANY
# KIND, either express or implied.  See the License for the
# specific language governing permissions and limitations
# under the License.
# -----------------------------------------------------------------------

import sys

version_min = [2, 7]
version_info = sys.version_info
version_error = False
if(version_info[0] < version_min[0]):
    version_error = True
elif(version_info[0] == version_min[0]):
    if(version_info[1] < version_min[1]):
        version_error = True
if(version_error):
    print('Python minimum requirement is version '+str(version_min[0])+'.'+str(version_min[1]))
    sys.exit(1)

import os
import subprocess
import traceback

ducc_runtime = '/home/ducc/ducc_runtime'
ducc_status = os.path.join(ducc_runtime,'bin/ducc_status')

state_dir = fpath = __file__.split('/')[0]

   	\end{greybatim}
   	
% ========================================	
\end{minipage}
% ========================================

% ========================================
\begin{minipage}{\textwidth}
% ========================================

	\begin{greybatim}
	
class Keepalived_Evaluator():
    
    flag_error = True
    flag_debug = False
    flag_info = True
    
    def __init__(self):
        pass
    
    def _cn(self):
        return self.__class__.__name__

    def _mn(self):
        return traceback.extract_stack(None,2)[0][2]
    
    def error(self,mn,text):
        if(self.flag_error):
            print mn+' '+text
    
    def debug(self,mn,text):
        if(self.flag_debug):
            print mn+' '+text
            
    def info(self,mn,text):
        if(self.flag_info):
            print mn+' '+text
    
    def do_cmd(self,cmd):
        mn = self._mn()
        p = subprocess.Popen(cmd, stdout=subprocess.PIPE, stderr=subprocess.PIPE)
        out, err = p.communicate()
        text = 'out:'+str(out)
        self.debug(mn,text)
        text = 'err:'+str(err)
        self.debug(mn,text)
        return out,err
    
    def get_ducc_status(self):
        mn = self._mn()
        cmd = [ducc_status,'--target','localhost']
        out, err = self.do_cmd(cmd)
        if('down=0' in out):
            status = 0
        else:
            status = 1
        text = str(status)
        self.debug(mn,text)
        return status
    
   	\end{greybatim}
   	
% ========================================	
\end{minipage}
% ========================================

% ========================================
\begin{minipage}{\textwidth}
% ========================================

	\begin{greybatim}
	
    def main(self, argv):
        mn = self._mn()
        rc = 0
        try:
            rc = self.get_ducc_status()
        except Exception as e:
            lines = traceback.format_exc().splitlines()
            for line in lines:
                text = line
                self.error(mn,text)
            rc = 1
        text = 'rc='+str(rc)
        self.info(mn,text)
        sys.exit(rc)

if __name__ == "__main__":
    instance = Keepalived_Evaluator()
    instance.main(sys.argv[1:])

   	\end{greybatim}
   	
% ========================================	
\end{minipage}
% ========================================

	\medskip
   	The script 
   	\begin{verbatim}
    /etc/keepalived/keepalived_evaluator.py
   	\end{verbatim}
   	determines the viability of DUCC on the present host, returning 0 when viable or 1 otherwise.
   	Note that in this example the value of {\em ``ducc\_runtime''} is hard-coded.
 
	Linux Commands
	
	Starting keepalived
	
    \begin{verbatim}
    > sudo service keepalived start
    Starting keepalived:                                       [  OK  ]
   	\end{verbatim}
   	
   	Querying keepalived
	
    \begin{verbatim}
    > /sbin/ip addr show dev bond0
    8: bond0: <BROADCAST,MULTICAST,MASTER,UP,LOWER_UP> mtu 1500 qdisc noqueue state UP group default qlen 1000
    link/ether 0c:c4:7a:bc:e6:7a brd ff:ff:ff:ff:ff:ff
    inet 10.185.47.235/23 brd 10.185.47.255 scope global bond0
       valid_lft forever preferred_lft forever
    inet 9.59.193.235/23 brd 9.59.193.255 scope global bond0:0
       valid_lft forever preferred_lft forever
    inet 9.59.193.8/32 scope global bond0
       valid_lft forever preferred_lft forever
    inet6 fe80::ec4:7aff:febc:e67a/64 scope link 
       valid_lft forever preferred_lft forever
   	\end{verbatim}

	Stopping keepalived
	
    \begin{verbatim}
    > sudo service keepalived stop
    Stopping keepalived: 
   	\end{verbatim}

\section{Configuring DUCC}  
    To configure DUCC to run reliable, one required property must
    be configured in the {\em site.ducc.properties} file.  Example:
    
    \begin{verbatim}
	ducc.head = 9.59.193.8
   	\end{verbatim}
    
    Use the virtual IP address configured for your host machines keepalived. 
    Use of the DNS name is also supported. 
    
\section{Webserver}

	Webserver for Master

	The {\em master} DUCC Webserver will display all pages normally with additional
	information in the heading upper left:
	
	reliable: \textbf{master}
   	
	Webserver for Backup
	
	The {\em backup} DUCC Webserver will display some pages normally with additional
	information in the heading upper left:
	
	\underline{reliable}: \textbf{backup}
   	
   	Hovering over \underline{reliable} will yield the following information:
   	{\em Click to visit master}
   	
   	Several pages will display the following information (or similar):
   	
   	\begin{verbatim}
	no data - not master
   	\end{verbatim}

\section{Database}

	Configure the database to be on a separate machine from the reliable DUCC head nodes.
	In {\em site.ducc.properties} specify:
	
	\begin{verbatim}
	# Database location
    ducc.database.host = dbhost123
    ducc.database.jmx.host = dbhost123
    ducc.database.automanage = false
   	\end{verbatim}
   	
   	The existing administrator commands {\em start\_ducc} and {\em stop\_ducc} will
   	honor the value specified for {\em ducc.database.automanage} in {\em site.ducc.properties}.
